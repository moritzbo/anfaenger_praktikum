\section{Diskussion}

Aus den Messdaten des Emissionsspektrums von Kupfer lassen sich zwei K-Linien eindeutig feststellen. Auch das Bremsspektrum enthält keine großen und bedenklichen Ausreißer.
Nicht betrachtet wurde dabei ein möglicher Winkelfehler durch die Bragg-Bedinung welcher im Anschluss zu Fehlern führt. Die bestimmten Energien
der $K_{\alpha}$- und $K_{\beta}$-Linie \eqref{eqn:q1} sind allerdings sehr nah am Literaturwert und weichen nur um $\SI{0.058}{\percent}$ bis $\SI{0.081}{\percent}$ ab.
\\
\newline
Bei der Betrachtung der Transmission wurden die Zählraten mit einer Poissonverteilung beschrieben. Dadurch entsteht automatisch ein unterschiedlich großer Fehler
bei einzelnen Messungen welcher sich nachher in den Ergebnissen wiederspiegelt. Die Wellenlängen sind unter der Annahme von exakten Winkeln auch genau bestimmbar und sorgen für
keine Messunsicherheit. Das Diagramm \ref{fig:plot2} mit der Ausgleichsgeraden passt sehr gut zu den Messwerten und somit scheinen die bestimmten Parameter \eqref{eqn:gerade} auch plausibel.
\\
\newline
Die Compton-Wellenlänge wurde mit den gemessenen Impulsen \ref{tab:idky} bestimmt, dabei sind diese ohne statischen Fehler angegeben welches eine mögliche Fehlerquelle darstellt.
Durch die Zuordnung der Intensitäten mit den Wellenlängen der Geradengleichung kann es zu Ungenauigkeiten kommen, dies ist allerdings genauer als ein einfaches Ablesen am Diagramm.
Der experimentelle Wert für die Compton-Wellenlänge besitzt hier eine Abweichung von $\SI{54.89}{\percent}$ dies ist sehr hoch und kann eventuell an dem nicht mit einbezogenem Winkelfehler liegen.
Eine Totzeitkorrektur ist nur bei größeren Zählraten nötig, aufgrund der kleinen Totzeit und der Konvergenz gegen 1 der Gleichung \eqref{eqn:totzeit} für kleine Zählraten. Da die Abweichung hier allerdings sehr groß erscheint
ist diese eventuell doch sinnvoll.

