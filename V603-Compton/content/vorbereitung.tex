\section{Vorbereitung}

Vor der Versuchsdurchführung wurden in der Literatur die Energiewerte der $K_{\alpha}$- und $K_{\beta}$ Linien herausgesucht. Daraus lassen sich die zugehörigen Wellenlängen, sowie der Winkel
$\alpha$ bei Beugung an einem LiF- (Lithiumfluorid) Kristall angeben. Die Wellenlänge lässt sich aus der Gleichung \eqref{eqn:photoneneq} bestimmen und aus der Bragg-Bedingung \eqref{eqn:lambda} folgt dann durch Gleichung \eqref{eqn:winkelmitlambda} der Winkel $\alpha$.
Im folgenden werden alle Naturkonstanten auf die hingewiesen wird, der Literatur \cite{Naturkonstanten} entnommen.
Die Literaturwerte der Energien der daraus bestimmten Parameter \cite{database} sind in der folgenden Tabelle \ref{tab:vorbereitung} angegeben.
\begin{table}
\centering
\caption{Literaturwerte \cite{database}.}
\label{tab:vorbereitung}
\begin{tabular}{c c c c}
    \toprule
    Linie & Energie $E$[$\si{\kilo\electronvolt}$] & Wellenlänge $\lambda$[$\si{\meter}$] & Winkel $\alpha$[$\textdegree$]\\
    $K_{\alpha}$    &  $\SI{8.048}{}$ & $\SI{1.541e-10}{}$  & $22.49$ \\
    $K_{\beta}$   &   $\SI{8.907}{}$  & $\SI{1.392e-10}{}$  & $20.22$ \\
    \midrule
\end{tabular}
\end{table}
Um spätere Ergebnisse der Compton-Wellenlänge zu vergleichen kann zunächst der theoretische Wert errechnet werden. Hierbei wird Gleichung
\eqref{eqn:comptonwavelength} verwendet und durch Einsetzen der Naturkonstanten ergibt sich.
\begin{equation}
\label{eqn:comptontheoriewert}
\lambda_{c} = \SI{2.42631e-12}{\meter}
\end{equation}