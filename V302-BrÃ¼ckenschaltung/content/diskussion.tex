\section{Diskussion}
Zum Vergleich werden im Folgenden Literaturwerte aus der Tabelle \ref{tab:lit} zurate gezogen.
Die Bestimmung eines ohmschen Widerstands mit der Wheatstonesches Brücke stellt sich als ziemlich genau heraus. Die einzigen Fehlerquellen sind
hier die verbauten Referenzteile welche relativ gering sind. Die prozentuale Abweichung des Mittelwerts vom Literaturwert $R_{10} =\SI{239}{\ohm}$ beträgt 
$\increment R_{10} = \SI{1.7(4)}{\percent}$.
\\
\newline
Auch bei der Kapazitätsmessung sind keine großen Abweichungen zur Literatur zu erwarten, da auch hier die Fehlerquellen gering sind. Die berechneten Werte $R_{9}$ und $C_{9}$
haben die folgenden prozentualen Abweichungen zu den Literaturwerten. 
\begin{align*}
\increment R_{9} &= \SI{3.3(29)}{\percent} \\
\increment C_{9} &= \SI{0.9(5)}{\percent}
\end{align*}
Die etwas größere Abweichung beim Widerstand im Vergleich zur vorherigen Messung entsteht vor allem durch die nur einmalige Messung. Ein Mittelwert über mehrere Messreihen liefert hier ein genaueres Ergebnis.
\\
\newline
Bei der Messung der verlustbehafteten Induktivität ist die Abweichung zum Literaturwert nun deutlich erkennbar. Problematisch war hierbei, dass die Brückenspannung hier noch einen erheblich
großen Wert über $\SI{0}{\volt}$ besaß. Aus diesem Grund wurde hier die Frequenz der Speisespannung angepasst um ein genaueres Ergebnis zu erzielen. Die optimale Frequenz könnte diesen Fehler
allerdings deutlich reduzieren. Alle Abweichungen von den Literaturwerten, sowie zwischen Induktivitätsmessbrücke und Maxwell-Brücke sind in Tabelle \ref{tab:hello} dargelegt.
\\
\newline
Für die Frequenzabhängige Brückenspannung wurde eine passende Theoriekurve ermittelt, welche gut mit den Messwerten übereinstimmt. Der geringe Klirrfaktor $k$ bestätigt ebenfalls
die hohe Genauigkeit von Theoriekurve und Messwerten im Minimum von Diagramm \ref{fig:plot1} Die Brückenspannung wurde bei einer Frequenz von $\SI{540}{\hertz}$ sehr klein mit nur noch kleinen Oberwellenspannungen. Es lässt 
sich somit sagen, dass der verwendete Generator nur einen geringen Anteil an Oberwellen im Verhältnis zur Grundwelle erzeugt.