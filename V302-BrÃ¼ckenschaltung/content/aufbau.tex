\section{Aufbau und Durchführung}

Für die Versuchsdurchführung wird grundsätzlich ein Spannungsmessgerät benötigt, dabei wurde ein digitales Oszilloskop verwendet. Die Aufbauten variieren je nach Messgröße und sind
im Folgenden unterteilt. Die sinusförmige Speisespannung wurde zur Optimierung einiger Brückenspannungswerte in den einzelnen Versuchen geändert. Zur Regelung von Widerständen wird ein Potentiometer mit $\SI{1}{\kilo\ohm}$ Gesamtwiderstand 
genutzt.

\subsection{Wheatstonesche Brücke}
\label{sec:weedyo}
Zunächst wird eine Schaltung gemäß Abbildung \ref{fig:abb4} aufgebaut. Dabei werden nur ohmsche Widerstände verwendet, wodurch es nur eine Abgleichbedingung gibt. Die Speisespannung $U_{S}$ wird mit einer Frequenz von
$\SI{80}{\hertz}$ und einer Spannungsamplitude von $\SI{1}{\volt}$ betrieben. Der Wiederstand mit dem Wert $10$ soll nun gemessen werden. Dazu wird ein Referenzwiderstand $R_{2}$ gewählt und anhand eines Potentiometers die Widerstände 
$R_{3}$ und $R_{4}$ so eingestellt, dass die Brückenspannung am Oszilloskop nahezu verschwindet. Dieser Vorgang wird für einen anderen Referenzwiderstand $R_{2}$ wiederholt. 
\\
\newline
Die Toleranz des Referenzwiderstands beträgt hierbei $\pm\SI{0.2}{\percent}$ und das Verhältnis $R_{3}$/$R_{4}$ zeigt unsystematische Abweichung von $\pm\SI{0.5}{\percent}$.
Folgende Werte aus Tabelle \ref{tab:weed} wurden bei erfüllter Abgleichbedingung gefunden.

\subsection{Kapazitätsmessbrücke}
\label{sec:nocapyo}
Die Schaltung wird wie in Abbildung \ref{fig:abb6} dargestellt aufgebaut. Hier soll ein Kondensator mit dielektrischen Verlusten gemessen werden, also wird ein Widerstand $R_{2}$ benötigt. Dieser wird wieder über einen
zweiten Potentiometer realisiert. Der zu messende Kondensator mit 
Verlusten hat den Wert $9$ und wird über die Regelung der Widerstände $R_{2}$, $R_{3}$ und $R_{4}$ ermittelt. Hierzu werden die Widerstände wieder so lange angepasst, bis die Brückenspannung am Oszilloskop verschwindet.
Die sinusförmige Speisespannung ist hier immer noch bei $\SI{80}{\hertz}$ mit gleicher Amplitude. Die Kapazität $C_{2}$ wurde hierbei festgelegt mit einer Kapazität von $\SI{399}{\nano\farad}$.
\\
\newline
Die Toleranz des Widerstands $R_{2}$ ist hier $\pm\SI{3}{\percent}$ und die der Referenzkapazität $C_{2}$ beträgt $\pm\SI{0.2}{\percent}$.
In der Tabelle \ref{tab:nocap} sind die gemessenen Daten angegeben.

\subsection{Induktivitätsmessbrücke}
\label{sec:spuleman}
Um die Induktivität einer Spule zu bestimmen wird eine Schaltung, wie in Abbildung \ref{??} gezeigt, aufgebaut. Dieser Aufbau ist identisch mit dem vorherigen, nur das hier die verlustbehafteten Kondensatoren mit verlustbehafteten Induktivitäten
ersetzt werden. Wie zuvor werden die drei Widerstände $R_{2}$, $R_{3}$ und $R_{4}$ so angepasst, dass die Brückenspannung verschwindet. Alle Widerstände und Induktivitäten werden anschließend notiert. Die sinusförmige Speisespannung $U_{S}$
wurde hierbei auf $\SI{1000}{\hertz}$ erhöht, damit eine geringere Brückenspannung erzielt werden konnte.
Der verwendete Referenzwert $L_{2}$ beträgt hier $\SI{27.5}{\milli\henry}$. 
\\
\newline
Die Toleranz des Widerstands $R_{2}$ ist $\pm\SI{3}{\percent}$ und die der Referenzinduktivität $L_{2}$ beträgt $\pm\SI{0.2}{\percent}$.
Alle Messwerte sind in Tabelle \ref{tab:spuleyo} notiert.

\subsection{Maxwell-Brücke}
Mit der Maxwell-Brücke lässt sich auf eine andere Art und Weise die Induktivität $L$ bestimmen. Um die vorherigen Ergebnisse überprüfen zu können wird hier die selbe Induktivität $L_{x}$ mit Wert 17 gemessen. Die Widerstände $R_{3}$ und $R_{4}$ sind
jetzt unabhängig voneinander verbaut, also nicht zusammen als Potentiometer. Statt einer Referenzinduktivität wie zuvor wird nun hinter den verstellbaren Widerstand $R_{3}$ eine Parallelschaltung von einer Referenzkapazität $C_{4}$ und einem 
weiteren verstellbaren Widerstand $R_{4}$ verbaut. Die Referenzkapazität $C_{4}$ beträgt hier $\SI{399}{\nano\farad}$ und der feste Widerstand $R_{2}$ hat $\SI{1}{\kilo\ohm}$. Die Speisespannung ist vom vorherigen Versuch unverändert.
\\
\newline
Die Toleranz der Widerstände $R_{3}$ und $R_{4}$ ist $\pm\SI{3}{\percent}$ und die der Referenzwerte $R_{2}$ und $C_{4}$ besitzen jeweils $\SI{0.2}{\percent}$.
Diese Messreihe ist in Tabelle \ref{tab:spuleyodiezweite} angegeben.

\subsection{Wien-Robinson-Brücke}
In dem letzten Versuchsabschnitt soll die Frequenzabhängigkeit der Brückenspannung $U_{B}$ untersucht werden, hierzu wird eine Wien-Robinson-Brücke mit dem Aufbau aus 
Abbildung \ref{fig:abb10} verwendet. Die Frequenz wird dabei in dem Intervall $20 \leq f \leq 20000 \si{\hertz}$ variiert und dazu die passende Brückenspannung notiert. Aus der Theorie in Gleichung \eqref{eqn:ma}
lässt sich die zu erwartende Frequenz für einen festgelegten Widerstand $R$ und eine Kapazität $C$ bestimmen. In dem Intervall um diesen Theoriewert ist es also besonders interessant und somit sinnvoll viele Messungen vorzunehmen.
Die Intervalle wurden dementsprechend an einigen Stellen größer oder kleiner gewählt. 
\\
\newline
Diese Messwerte sind in der Tabelle \ref{tab:frequencyitis} notiert.

