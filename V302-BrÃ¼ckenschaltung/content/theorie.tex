\section{Theorie}
Die Anwendung von Brückenschaltungen liegt in der Messtechnik,  vorallem von Wiederständen. 
Hier sollen sie durch eine Nullmethode die Auflösung der Messsung erhöhen um so präzisere Ergebnise
auslesen zu können. Dadurch lassen sich nicht nur reele wie komplexe Wiederstände, sondern auch eben 
solche Daten messen die den Wiederstand beeinflussen, wie zum Beispiel Temperatur oder Längenänderung. %ist das der grund?
Die Funktionsart eine Brückenschaltung beschränkt sich im wesentlichen auf die Abtastung einer 
Potentialdifferenz zwischen zwei Punkten. 
%figure abb1
Diese Differenz der Spannung $U_Bs$ wir im allgemeinen Brückenspannung genannt und zwischen den Punkten $A$ und $B$ gemessen.

\subsection{Kirchhoff'schen Gesetze}
Zentrale Grundlage der Berechnung sind die sogenannten Kirchhoff'schen Gesetze. Diese sind zweigeteilt und 
geben Auskumpft über den fließenden Strom in einem geschlossenen Kreis.
Das erste Gesetz besagt, dass in einem Knoten, also in einem Ort in dem der Strom zusammen - oder auseinander fließt, 
die Summe von elektrischen Strömen gleich Null ist.
\begin{equation}
    \sum_k \text{I}_k = 0
\end{equation}
Das zweite Gesetzt erläutert den Erhalt des fließenden Stroms, so dass in jedem abgeschlossenen Stromkreis, also einer Masche,
die Summe aller Teilspannungen gleich Null ist.
\begin{equation}
    \label{eqn:kirch2}
    \sum_k \text{U}_k = 0
\end{equation}
%maybe abb3.
Die Spannung U aus der Gleichung ist ein Produkt aus dem Strom und den von diesem erfahrenen Wiederstand.
\begin{equation}
    \text{U} = \text{R}\cdot \text{I}
\end{equation}
Unter Anwendung dieser Gesetze auf den Schaltplan %ref auf fig abb1
lässt sich der folgene Zusammenhang finden. Hier bezeichnet $U_s$ die vom Generator erezugte Spannung, also die Speisespannung. 
\begin{equation}
     U= \frac{R_2R_3 - R_1R_4}{(R_3+R_4)(R_1+R_2)}\cdot U_s
\end{equation}
Sollte die Brückenspannung verschwinden liegt eine sogenante abgeglichene Brücke da. 
Hieraus lassen sich also die Werte für unekannten Widerständen in diesem Stromkreis finden.
Die Bedingung dafür ist.
\begin{equation}
    \label{eqn:bedingung}
    R_1 R_4 = R_2 R_3
\end{equation}

\subsection{komplexe Widerstände}
Wenn ein Widerstand von einem Kondensator oder durch Induktivität erzeugt wird, ist dieser nicht mehr reines 
Element der reellen Zahlen. 
Die jetzt zum Teil komplexe Darstellung setzt sich aus dem reellen Teil des Wirkwiderstandes $X$ mit 
dem imaginären Blindwiederstand $Y$ zusammen. 
\begin{align*}
    R &= X +iY \\
    \Re{R} &= X \\
    \Im{R} &= Y 
\end{align*} 
Hier liegen also pro Widerstand zwei Paramter vor. Nach der Bedingung \eqref{eqn:bedingung} müssen also
nun der reelee und der imaginäre Teil übereinstimmen. 


















%\subsecton{Kirchhoffscheregel} %check title,
%\subsection{komplexe wiederstände}
%\subsection{wheatstone}
%\subsection{kapazitätbrücke}
%\subsection{Induktivitätsbrücke}
%\subsection{maxwellBrücke}
%\subsection{wienrobinson}
%\subsection{TT-Brücke}