\section{Aufbau}
Grundlegend wir der Versuch an einem Oszillator ausgeführt.
Dafür wird insbesondere ein Generator gebraucht, der in der Lage ist Spannungen verschiedener Typen herzustellen. Hier beschränken wir uns auf eine Rechteck - und Sinusspannung. Vom Oszillator soll dann jeweils die unmittelbare Spannung gemessen werden und die, die durch einen Kondensator und Widerstand modifiziert wurde. Auf dem Display sieht man also sowohl die eingehende (Rechteck/ - Sinusspannung) als auch die ausgehende Spannung, also die Darstellung des Lade - und Entladungsevorgang des Kondensators.

\section{Durchführung}
Die Durchführung teilt sich in die  drei folgenden Konfigurationen auf.
\subsection{Bestimmung der Zeitkonstante}
Um das Maß für die Geschwindigkeit zu errechnen gilt es also die Konstante $\text{RC}$ zu messen.
Dafür wird eine Rechtecksspannung vom Generator angelegt und am Oszillator das Laden beziehungsweise Entladen des Kondensators beobachtet. In Intervallen von $\Delta t = \SI{2.5}{ms}$ wird der Strom $U_c$ abgelesen und die Daten in eine Tabelle aufgenommen. %maybe reff zum anhang.
\subsection{Amplitudenbestimmung der Kondensatorspannung}
\label{sectionref}
Nun wird eine Sinusspannung vom Generator erzeugt, mit einer Frequenz die im Laufe jeweils um einen Wert von $\Delta\si{\hertz}= \SI{500}{\hertz}$ angehoben wird. Die entsprechende Amplitude $U_c$ wir dem Oszillator entnommen und in eine Tabelle eingetragen.
\subsection{Messung der Phasenverschiebung}
Bei Beibehaltung der aktuellen Konfiguration wird der Vorgang aus \ref{sectionref} im gleichen Intervall wiederholt. Dabei werden beide Spannungen, die Eingehende und Ausgehende, angezeigt und achsensymetrisch zur X - Achse ausgerichtet.
%inclucde graphic
Ein nun messbarer Abstand $a \text{und} b$ %ref zur darstlellung
ist nun bei entsprechender Frequenz und Amplitude zu entnehmen. Dieser wird wiederum in eine Tabelle 
eingetragen und im späteren zur Bestimmung der Phasenverschiebung zu benutzen.

%Fehler noch einfügen