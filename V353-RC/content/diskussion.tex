\section{Diskussion}

Zunächst lässt sich an der Auswertung folgendes feststellen. Die aufgenommene Entladungskurve liefert einen Wert für die 
Relaxationszeit $\tau$ mittels linearer Ausgleichsgeraden. Diese besitzt für das gesamte Entladungsintervall von $\increment t = \SI{42.5}{\milli\second}$ eine annehmbare
Größenordnung. Auf diesem Ergebnis liegen sowohl statistische als auch systematische Fehler welche durch ein größeres Messintervall und weitere Messungsdurchläufe verringert werden 
können. Das Oszilloskop besitzt zusätzlich nur eine gewisse Genauigkeit bei der Aufzeichnung von Auf- und Entladevorgängen. Die Spannung am Kondensators erreicht beispielweise 
nie den gänzlich vollendladenen Zustand.
Es gibt allerdings keinen Grund zur Annahme, dass das Ergebnis für die Relaxationszeit eine große Abweichung besitzt.
\\
\newline
Zur Bestimmung einer optimierten Relaxationszeit wurde die Spannungsamplitude am Kondensators in Abhängigkeit von der Frequenz gemessen. 
Erwartet wird, dass bei größer werdenden Frequenzen die Spannungsamplitude am Kondensators abfällt.
Dies konnte ebenfalls durch die Messwerte festgestellt werden und es bestätigt die Eigenschaft des 
RC-Kreises als Tiefpassfilter. Er lässt kleine Frequenzen ungehindert hindurch aber hohe Frequenzen werden unterdrückt.
Ein großes Problem ergibt sich allerdings durch das gewählte Messintervall der Frequenz.
Der größte Spannungsabfall des Kondensators geschieht schon bei Frequenz zwischen $\SI{10}{}$-$\SI{100}{\hertz}$ welche in der aufgenommenen Messreihe ausgelassen wurden.
Sinnvoll ist es also bei der Frequenzmessung logarithmische Abstände zu wählen. Diese Wahl der Messintervalle stellen vor allem bei der Phasenverschiebung ein größeres Problem dar.
Die Abweichung der Relaxationszeit beträgt hier bereits $\SI{24.4(18)}{\percent}$, deshalb kann das Ergebnis nicht als Optimierung betrachtet werden.
\\
\newline
Die Phasenverschiebung zwischen Generator- und Kondensatorspannung sollte bei größeren Frequenzen zunehmen, dies ist auch erkennbar allerdings
wieder nur zwischen zwei Messwerten.
Das Resultat ist ein Sprung zwischen der Frequenz $\SI{10}{\hertz}$ 
und $\SI{500}{\hertz}$ welcher genau diesen Sachverhalt zeigt. Anschließend ist allerdings ein Rücklauf der Phasenverschiebung für noch größere Frequenzwerte erkennbar.
Dies ist wieder auf die ungünstig gewählten Messintervalle zurückzuführen. Statt einem grundsätzlich steigenden Verlauf erhalten wir also einen leicht abfallenden Verlauf über die meisten Werte, welche die 
Ausgleichsrechnung drastisch beeinflusst. Erkennbar ist dennoch die zu Beginn kleine Phasenverschiebung bei $\SI{10}{\hertz}$, welche stark ansteigt und sich einer
Phasenverschiebung von $\pi$/$2$ annähert wie vorhergesagt.
\\
\newline
Aus dem untersuchten Relaxationsverhalten bei anliegender Wechselspannung kann die Integratoreigenschaft gut festgestellt werden. Eine Betrachtung von weiteren Eingangsspannungsformen
kann hier hilfreich sein.