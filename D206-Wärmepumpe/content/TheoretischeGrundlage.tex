\section{Theoretische Grundlage}
\label{sec:TheoretischeGrundlagen}
Der Wärmefluss innerhalb eines abgeschlossenen Systems verhält sich erfahrungsgemäß immer so, dass
ein heißerer Körper Wärme an einen kälteren Körper abgibt. 
Dieser Vorgang kann mit einer Wärmepumpe umgekehrt werden, dabei muss lediglich eine Form von Arbeit geleistet werden.
In diesem Versuch werden zwei Wärmereservoirs mit unterschiedlichen Temperaturen $T_{1}$ und $T_{2}$ verwendet, zwischen denen ein
Wärmeaustausch stattfindet.

\subsection{Grundlegende Größen}
\label{sec:GrundlegendeGroessen}
Im folgenden seien einige grundlegende Größen zur Beschreibung des Wärmeaustausch einer Wärmepumpe genannt.
\begin{flalign*}
Q_{1}&:\text{beschreibt die an das warme Reservoir abgegebene Wärmemenge}\\
Q_{2}&:\text{beschreibt die entnommene Wärmemenge aus dem kalten Reservoir}\\ 
T_{1}&:\text{Temperatur des Wassers in Reservoir 1}\\
T_{2}&:\text{Temperatur des Wassers in Reservoir 2}\\
p_{1}&:\text{Druck auf der Seite von Reservoir 1}\\
p_{2}&:\text{Druck auf der Seite von Reservoir 2}\\
A&:\text{angewendete Arbeit}
\end{flalign*}
Mit diesen Größen kann nachher unteranderem die Güteziffer $\nu$ berechnet werden.

\subsection{Funktionsweise einer Wärmepumpe}
\label{sec:Funktionsweise}
Der Wärmeaustausch eines Systems mit zwei wassergefüllten Reservoirs muss irgendwie transportiert werden, dazu verwendet
man bei einer Wärmepumpe ein reales Gas. Dieses Gas nimmt beim Verdampfen Wärme auf und gibt diese bei
der Kondensation wieder ab. Dadurch kann es Wärme durch seine Formumwandlung transportieren.
Das reale Gas wird anhand eines Kompressors durch die Apperatur durchgeführt. Im folgenden sei die Temperatur $T_{1}$ zugehörig zu
Reservoir 1 mit einer höheren Temperatur als der Temperatur $T_{2}$ des zweiten Reservoirs .
Das Gas ist so konzipiert, dass es bei der Temperatur $T_{1}$ und dem Druck $p_{1}$ kondensiert und bei $T_{2}$ und $p_{2}$ verdampft.
Wenn das verflüssigte Gas nun durch Reservoir 2 geführt wird, entzieht es dem Wasser eine Verdampfungswärme $L$ pro Gramm. Es nimmt also die Wärmemenge $Q_{2}$
des kälteren Wassers auf. Im Kompressor wird das zuvor verdampfte Gas adiabatisch komprimiert, also steigt sowohl der Druck, sowie die Temperatur des Gases.
Dadurch gibt es die Wärmemenge $Q_{1}$ an das Wasser in Reservoir 1 ab und wird kondensiert. Damit dies umsetzbar ist benötigt man ein reales Gas mit hoher Kondensationswärme.

\subsection{Die Güteziffer}
\label{sec:Gueteziffer}
Die Güteziffer einer Wärmepumpe beschreibt das Verhältniss von abgegebener Wärmemenge $Q_{1}$ und der geleisteten Arbeit $A$. Es folgt
\begin{equation}
\label{eqn:eqGueteziffer}
\nu = \frac{Q_{1}}{A}
\end{equation}
Dabei gilt stets der erste Hauptsatz der Wärmelehre welcher direkt aus der Energieerhaltung entspringt.
\begin{equation}
\label{eqn:eqHauptsatz}
Q_{1} = Q_{2} + A
\end{equation}
Bei dem zweiten Hauptsatz der Wärmelehre muss nun zwischen einer realen und einer idealen Wärmepumpe unterschieden werden.
Wenn die Wärmeübertragung durch die Wärmepumpe reversibel, also ohne Verluste wieder umgekehrt werden kann, dann gilt der ideale Fall.
\begin{equation}
\label{eqn:idealzweiterhauptsatz}
\frac{Q_{1}}{T_{1}} - \frac{Q_{2}}{T_{2}} = 0 \quad \to \quad Q_{2} = \frac{T_{2}}{T_{1}} Q_{1}
\end{equation}
Die Formel \eqref{eqn:idealzweiterhauptsatz} gibt also die Summe der reduzierten Wärmemengen im Idealfall an, also wenn sich die Temperatur der Reservoirs nicht ändern würde.
Im Realfall ist der Wärmeaustausch nicht reversibel und es folgt.
\begin{equation}
\label{eqn:realzweiterhauptsatz}
\frac{Q_{1}}{T_{1}} - \frac{Q_{2}}{T_{2}} > 0 \quad \to \quad Q_{2} < \frac{T_{2}}{T_{1}} Q_{1}
\end{equation}
Damit ergeben sich zwei Berechnungen für die Güteziffer durch einsetzen der Ergebnisse für $Q_{2}$ in \eqref{eqn:eqHauptsatz} und anschließend \eqref{eqn:eqGueteziffer}
\begin{align}
\nu_{\text{ideal}} &= \frac{Q_{1}}{A} = \frac{T_{1}}{T_{1}-T_{2}} \\
\nu_{\text{real}} &< \frac{T_{1}}{T_{1}-T_{2}}
\end{align}
Daraus kann man sofort schließen, dass bei nur kleinen Temperaturunterschieden $\increment T$, also der Differenz zwischen $T_{1}$ und $T_{2}$, eine große Güteziffer
entsteht. Es wird folglich nur eine kleine Arbeit gebraucht um eine beliebige Wärmemenge $Q_{1}$ zu transportieren. Bei nur kleinen $\increment T$ ist die Wärmepumpe also 
besonders effizient.
Um nun die Güteziffer experimentell zu bestimmen, macht es Sinn die relevanten Größen über einen gewissen Zeitraum $\increment t$ zu messen.
Allgemein gilt für eine beliebige Wärmemenge $Q$
\begin{equation}
\label{eqn:allgwaerme}
Q = cm \increment T
\end{equation}
Dabei ist $c$ die spezifische Wärmekapazität, $m$ die Masse des Stoffes und $\increment T$ die Temperaturänderung.
Durch diesen Zusammenhang lässt sich die Änderung der Wärmemenge $Q_{1}$ pro Zeit $\increment t$ wie folgt berechnen.
\begin{equation}
\label{eqn:eqwaermeq1prozeit}
\frac{\increment Q_{1}}{\increment t} = (m_{1}c_{w} + m_{k}c_{k})\frac{\increment T_{1}}{\increment t}
\end{equation}
Wobei $m_{1}c_{w}$ die Wärmekapazität des Wassers in Reservoir 1 und $m_{k}c_{k}$ die Wärmekapazität der Kupferschlange und des Eimers.
Aus der Gleichung \eqref{eqn:eqGueteziffer} folgt dann sofort
\begin{equation}
\label{eqn:gueteziffermesswerte}
\nu = \frac{\increment Q_{1}}{\increment t N} = (m_{1}c_{w} + m_{k}c_{k})\frac{\increment T_{1}}{\increment t N}
\end{equation}
Hierbei ist wichtig, dass $N$ eine über die Zeit $\increment t$ gemittelte Leistung des Kompressors ist.
Dadurch stimmt am Ende die Dimension der Güteziffer wieder mit der ursprünglichen Gleichung \eqref{eqn:eqGueteziffer} überein.

\subsection{Massendurchsatz}
\label{sec:massedurchsatz}
Mit der Gleichung \eqref{eqn:eqwaermeq1prozeit} findet man analog die pro Zeiteinheit entgenommene Wärmemenge $Q_{2}$
\begin{equation}
\frac{\increment Q_{2}}{\increment t} = (m_{2}c_{w} + m_{k}c_{k})\frac{\increment T_{2}}{\increment t}
\end{equation}
Wie in \ref{sec:TheoretischeGrundlagen} erwähnt kann man die entzogene Wärmemenge $\increment Q_{2}$ auch als Verdampfungswärme $L$ pro Gramm schreiben.
Dadurch erhält man einen Ausdruck für den Massenstrom.
\begin{equation}
\frac{\increment Q_{2}}{\increment t} = L \frac{\increment m}{\increment t} \quad \to \quad \frac{\increment m}{\increment t} = \frac{\increment Q_{2}}{L \increment t}
\end{equation}
Also lässt sich der Massendurchsatz innerhalb der Wärmepumpe berechnen, wenn $L$ bekannt ist.

\subsection{Mechanische Kompressorleistung}
Unter der Annahme, dass die Kompression adiabatisch erfolgt gilt die Poisson Gleichung
\begin{equation}
\label{eqn:poisson}
p_{2}V_{2}^{\kappa} = p_{1}V_{1}^{\kappa} = pV^{\kappa}
\end{equation}
Dabei ist $\kappa$ der Isentropenexponent und ist definiert als das Verhältniss der Molwärmen $C_{p}$ und $C_{v}$.
Die mechanische Arbeit um ein Gas mit Druck $p$ vom Volumen $V_{2}$ auf $V_{1}$ zu reduzieren wird über das Integral
\begin{equation}
A_{mech} = - \int_{V_{2}}^{V_{1}} p \symup{dV} = \int_{V_{1}}^{V_{2}} p \symup{dV}
\end{equation}
berechnet. Wenn man nun die Poisson Gleichung \eqref{eqn:poisson} nach $p$ auflöst, in das Integral einsetzt und ausrechnet erhält man.
\begin{equation}
A_{mech} = \frac{1}{\kappa - 1} \left( p_{1}\sqrt[\kappa]{\frac{p_{2}}{p_{1}}} - p_{2} \right) V_{2}
\end{equation}
Die Kompressorleistung $N_{mech}$ folgt direkt als Arbeit pro Zeit und kann dann auch über den Massendurchsatz ausgedrückt werden.
\begin{equation}
N_{mech} = \frac{1}{\kappa - 1} \left( p_{1}\sqrt[\kappa]{\frac{p_{2}}{p_{1}}} - p_{2} \right) \frac{V_{2}}{\increment t} = \frac{1}{\kappa - 1} \left( p_{1}\sqrt[\kappa]{\frac{p_{2}}{p_{1}}} - p_{2} \right) \frac{1}{\rho} \frac{\increment m}{\increment t}
\end{equation}
Die Dichte bezieht sich hier auf den Zustand von $V_{2}$ also im gasförmigen Zustand kurz nach dem Verdampfen. Man kann diese hier
durch die ideale Gasgleichung beschreiben.
\begin{equation}
p V = nRT = p \cdot \frac{m}{\rho} = nRT \quad \to \quad \rho = \frac{pm}{nRT}
\end{equation}
$R$ ist hier die allgemeine Gaskonstante, $n$ die Stoffmenge und $p$ der Druck.
%der letzte Teil ist noch unklar also die Berechnung von rho über die Gasgleichung, stellt sich wahrscheinlich bei der Auswertung heraus und 
%sollte dann noch ergänzt werden