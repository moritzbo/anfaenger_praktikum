\section{Diskussion}
Die gemessenen Daten scheinen plausibel, da es keine großen Ausreißer in den gemessenen Temperaturen
sowie Drücken gab. Die ermittelten Werte inklusive Fehler auf die Güteziffern, Kompressorleistungen und Massendurchsätze 
bewegen sich in realistischen Verhältnissen. 
Es entstehen logischerweise einige Fehler beim Ablesen der Messwerte in nicht immer exakten Intervallen, welche die 
Auswertung der Ergebnisse beeinflussen können.
Einige Unsicherheiten entstehen auch durch den nicht völlig optimierten Aufbau. Beispielweise lässt sich nie gänzlich
verhindern, dass die Reservoire Wärme an die Umgebung abgeben. 
Erkennbar ist ein sehr großer Unterschied zwischen den ausgerechneten Werten $\nu_{real}$ und $\nu_{ideal}$. 
Da der reale Wärmeaustausch allerdings im Gegensatz zum idealen, irreversibel ist,
und nach Gleichung \eqref{eqn:idealgueteziffer}, sowie \eqref{eqn:realgueteziffer} $\nu_{\text{real}}$ stets kleiner als $\nu_{\text{ideal}}$, scheinen die Ergebnisse plausibel.