\section{Auswertung}

Untersucht wurden die Ausgangsspannungen am Lock-In-Verstäker in Abhängigkeit von dem Phasenunterschied zwischen Signal- und Referenzspannung. Im Folgenden ist die Auswertung 
für eine Signalspannung mit und ohne Rauschen unterteilt.

\subsection{Ausgangsspannung ohne Rauschen}
In diesem Versuchsteil wurden die Ausgangsspannungen am Lock-In-Verstäker einer mit einer Rechteckspannung modifizierten sinusförmigen Signalspannung gemessen. In der Tabelle \ref{tab:1}
sind die Ausgangsspannung $U_{\text{out}}$ in Abhängigkeit von der Phase notiert. An einem Oszilloskop ließen sich Momentanaufnahmen von unverstärkten Signalspannung (in Gelb), sowie den Ausgangsspannungen (in Türkis) anfertigen.
Die Abbildungen \ref{fig:11} bis \ref{fig:18} zeigen die Spannungen in Abhängigkeit von der Phase $\phi$ welche von $\SI{0}{\degree}$ bis $\SI{360}{\degree}$ verändert werden.
\\
\newline 
Die Beziehung \eqref{eqn:yesss} lässt sich anhand dieser Messwerte mit einer Fitfunktion der Art 
\begin{equation}
    \label{eqn:fit}
U_{\text{out}} = a \cdot \text{cos}(b \cdot \frac{180}{\pi} \phi + c) + d,
\end{equation}
überprüfen. Dazu wird ein \enquote{Curvefit} mit Scipy \cite{scipy} verwendet.
Es ergeben sich die folgenden Parameter
\begin{align*}
    a &= \SI{94.671(14105)}{\volt},\\
    b &= \SI{0.807(81)}{},\\
    c &= \SI{-13.358(307)}{\radian},\\
    d &= \SI{-4.382(10165)}{\volt}.\\
\end{align*}
%a = 94.6713 ± 14.1051
%b = 0.8065 ± 0.0808
%c = -16.4998 ± 0.3073
%d = -4.3822 ± 10.1645
Die Fitfunktion mit diesen Parametern, sowie die einzelnen Messwerte aus Tabelle \ref{tab:1} sind in der Abbildung \ref{fig:licht} dargestellt.
\begin{figure}
    \centering
    \includegraphics[width=0.9\textwidth]{build/plot1.pdf}
    \caption{Phasenabhängige Ausgangsspannungen ohne Rauschen mit Fitfunktion.} 
    \label{fig:licht}
\end{figure}
\begin{figure}
    \begin{minipage}{0.5\textwidth}
        \centering
        \includegraphics[width=0.8\textwidth]{bilder/0ohne.png}
        \caption{$\phi = \SI{0}{\degree}$.} 
        \label{fig:11}
    \end{minipage}
    \hfill
    \begin{minipage}{0.5\textwidth}
        \centering
        \includegraphics[width=0.8\textwidth]{bilder/45ohne.png}
        \caption{$\phi = \SI{45}{\degree}$.} 
        \label{fig:12}
    \end{minipage}
    \vspace{1cm}
    \vfill
    \begin{minipage}{0.5\textwidth}
        \centering
        \includegraphics[width=0.8\textwidth]{bilder/90ohne.png}
        \caption{$\phi = \SI{90}{\degree}$.} 
        \label{fig:13}
    \end{minipage}
    \hfill
    \begin{minipage}{0.5\textwidth}
        \centering
        \includegraphics[width=0.8\textwidth]{bilder/135ohne.png}
        \caption{$\phi = \SI{135}{\degree}$.} 
        \label{fig:14}
    \end{minipage}
    \vspace{1cm}
    \vfill
    \begin{minipage}{0.5\textwidth}
        \centering
        \includegraphics[width=0.8\textwidth]{bilder/180ohne.png}
        \caption{$\phi = \SI{180}{\degree}$.} 
        \label{fig:15}
    \end{minipage}
    \hfill
    \begin{minipage}{0.5\textwidth}
        \centering
        \includegraphics[width=0.8\textwidth]{bilder/240ohne.png}
        \caption{$\phi = \SI{240}{\degree}$.} 
        \label{fig:16}
    \end{minipage}
    \vspace{1cm}
    \vfill
    \begin{minipage}{0.5\textwidth}
        \centering
        \includegraphics[width=0.8\textwidth]{bilder/300ohne.png}
        \caption{$\phi = \SI{300}{\degree}$.} 
        \label{fig:17}
    \end{minipage}
    \hfill
    \begin{minipage}{0.5\textwidth}
        \centering
        \includegraphics[width=0.8\textwidth]{bilder/360ohne.png}
        \caption{$\phi = \SI{360}{\degree}$.} 
        \label{fig:18}
    \end{minipage}
    \caption{Momentaufnahmen der phasenabhängigen Ausgangspannungen an einem Lock-In-Verstärker ohne Rauschen.}
\end{figure}

\subsection{Ausgangsspannung mit Rauschen}
Die Abbildungen \ref{fig:21} bis \ref{fig:27} zeigen die Ausgangsspannungen in Abhängigkeit von der Phase $\phi$, im Fall der verrauschten Signalspannung. Die Spannungen in der Tabelle \ref{tab:2}
wurden dabei über die einzelnen Peaks gemittelt und bereits gemittelt in die Tabelle eingetragen.
\\
\newline 
Für diese Auswertung wird das gleiche Verfahren wie ohne Rauschen mit der Messreihe aus Tabelle \ref{tab:2} wiederholt. Mit der Beziehung \eqref{eqn:fit} ergeben sich nun die folgenden
Parameter
\begin{align*}
    a &= \SI{ 108.811(20210)}{\volt},\\
    b &= \SI{0.792(100)}{},\\
    c &= \SI{-45.052(349)}{\radian},\\
    d &= \SI{9.608(14443)}{\volt}.\\
\end{align*}
%a = 108.8108 ± 20.2103
%b = 0.7918 ± 0.1001
%c = -45.0519 ± 0.3486
%d = 9.6080 ± 14.4434
In der Abbildung \ref{fig:licht2} sind nun die Messwerte dieser Messreihe, sowie die Fitfunktion dargestellt.
\begin{figure}
    \centering
    \includegraphics[width=0.9\textwidth]{build/plot2.pdf}
    \caption{Phasenabhängige Ausgangsspannungen mit Rauschen und passende Fitfunktion.} 
    \label{fig:licht2}
\end{figure}
\begin{figure}
    \begin{minipage}{0.5\textwidth}
        \centering
        \includegraphics[width=0.8\textwidth]{bilder/0mit.png}
        \caption{$\phi = \SI{0}{\degree}$.} 
        \label{fig:21}
    \end{minipage}
    \hfill
    \begin{minipage}{0.5\textwidth}
        \centering
        \includegraphics[width=0.8\textwidth]{bilder/60mit.png}
        \caption{$\phi = \SI{60}{\degree}$.} 
        \label{fig:22}
    \end{minipage}
    \vspace{1cm}
    \vfill
    \begin{minipage}{0.5\textwidth}
        \centering
        \includegraphics[width=0.8\textwidth]{bilder/120mit.png}
        \caption{$\phi = \SI{120}{\degree}$.} 
        \label{fig:23}
    \end{minipage}
    \hfill
    \begin{minipage}{0.5\textwidth}
        \centering
        \includegraphics[width=0.8\textwidth]{bilder/180mit.png}
        \caption{$\phi = \SI{180}{\degree}$.} 
        \label{fig:24}
    \end{minipage}
    \vspace{1cm}
    \vfill
    \begin{minipage}{0.5\textwidth}
        \centering
        \includegraphics[width=0.8\textwidth]{bilder/240mit.png}
        \caption{$\phi = \SI{240}{\degree}$.} 
        \label{fig:25}
    \end{minipage}
    \hfill
    \begin{minipage}{0.5\textwidth}
        \centering
        \includegraphics[width=0.8\textwidth]{bilder/300mit.png}
        \caption{$\phi = \SI{300}{\degree}$.} 
        \label{fig:26}
    \end{minipage}
    \vspace{1cm}
    \vfill
        \centering
        \includegraphics[width=0.4\textwidth]{bilder/360mit.png}
        \caption{$\phi = \SI{360}{\degree}$.} 
        \label{fig:27}
    \caption{Momentaufnahmen der phasenabhängigen Ausgangspannungen an einem Lock-In-Verstärker mit Rauschen.}
\end{figure}

\subsection{Messung mit Photodetektor}
In dem letzten Versuchteil wurde eine LED mit einer Signalspannung betrieben und durch einen Photodetektor in unterschiedlichen Abständen eine Ausgangsspannung gemessen.
Hierbei lässt sich der vermutete Intensitätsabfall von $1$/$d^2$ überprüfen. Da die Intensität $I \propto U^2$ ist, ist somit ein Spannungsabfall von $U \propto 1$/$d$ zu erwarten.
Die gemessenen abstandsabhängigen Spannungen sind in der Tabelle \ref{tab:3} angegeben und lassen sich nun mit einer Fitfunktion vom Typ
\begin{equation}
U_{\text{out}} = a \cdot \frac{1}{d} + b
\end{equation}
fitten. Die Parameter ergeben sich dabei zu
\begin{align*}
    a &= \SI{ 0.086(6)}{\volt\meter},\\
    b &= \SI{1.579(216)}{\volt}.\\
\end{align*}
Die Messwerte, sowie die Fitfunktion sind in der Abbildung \ref{fig:licht123} dargestellt.
\begin{figure}
    \centering
    \includegraphics[width=0.9\textwidth]{build/plot3.pdf}
    \caption{Abstandsabhängige Ausgangsspannung einer mit LED Licht bestrahlten Photodiode.} 
    \label{fig:licht123}
\end{figure}