\section{Diskussion}

Zunächst lässt sich sagen, dass der geschaltete Lock-In-Verstärker anfällig für kleine Erschütterungen, auf Grund der Verkabelung, ist.
Es muss also stets darauf geachtet werden, dass kein Wackelkontakt existiert und unnötige Berührungen vermieden werden.
\\
\newline
An den Messwerten mit und ohne Rauschen in der Signalspannung lässt sich bereits ein kosinusförmiger Verlauf erkennen. Dieser konnte
durch die Fitfunktionen auch bestätigt werden. Es kommt allerdings zu kleinen Phasenverschiebungen, was darauf schließen lässt, dass die Angabe der $\increment \phi = \SI{0}{\degree}$ Phasenverschiebungen am
Lock-In-Verstärker keine \enquote{echten} $\increment \phi = \SI{0}{\degree}$ sind. Die Aussage des Parameters $a$ ist nur bedingt in Verbindung zu ziehen mit der in Gelb eingezeichneten
sinusförmigen Signalspannung aus den Abbildungen \ref{fig:11} bis \ref{fig:18}, da diese noch verstärkt wurde.
\\
Insgesamt lässt sich aber sagen, dass der nach der Theorie vermutete kosinusförmige Verlauf bestätigt werden kann.
\\
\newline
Bei der Messung der abstandsabhängigen Spannung an der Photodiode kommt es zu Messungenauigkeiten. Dadurch, dass der Raum nicht
komplett abgedunkelt werden kann, kommt es zu einer konstanten Ausgangsspannung, wodurch die Werte an sich größer und verrauschter sind als sie erwartet werden. Desweiteren 
existiert somit auch bei sehr großen Abständen immer noch eine messbare Spannung, dies ist auch an der Abbildung \ref{fig:licht123} erkennbar. Die Fitfunktion passt 
dementsprechend nicht völlig zu den gemessenen Werten. Der allgemeine Verlauf scheint aber plausibel.