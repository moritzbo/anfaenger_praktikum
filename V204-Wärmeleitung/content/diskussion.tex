\section{Diskussion}

Beim Versuchsaufbau fallen zunächst folgende Dinge auf. Die Aufzeichnung der Messwerte über den Datenlogger ist genau und deutlich besser als mit der Hand notierte Werte. Die Anzahl der Messungen
ist in allen Versuchsmethode außerdem sehr hoch und lässt somit auf eine hohe Genauigkeit schließen. Eine Problematik stellt allerdings die Isolation der Metallstäbe bei der Erwärmung mit dem Peltier-Element dar, denn
diese ist nur sehr schwach. Dies ist unteranderem an den Wärmestrommesswerten aus den Tabellen \ref{tab:mesbreit} bis \ref{tab:edel} zu erkennen. Während die Temperatur seinem Maximum nahe kommt wird der Wärmestrom zwischen den beiden Enden eines Stabes 
reduziert, er wird allerdings hier niemals Null, also gibt das Metall noch Wärme an die Umgebung ab.
\\
\newline
Alle vier Metallstabenden zeigen am Anfang einen starken Temperaturanstieg, welcher mit der Zeit abflacht. Anhand der Verläufe kann bereits eine Abschätzung für die Größenunterschiede der Wärmeleitfähigkeiten 
$\kappa$ gemacht werden. Erwartet wurde, dass Aluminium die höchste und Edelstahl die geringste Wärmeleitfähigkeit besitzt. Dies konnte durch die errechneten Werte und den Literaturwerten aus Tabelle \ref{tab:lit} bestätigt werden.
\\
Misst man die Differenz zwischen den inneren und äußeren Thermoelementen fallen zwei unterschiedliche Verläufe bei Messing und Edelstahl auf. Während die Temperaturdifferenz von Edelstahl streng monoton wächst, ist bei Messing ein 
nach dem Maximum auftretender Abfall bemerkbar. Daraus lässt sich deuten, dass die Wärmeleitfähigkeit mit höherer Temperatur ansteigt und dann konstant wird.
\\
\newline
Bei der dynamischen Methode wurde für alle Messungen eine Periodendauer von $T = \SI{200}{\second}$ gewählt. Dies stellt sich für die Bestimmung der Wärmeleitfähigkeit von Messing und Aluminium als problematisch heraus, da es bereits
zu einer Sättigung kommt und somit die Amplituden der beiden Thermoelementen nahbeiander liegen. Die Wärmeleitfähigkeit liegen somit deutlich über dem Literaturwert, allerdings sind die Größenunterschiede untereinander wie erwartet.
Für die Edelstahlberechnung hingegen ist das Periodenintervall ideal und es entsteht ein sehr genaues Ergebnis. Durch die Verwendung von Mittelwerten und ihren Fehlern wird das Ergebnis ebenfalls optimiert.
\\
\newline
Im Folgenden sind die prozentualen Abweichungen der ermittelten Wärmeleitfähigkeiten von den Literaturwerten nach Gleichung \ref{eqn:lol} angegeben.
\begin{align}
\kappa_{\si{\percent}\text{,mess}} &=  \SI{210.10(5128)}{\percent} \\ 
\kappa_{\si{\percent}\text{,alu}} &=   \SI{366.07(10506)}{\percent} \\
\kappa_{\si{\percent}\text{,edel}} &= \SI{13.58(1092)}{\percent}
\end{align}