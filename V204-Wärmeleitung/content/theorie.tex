\section{Theorie}
Befinden sich zwei Systeme in einem thermischen Gleichgewicht, also die Temperatur der beiden ist gleich, 
wird dieser Zustand auch nach beliebiger Zeit, mit gleichem Ergbenis, erneut messbar sein. Sollte ein System jedoch eine andere Temperatur besitzen, beobachtet 
man einen Energiefluss in Richtung des Systems mit niedriger Temperatur. Dieser Prozess findet in Form von Wärmeleitung, durch frei bewegliche Elektronen und Phononen statt.
\\
\newline
Für eben diesen Prozess muss sowhol die Länge $L$ des Werkstoffs, als auch der Quesrschnitt $A$ bekannt sein. Die Länge trennt in diesem Fall die zwei Systeme,
hier die verschieden Orte der Messung. Desweitern wird die Dichte $\rho$ und spezifische Wärme $c$ des enstprechenden Werkstoffs beötigt.
Die aus der Wärmeleitung resultiernende, fließende Wärmeenergie $dQ$ lässt sich also wie folgt für ein Zeitintervall $dt$ bestimmen.
\begin{equation}
    \label{eqn:itsHeadacheTimeYeeeay}
    dQ = -\kappa A \frac{\partial T}{\partial x} dt 
\end{equation}
Hierbei ist $\kappa$ ein Wert für die sogenannte Wärmeleitfähigkeit des enstprechenden Metalls. Aus der Natur der Thermodynamik 
folgt, dass nur eine Fluss von Wärme zum kühleren System möglich ist, nicht andersherum. Ist der Gradient $\frac{\partial T}{\partial x}$ also postiv, 
wird durch das Minuszeichen die Energie eben von wärmeren System abgegeben, nicht aufgenommen.
Analog gilt für die Wärmestromdichte.
\begin{equation}
    \label{eqn:stromdichte}
    j_w = -\kappa \frac{\partial T}{\partial x}
\end{equation}