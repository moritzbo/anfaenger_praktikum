\section{Diskussion}

In der gesamten Versuchsdurchführung wurde die Ableseungenauigkeit auf die minimale Skalengröße also auf $\SI{0.5}{\degree}$ geschätzt, diese betrug in der
Berechnung der einzelnen Größen immer einen ausschlaggebenderen Wert als die statistischen Fehler. Diese Definition als einzige Fehlerquelle vernachlässigt jegliches Verrutschen der Messskala, welche ebenfals zu einem
größeren Fehler beitragen könnte. Das verwendete Gitter bei der Betrachtung des Beugungsverhalten konnte nicht in eine feste Halterung platziert werden, da er sonst nicht von dem festgebauten grünen Laser getroffen wird. Dadurch
kann es ebenfalls zu leichten Abweichungen der Intensitätsmaxima kommen.
\newline
\\
Das Reflexionsgesetz wurde mit den gemessenen Werten präzise bestätigt, woraus sich im Allgemeinen eine sehr genaue Apparatur vermuten lässt. Grundsätzlich sind auch auf diesen Messwerten die Ableseungenauigkeiten zu berücksichtigen, allerdings
sind die gemessenen Werten bereits identisch mit den zu erwartenden Theoriewerten.
\newline
\\
Die prozentuale Abweichung des ermittelten Brechungsindex $n$ zu dem Theoriewert aus Tabelle \ref{tab:litindex} beträgt
\begin{equation}
    \label{eqn:prozn}
\increment n = \SI{2.61(134)}{\percent}.
\end{equation}
Die Untersuchung der Brechung zeigt bei dem berechneten Brechungsgesetzes also eine sehr geringe prozentuale Abweichung von unter $\SI{3}{\percent}$ zum Literaturwert in Tabelle \ref{tab:litindex}. Dieses Ergebnis ist also sehr genau und auch
die Lichtgeschwindigkeit in Plexiglas ist demnach annehmbar und bestätigt die Annahme, dass Licht im optisch dünneren Medium langsamer propagiert.
\newline
\\
Der ermittelte Strahlversatz $s$ an der planparallelen Plexiglasplatte liefert ein realistisches Ergebnis nach Augenmaß und es wurden nur die bereits als gut eingeschätzten Werte der Brechung verwendet, welche auch dem
Literaturvergleich standhalten. Auch der Vergleich der beiden Bestimmungsmethode zeigt diesen Zusammenhang. Auf den ersten Blick wirken die beiden Methoden in ihrem Ergbenis sehr ähnlich. Die Ähnlichkeit resultiert aus 
der Abhängigkeit vom Winkel $\beta$. Das soll heißen, dass bei der Brechung bereits der Brechungsindex $n(\alpha, \beta)$
berechnet wurde und aus diesem schließlich wieder $\beta$ selbst. 
\newline
\\
Für den Strahlengang an einem Prisma wurden jeweils immer die gleichen Einfallswinkel $\alpha_1$ verwendet und da der Brechungsindex von der Wellenlänge unabhängig bestimmt wurde ergeben sich automatisch gleiche Winkel $\beta_1$ und $\beta_2$ 
unabhängig vom verwendeten Laser. Dies könnte genauer dargestellt werden, wenn der Brechungsindex für beide Laser einzeln bestimmt wird. Da sich die Austrittswinkel $\alpha_2$ allerdings in der Messung unterscheiden ergeben sich dennoch die zu 
erwartenden unterschiedlichen Ablenkungen $\delta$. Da generell Licht mit größeren Wellenlängen stärker gebrochen wird, bestätigt die Rechnung diese Annahme.
\newline
\\
Bei der Beugung an einem Gitter stellt sich heraus, dass die Wellenlänge sehr präzise gemessen werden kann.
Zu den gegebenen Literaturwerten der verwendeten Laser 
\begin{align*}
    \lambda_{\text{grün,lit}} &= \SI{532}{\nano\meter}, \\
    \lambda_{\text{rot,lit}} &= \SI{635}{\nano\meter}, \\
\end{align*}
entsteht eine prozentuale Abweichung von 
\begin{align*}
    \increment \lambda_{\text{grün}} &= \SI{2.76(165)}{\percent}, \\
    \increment \lambda_{\text{rot}} &= \SI{3.05(138)}{\percent}.\\
\end{align*}
Die prozentualen Literaturabweichungen liegen also nur bei einigen Prozenten, was gut durch den Ablesefehler der Winkel erklärt werden kann. Das Ergebnis lässt sich durch eine größere Anzahl an Messwerten noch weiter verbessern.