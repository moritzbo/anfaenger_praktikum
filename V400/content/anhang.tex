\section{Anhang}

\begin{table}
    \caption{Einfall- und Reflexionswinkel bei der Reflexion an einem Spiegel.}
    \centering
    \label{tab:reflexion}
    \begin{tabular}{c c}
        \toprule
        Einfallswinkel $\alpha$ [$\si{\degree}$] & Reflexionswinkel $\beta$ [$\si{\degree}$] \\
        \midrule
        70 & 70 \\
        65 & 65 \\
        60 & 60 \\
        55 & 55 \\
        50 & 50 \\
        45 & 45 \\
        40 & 40 \\
        \bottomrule    
    \end{tabular}
\end{table}

\begin{table}
    \caption{Einfall- und Brechungswinkel bei der Untersuchung der Brechung an einer planparallelen Platte.}
    \centering
    \label{tab:brechung}
    \begin{tabular}{c c}
        \toprule
        Einfallswinkel $\alpha$ [$\si{\degree}$] & Brechungswinkel $\beta$ [$\si{\degree}$] \\
        \midrule
        20 & 14 \\
        25 & 17 \\
        30 & 20 \\
        35 & 23 \\
        40 & 26 \\
        45 & 29 \\
        50 & 31 \\
        \bottomrule    
    \end{tabular}
\end{table}

\begin{table}
    \caption{Einfalls- und Ausfallswinkel des grünen Lasers mit Wellenlänge $\protect \lambda = \SI{532}{\nano\meter}$ an einem optischen Prisma.}
    \centering
    \label{tab:prismagreen}
    \begin{tabular}{c c}
        \toprule
        Einfallswinkel $\alpha$ [$\si{\degree}$] & Brechungswinkel $\beta$ [$\si{\degree}$] \\
        \midrule
        30.0 & 77.0 \\
        35.0 & 68.0 \\
        40.0 & 60.0 \\
        45.0 & 53.0 \\
        50.0 & 48.0 \\
        \bottomrule    
    \end{tabular}
\end{table}

\begin{table}
    \caption{Einfalls- und Ausfallswinkel des roten Lasers mit Wellenlänge $\protect \lambda = \SI{635}{\nano\meter}$ an einem optischen Prisma.}
    \centering
    \label{tab:prismared}
    \begin{tabular}{c c}
        \toprule
        Einfallswinkel $\alpha$ [$\si{\degree}$] & Brechungswinkel $\beta$ [$\si{\degree}$] \\
        \midrule
        30.0 & 75.5\\
        35.0 & 67.0\\  
        40.0 & 59.0\\
        45.0 & 52.0\\
        50.0 & 47.0\\
        \bottomrule    
    \end{tabular}
\end{table}

\begin{table}
    \centering
    \caption{Winkelmessung der sichtbaren Beugungsordnungen $\protect k$ bei Gitterkonstante $\protect d = {600^{-1}}\si{\milli\meter\per{\text{Linie}}}$ für den roten und grünen Laser.}
    \label{tab:600linien}
    \begin{tabular}{c || c | c }
        \multicolumn{1}{c}{~} &\multicolumn{1}{c}{$\lambda = \SI{532}{\nano\meter}$} & \multicolumn{1}{c}{$\lambda = \SI{635}{\nano\meter}$} \\
        \midrule
        Beugungsordnung $k$  & $\alpha$ [$\si{\degree}$]  & $\alpha$ [$\si{\degree}$] \\
        \midrule
        1    &   19.0     & 22.5 \\ 
        2    &   39.0     & 47.0 \\ 
        \bottomrule
    \end{tabular}
\end{table}

\begin{table}
    \centering
    \caption{Winkelmessung der sichtbaren Beugungsordnungen $\protect k$ bei Gitterkonstante $\protect d = {300^{-1}}\si{\milli\meter\per{\text{Linie}}}$ für den roten und grünen Laser.}
    \label{tab:300linien}
    \begin{tabular}{c || c | c }
        \multicolumn{1}{c}{~} &\multicolumn{1}{c}{$\lambda = \SI{532}{\nano\meter}$} & \multicolumn{1}{c}{$\lambda = \SI{635}{\nano\meter}$} \\
        \midrule
        Beugungsordnung $k$  & $\alpha$ [$\si{\degree}$]  & $\alpha$ [$\si{\degree}$] \\
        \midrule
        1    &    9.25    & 11.00 \\ 
        2    &   18.75    & 22.00 \\ 
        3    &   28.25    & 33.00 \\ 
        \bottomrule
    \end{tabular}
\end{table}

\begin{table}
    \centering
    \caption{Winkelmessung der sichtbaren Beugungsordnungen $\protect k$ bei Gitterkonstante $\protect d = {100^{-1}}\si{\milli\meter\per{\text{Linie}}}$ für den roten und grünen Laser.}
    \label{tab:100linien}
    \begin{tabular}{c || c | c }
        \multicolumn{1}{c}{~} &\multicolumn{1}{c}{$\lambda = \SI{532}{\nano\meter}$} & \multicolumn{1}{c}{$\lambda = \SI{635}{\nano\meter}$} \\
        \midrule
        Beugungsordnung $k$  & $\alpha$ [$\si{\degree}$]  & $\alpha$ [$\si{\degree}$] \\
        \midrule
        1    &    3.00    & 3.75 \\ 
        2    &    6.25    & 7.50 \\ 
        3    &    9.50    & 11.25 \\ 
        4    &   12.50    & 15.00 \\
        5    &   15.75    & 19.00 \\
        6    &   19.00    & 23.00 \\
        7    &   22.25    & 27.00 \\
        8    &   25.75    & 31.00 \\
        \bottomrule
    \end{tabular}
\end{table}

\begin{table}
    \caption{Literaturwerte einiger Brechungsindizes \cite{lit2} \cite{lit3}.}
    \centering
    \label{tab:litindex}
    \begin{tabular}{c c}
        \toprule
        Stoff & Brechungindex $n$\\
        \midrule
        Luft  &  1.00    \\
        Plexiglas  &   1.49   \\
        \bottomrule    
    \end{tabular}
\end{table}


