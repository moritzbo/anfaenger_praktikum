\section{Auswertung}

Für die Auswertung wurden einige Bibliotheken in Python, wie \enquote{NumPy} \cite{numpy}, \enquote{SciPy} \cite{scipy} und \enquote{uncertainties} \cite{uncertainties} verwendet.

\subsection{Reflexionsgesetz}
Die gemessen Reflexionswinkel $\alpha_2$ lassen sich der Tabelle \ref{tab:reflexion} entnehmen. Sie zeigen ausnahmslos
das Verhalten
\begin{equation*}
    \alpha_1=\alpha_2.
\end{equation*}


\subsection{Brechungsgesetze}
\label{brechung}
Die Winkel der zweiten Messreihe wurden in der Tabelle \ref{tab:brechung} gesammelt. Diesen liegt jeweils ein Messfehler von
$\Delta = \SI{0.5}{\degree}$ bei welcher durch das ungenaue Ablesen kommt. Aus der Formel \eqref{eqn:snell} ergibt sich dann 
der Brechungsindex für die Platte. Es bietet sich also an die gemessen Winkel für jeden der sieben Messungen in die Formel einzusetzen 
und anschließend die gefundenen Werte zu mitteln. Der Mittelwert des Brechungsindex ergibt sich zu
\begin{equation}
    \label{eqn:n}
    \bar{n} = \SI{1.451(0020)}{}.
\end{equation}
Der Fehler resultiert nur aus der Ableseungenauigkeit und lässt sich durch eine Gaußsche Fehlerfortpflanzung der Mittelwertsberechnung bestimmen
\begin{equation*}
\increment \bar{n} = \frac{1}{N} \sqrt{\sum_{i=1}^{N} (\increment n_{i})^2}.
\end{equation*}
Dabei gibt $N$ die Anzahl der bestimmten Brechungsindizes an über die gemittelt wird.
Der statistische Fehler des Mittelwerts ist deutlich geringer und lässt sich folgendermaßen bestimmen
\begin{equation}
\increment \overline{n_{\text{statistisch}}} = \frac{1}{\sqrt{N}} \sqrt{\frac{1}{N-1}\sum_{i=1}^{N} (n_{i} -\bar{n})^2 } = \SI{0.009}{}. 
\end{equation}
Somit ist dieser bereits im obrigen Fehler enthalten. Mit der Gleichung \eqref{eqn:snell} lässt sich zusammen mit der Näherung $n_1 \approx 1$
und $v_1 = c$ \cite{scipy} die Lichtgeschwindigkeit bestimmen. Hieraus folgt
\begin{align*}
    v &= \frac{c}{n} \\
    v &= \SI{206.60(00284)e6}{\meter\per\second}
\end{align*}
Der Fehler entsteht durch eine Gaußsche Fehlerfortpflanzung die wie folgt lautet
\begin{equation}
    \increment v = \frac{c \increment n}{n^2}.
\end{equation}

\subsection{Planparallele Platten}
Mit fünf Messwerten aus Tabelle \ref{tab:brechung} lässt sich nun der Strahlversatz errechnen. Hierzu wird die Gleichung \eqref{eqn:versatz}
benutzt und die Winkel mit entsprechendem Fehler eingesetzt. Die Dicke der Platte ist dem Aufbau zu entnehmen \ref{idkwhatthisissss}.
Die errechneten Strahlversätze $s$ sind in der folgenden Tabelle \ref{tab:wrtfffffffffff} angegeben.
\begin{table}
    \centering
    \caption{Angegeben ist der Strahlversatz $s$ berechnet mit Gleichung \eqref{eqn:versatz} aus den gemessenen Winkeln $\alpha$ und $\beta$ aus Tabelle \ref{tab:brechung}.}
    \label{tab:wrtfffffffffff}
    \begin{tabular}{c | c c}
        \toprule
        Strahlversatz $s$ [$\si{\milli\meter}$] & Einfallswinkel $\alpha$ [$\si{\degree}$] & Brechungswinkel $\beta$ [$\si{\degree}$] \\
        \midrule
        $\SI{6.30(73)}{}$ 20.00 & 14.00 \\
        $\SI{8.51(73)}{}$ 25.00 & 17.00 \\
        $\SI{10.81(73)}{}$ 30.00 & 20.00 \\
        $\SI{13.21(73)}{}$ 35.00 & 23.00 \\
        $\SI{15.74(73)}{}$ 40.00 & 26.00 \\
        \bottomrule    
    \end{tabular}
\end{table}

%\begin{equation*}
%    s = \SI{0.0109(00003)}{\meter}
%\end{equation*}
Die Rechnung beinhaltet wieder fehlerbehaftete Paramter was analog zur vorherigen Brechung auf eine Gaußsche Fehlerfortpflanzung 
führt. Die Abweichung lässt sich also wie folgt errechnen
\begin{equation}
\increment s = d \cdot \sqrt{\left( \frac{\text{cos}(\alpha - \beta) (\increment \beta)}{\text{cos}^2(\beta)}\right)^2 + \left( \frac{\text{cos}(\alpha) (\increment \alpha)}{\text{cos}^2(\beta)}\right)^2}.
\end{equation}
Eine alternative Berechnung des Strahlversatz kann durch den theoretischen Wert des Brechungswinkel $\beta$ geschehen. 
Dieser wird nach Umformen der Formel \eqref{eqn:snell} gefunden, wobei es noch gilt den errechneten Brechungsindex
einzusetzen. Der Winkel $\beta$ entspricht also
\begin{equation}
    \label{eqn:yowhatever}
    \beta  = \text{arcsin} \left( \frac{\text{sin}(\alpha)}{n}\right).
\end{equation}
Dabei ergibt sich der Fehler durch eine Fehlerfortpflanzung der beiden fehlerbehafteten Terme $\alpha$ und $n$.
\begin{equation}
\increment \beta = \sqrt{\frac{\text{cos}^2(\alpha)}{n^2 - \text{sin}^2(\alpha)} (\increment \alpha)^2 + \frac{\text{sin}^2(\alpha)}{n^4 - n^2 \cdot \text{sin}^2(\alpha)} (\increment n)^2  }
\end{equation}
Die Formel zur alternativen Berechnung des Strahlversatz lautet also mit eingesetztem $\beta$ und Brechungsindex $n$
\begin{equation*}
    s = d  \cdot \frac{\text{sin} \left (  \alpha- \text{arcsin} \left( \frac{\text{sin}(\alpha)}{n}\right) \right)} {\text{cos}\left(\text{arcsin} \left( \frac{\text{sin}(\alpha)}{n}\right)\right)}.
\end{equation*}
Nun sind die mit der alternativen Methode berechneten Werte in Tabelle \ref{tab:wrtfffffffffffffffffffffffffffffff} angegeben.
\begin{table}
    \centering
    \caption{Angegeben ist der Strahlversatz $s$ berechnet mit Gleichung \eqref{eqn:versatz} aus dem gemessenen Winkeln $\protect \alpha$ aus Tabelle \ref{tab:brechung} und dem errechneten Winkel $\protect \beta_{\text{ber}}$ mit Gleichung \ref{eqn:yowhatever}.}
    \label{tab:wrtfffffffffffffffffffffffffffffff}
    \begin{tabular}{c | c c }
        \toprule
        Strahlversatz $s$ [$\si{\milli\meter}$]  & Einfallswinkel $\alpha$ [$\si{\degree}$] & Brechungswinkel $\beta_{\text{ber}}$ [$\si{\degree}$] \\
        \midrule
        $\SI{6.68(30)}{}$  & 20.00 & $\SI{13.63(32)}{}$\\
        $\SI{8.58(35)}{}$  & 25.00 & $\SI{16.93(35)}{}$\\
        $\SI{10.65(39)}{}$ & 30.00 & $\SI{20.16(38)}{}$\\
        $\SI{12.93(44)}{}$ & 35.00 & $\SI{23.28(41)}{}$\\
        $\SI{15.46(49)}{}$ & 40.00 & $\SI{26.29(45)}{}$\\
        \bottomrule    
    \end{tabular}
\end{table}
% s &= \SI{0.0109(00003)}{\meter}
\subsection{Prisma}
Für die Ablenkung $\delta$ am Prisma werden nach der Formel \eqref{eqn:deltaw} vier verschiedene Winkel benötigt. 
Da $\alpha_1$ und $\alpha_2$  aus den Messreihen entnommen werden, können daraus dann die Winkel $\beta_1$ und $\beta_2$ aus dem Brechungsgesetz und den Winkelbeziehungen hergeleitet werden.
\begin{align}
    \label{eqn:11}
    \text{sin}\alpha &= n \cdot \text{sin}\beta \\
    \label{eqn:12}
    \beta_1 + \beta_2 &= \gamma
\end{align}
Aus dem Einfallswinkel $\alpha_1$ lässt sich direkt über das Snelliussche Brechungsgesetz mit dem zuvor bestimmten Brechungsindex $n$ aus Gleichung \eqref{eqn:n}, $\beta_1$ bestimmen.
Der oben genannte Zusammenhang zu $\gamma$ gibt dann sofort den Winkel $\beta_2$. Diese Winkel sind in der Tabelle \ref{tab:prismagreenber} dargestellt.

\begin{table}
    \centering
    \caption{Berechnete Winkel $\beta_1$ und $\beta_2$ aus den gemessenen Einfalls- und Ausfallswinkeln am Prisma \ref{tab:prismagreen} also aus den Gleichungen \eqref{eqn:11} und \eqref{eqn:12} für den grünen Laser mit Wellenlänge $\protect \lambda = \SI{532}{\nano\meter}$. Der Fehler entsteht wieder durch
    den Ablesefehler der gemessenen Winkel $\alpha_1$ und $\alpha_2$.}
    \label{tab:prismagreenber}
    \begin{tabular}{c c}
        \toprule
        $\beta_1$ [$\si{\degree}$] & $\beta_2$ [$\si{\degree}$] \\
        \midrule
        $\SI{20.16(43)}{}$ & $\SI{39.84(43)}{}$\\
        $\SI{23.28(46)}{}$ & $\SI{36.71(46)}{}$\\  
        $\SI{26.29(49)}{}$ & $\SI{33.71(49)}{}$\\
        $\SI{29.16(52)}{}$ & $\SI{30.84(52)}{}$\\
        $\SI{31.86(55)}{}$ & $\SI{28.14(55)}{}$\\
        \bottomrule    
    \end{tabular}
\end{table}

Zusammen mit den Messdaten aus Tabelle \ref{tab:prismagreen} lässt sich nun mit Gleichung \eqref{eqn:deltaw} die Ablenkung $\delta$ bestimmen. Es ergeben sich die folgenden Werte in Tabelle \ref{tab:keinb}.
\begin{table}
    \centering
    \caption{Angegeben ist die Ablenkung $\delta$ bei Verwendung des grünen Lasers an einem optischen Prisma mit Wellenlänge $\protect \lambda = \SI{532}{\nano\meter}$. Berechnet wurde $\delta$ mit Gleichung \eqref{eqn:deltaw} aus dem gemessenen Einfallswinkel $\protect \alpha_1$ und Brechungswinkel $\protect \alpha_2$ aus Tabelle \ref{tab:prismagreen}. }
    \label{tab:keinb}
    \begin{tabular}{c | c c}
        \toprule
        Ablenkung $\delta_{\text{grün}}$ [$\si{\degree}$] & Einfallswinkel $\alpha_1$ [$\si{\degree}$] & Brechungswinkel $\alpha_2$ [$\si{\degree}$] \\
        \midrule
        $\SI{47.00(71)}{}$ & 30.00 & 77.00 \\
        $\SI{43.00(71)}{}$ & 35.00 & 68.00 \\
        $\SI{40.00(71)}{}$ & 40.00 & 60.00 \\ 
        $\SI{38.00(71)}{}$ & 45.00 & 53.00 \\
        $\SI{38.00(71)}{}$ & 50.00 & 48.00 \\
        \bottomrule    
    \end{tabular}
\end{table}
%\begin{equation}
%    \overline{\delta_{\text{grün}}} = \SI{41.20(32)}{\degree}.
%\end{equation}
Ganz analog kann das für die Messwerte des roten Lasers mit Wellenlänge $\lambda = \SI{635}{\nano\meter}$ ausgerechnet werden.
Die berechneten Winkel $\beta_1$ und $\beta_2$ sind in der Tabelle \label{tab:prismaredber} angegeben.

\begin{table}
    \centering
    \caption{Berechnete Winkel $\beta_1$ und $\beta_2$ aus den gemessenen Einfalls- und Ausfallswinkeln \ref{tab:prismared} also aus den Gleichungen \eqref{eqn:11} und \eqref{eqn:12} am Prisma für den roten Laser mit Wellenlänge $\protect \lambda = \SI{635}{\nano\meter}$. Der Fehler entsteht wieder durch
    den Ablesefehler der gemessenen Winkel $\alpha_1$ und $\alpha_2$.}
    \label{tab:prismaredber}
    \begin{tabular}{c c}
        \toprule
        $\beta_1$ [$\si{\degree}$] & $\beta_2$ [$\si{\degree}$] \\
        \midrule
        $\SI{20.16(43)}{}$ & $\SI{39.84(43)}{}$\\
        $\SI{23.28(46)}{}$ & $\SI{36.71(46)}{}$\\  
        $\SI{26.29(49)}{}$ & $\SI{33.71(49)}{}$\\
        $\SI{29.16(52)}{}$ & $\SI{30.84(52)}{}$\\
        $\SI{31.86(55)}{}$ & $\SI{28.14(55)}{}$\\
        \bottomrule    
    \end{tabular}
\end{table}
Es ergeben sich die folgenden Werte für $\delta$ in Tabelle \ref{tab:keinm}.
\begin{table}
    \centering
    \caption{Angegeben ist die Ablenkung $\delta$ bei Verwendung des roten Lasers an einem optischen Prisma mit Wellenlänge $\protect \lambda = \SI{635}{\nano\meter}$. Berechnet wurde $\delta$ mit Gleichung \eqref{eqn:deltaw} aus dem gemessenen Einfallswinkel $\protect \alpha_1$ und Brechungswinkel $\protect \alpha_2$ aus Tabelle \ref{tab:prismared}. }
    \label{tab:keinm}
    \begin{tabular}{c | c c}
        \toprule
        Ablenkung $\delta_{\text{rot}}$ [$\si{\degree}$] & Einfallswinkel $\alpha_1$ [$\si{\degree}$] & Brechungswinkel $\alpha_2$ [$\si{\degree}$] \\
        \midrule
        $\SI{45.50(71)}{}$ & 30.0 & 75.5\\
        $\SI{42.00(71)}{}$ & 35.0 & 67.0\\  
        $\SI{39.00(71)}{}$ &  40.0 & 59.0\\
        $\SI{37.00(71)}{}$ & 45.0 & 52.0\\
        $\SI{37.00(71)}{}$ & 50.0 & 47.0\\
        \bottomrule    
    \end{tabular}
\end{table}
%\begin{equation}
%\overline{\delta_{\text{rot}}} = \SI{40.10(32)}{\degree}.
%\end{equation}

\subsection{Beugung am Gitter}
Für diesen Versuchsteil wurden wieder der grüne und der rote Laser verwendet und das Beugungsverhalten an drei unterschiedlichen Gittern untersucht. Die Gitter besitzen
eine bestimmte Anzahl an Linien pro Millimeter. Die Gitterkonstante $d$, also der Abstand zwischen zwei nebeneinanderliegenden Linien lässt sich durch den Kehrwert bestimmen. 
Daraus folgen die Gitterkonstanten der verwendeten Gitter zu
\begin{align*}
\SI{600}{{\text{Linien}}\per\milli\meter} \quad &\to \quad d_{600} = {600^{-1}}\si{\milli\meter\per{\text{Linie}}} \\
\SI{300}{{\text{Linien}}\per\milli\meter} \quad &\to \quad d_{300} = {300^{-1}}\si{\milli\meter\per{\text{Linie}}} \\
\SI{100}{{\text{Linien}}\per\milli\meter} \quad &\to \quad d_{100} = {100^{-1}}\si{\milli\meter\per{\text{Linie}}} \\
\end{align*}
Aus den Messwerten der verschiedenen Gitter für den Beugungswinkel $\alpha$ lassen sich mit Gleichung \ref{eqn:whatev} die Wellenlänge der verwendeten Laser bestimmen.
Dabei wird zunächst zwischen den einzelnen Gittern unterschieden und am Ende über alle Wellenlängen $\lambda$ des jeweiligen Lasers gemittelt.
In den folgenden Tabellen \ref{tab:wave600} bis \ref{tab:wave100} sind die einzelnen Berechnungen für $\lambda$ aus den Messwerten aufgezeigt.

\begin{table}
    \centering
    \caption{Berechnete Wellenlängen bei einem Beugungsverhalten an einem Gitter mit Gitterkonstante $\protect d = {600^{-1}}\si{\milli\meter\per{\text{Linie}}}$. Die verwendeten Messwerte stammen dabei aus der Tabelle \ref{tab:600linien} und wurden mit Gleichung 
    \ref{eqn:whatev} ausgewertet.}
    \label{tab:wave600}
    \begin{tabular}{c c}
        \toprule
        $\lambda_{\text{grün}}$ [$\si{\nano\meter}$] & $\lambda_{\text{rot}}$ [$\si{\nano\meter}$] \\
        \midrule
        $\SI{542.61(1375)}{}$ & $\SI{637.81(1344)}{}$ \\
        $\SI{556.34(6855)}{}$ & $\SI{664.58(6700)}{}$  \\
        \bottomrule    
    \end{tabular}
\end{table}

\begin{table}
    \centering
    \caption{Berechnete Wellenlängen bei einem Beugungsverhalten an einem Gitter mit Gitterkonstante $\protect d = {300^{-1}}\si{\milli\meter\per{\text{Linie}}}$. Die verwendeten Messwerte stammen dabei aus der Tabelle \ref{tab:300linien} und wurden mit Gleichung 
    \ref{eqn:whatev} ausgewertet.}
    \label{tab:wave300}
    \begin{tabular}{c c}
        \toprule
        $\lambda_{\text{grün}}$ [$\si{\nano\meter}$] & $\lambda_{\text{rot}}$ [$\si{\nano\meter}$] \\
        \midrule
        $\SI{535.81(2872)}{}$ & $\SI{636.03(2855)}{}$ \\
        $\SI{542.98(1435)}{}$ & $\SI{636.03(1428)}{}$ \\
        $\SI{545.38(957)}{}$ & $\SI{636.03(952)}{}$  \\
        \bottomrule    
    \end{tabular}
\end{table}
\begin{table}
    \centering
    \caption{Berechnete Wellenlängen bei einem Beugungsverhalten an einem Gitter mit Gitterkonstante $\protect d = {100^{-1}}\si{\milli\meter\per{\text{Linie}}}$. Die verwendeten Messwerte stammen dabei aus der Tabelle \ref{tab:100linien} und wurden mit Gleichung 
    \ref{eqn:whatev} ausgewertet.}
    \label{tab:wave100}
    \begin{tabular}{c c}
        \toprule
        $\lambda_{\text{grün}}$ [$\si{\nano\meter}$] & $\lambda_{\text{rot}}$ [$\si{\nano\meter}$] \\
        \midrule
        $\SI{523.36(8715)}{}$ & $\SI{654.03(8708)}{}$ \\
        $\SI{545.15(4357)}{}$ & $\SI{654.03(4344)}{}$ \\
        $\SI{552.41(2904)}{}$ & $\SI{654.03(2903)}{}$ \\
        $\SI{545.14(2178)}{}$ & $\SI{654.03(2903)}{}$ \\
        $\SI{549.50(1743)}{}$ & $\SI{662.74(1741)}{}$ \\
        $\SI{552.41(1452)}{}$ & $\SI{668.54(1451)}{}$ \\
        $\SI{554.48(1245)}{}$ & $\SI{672.69(1244)}{}$ \\
        $\SI{561.48(1089)}{}$ & $\SI{675.80(1088)}{}$ \\
        \bottomrule    
    \end{tabular}
\end{table}
\newpage
Dabei entsteht der Fehler 
\begin{equation}
\increment \lambda = d \cdot \frac{\text{cos}(\alpha)}{k} \cdot \increment \alpha
\end{equation}
durch eine Fehlerfortpflanzung des Winkels $\alpha$ auf die Wellenlänge.
Die Mittelung über alle Wellenlängen des gleichen Lasers liefern
\begin{align*}
\overline{\lambda_{\text{grün}}} &= \SI{546.70(875)}{\nano\meter},\\
\overline{\lambda_{\text{rot}}} &= \SI{654.34(874)}{\nano\meter}.\\
\end{align*}
Dabei ist der Fehler durch die Ableseungenauigkeiten der Winkel wieder deutlich größerer Ordnung als der des Standardfehlers des Mittelwerts. Eine Fehlerfortpflanzung auf eine Mittelwertberechnung
sieht in diesem Fall folgendermaßen aus
\begin{equation}
\increment \lambda = \frac{1}{N} \sqrt{\sum_{i=1}^{N} (\increment \lambda_{i})^2}.
\end{equation}
Dabei ist $N$ die Anzahl aller berechneten Wellenlängen eines Lasers.


%
%Wird also der gemittelte Winkel $\bar{\alpha_1}$ und der in \ref{brechung} gefundene Brechungsindex $n$ dort eingesetzt, findet
%sich der theoretische Wert für den Winkel $\beta_1$. Dieser ist selbst fehlerbehaftet und lautet für den grünen Laser
%\begin{equation}
%   \beta_{1\text{grün}} = \SI{0.3501(00043)}{\radian}.
%\end{equation}
%Mit den Angaben für den Winkel $\gamma = \SI{60}{\degree}$ \cite{skript} und der Winkelbeziehung
%lässt sich nun $\beta_{2\text{grün}}$ berechnen
%\begin{equation}
%    \beta_{2\text{grün}} = \SI{0.6971(00043)}{\radian}.
%\end{equation}
%Die finale Berechnung von $\delta$ verlangt zuletzt die vier Winkel in die Formel einzusetzen was eine fehlerbehaftete 
%Auslenkung von 
%\begin{equation}
%    \delta = \SI{0.1745(00055)}{\radian}
%\end{equation}
%gibt. Analog liefern die gleichen Rechnungen mit Messwerten des roten Lasers folgene totale Ablenkung beim Prisma.

    