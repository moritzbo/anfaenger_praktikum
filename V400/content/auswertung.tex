\section{Auswertung}

\subsection{Reflexionsgesetz}
Die gemessen Reflexionswinkel $\alpha_2$ lassen sich der Tabelle \ref{....} entnehmen. Sie bestätigen ausnahmslos
das Reflexionsgesetz, also %oder ref einfach lol
\begin{equation*}
    \alpha_1=\alpha_2.
\end{equation*}


\subsection{Brechungsgesetze}
\label{brechung}
Die Winkel der zweiten Messreihe wurden in der Tabelle \ref{....} gesammelt. Diesen liegt jeweils ein Messfehler von
$\Delta = \SI{0.5}{\degree}$ bei welcher durch das ungenaue Ablesen kommt. Aus der Formel \eqref{....} ergibt sich dann 
der Brechungsindex für die Platte. Es bietet sich also an die gemessen Winkel für jeden der sieben Messungen in die Formel einzusetzten 
und anschließend die gefundenen Werte zu mitteln. Der Mittelwert des Brechungsindex ergibt sich zu
\begin{equation}
    \bar{n} = \SI{1.451(0020)}{}.
\end{equation}
Der Fehler resultiert nur aus der Ableseungenauigkeit. Der statistische Fehler des Mittelwerts ist deutlich geringer mit
\begin{equation*}
    \Delta n = \SI{0.009}{}
\end{equation*}
und somit breits im obrigen Fehler enthalten. Mit der Gleichung \eqref{...} lässt sich zusammen mit der Näherung $n_1 \approx 1$
und $v_1 = c$ \cite{...} die Lichtgeschwindigkeit bestimmen. Hieraus folgt
\begin{align*}
    v &= \frac{c}{n} \\
    v &= \SI{206.60(00284)e6}{\meter\per\second}
\end{align*}
Der Fehler entsteht durch eine Gaußsche Fehlerfortpflanzung die wie folgt lautet
\begin{equation}
    \increment v = \frac{c \increment n}{n^2}
\end{equation}


\subsection{Planparallele Platten}
Mit fünf Messwerten aus Tabelle \ref{..} lässt sich nun der Strahlversatz errechnen. Hierzu wird die Gleichung \eqref{...}
benutzt und die Winkel mit entsprechendem Fehler eingesetzt. Die Dicke der Platte ist dem Aufbau zu entnehmen \ref{idkwhatthisissss}.
\begin{equation*}
    s = \SI{0.0109(00003)}{\meter}
\end{equation*}
Die Rechnung beinhaltet wieder fehlerbehaftete Paramter was analog zur Aufgabe 2 \ref{brechung} auf eine Gaußsche Fehlerfortpflanzung 
fürhrt. Die Abweichung lässt sich also wie folgt errechnen....
\\

Eine alternative Berechnung des Strahlversatz kann durch den theoretischen Wert des Brechungswinkel $\beta$ geschehen. 
Dieser wird nach umformen der Formel \eqref{...} gefunden, wobei es noch gilt den in Aufgabe 2 \ref{brechung} errechneten Brechungsindex
einzusetzen. Der Winkel $\beta$ entspricht also
\begin{equation*}
    \beta  = \text{arcsin} \left( \frac{\text{sin}(\alpha)}{n}\right).
\end{equation*}
Die Formel zur alternativen Berechnung des Strahlversatz lautet also mit eingesetztem $\beta$ und Brechungsindex $n$
\begin{align*}
    s &= d  \cdot \frac{\text{sin} \left (  \alpha- \text{arcsin} \left( \frac{\text{sin}(\alpha)}{n}\right) \right)} {\text{cos}\left(\text{arcsin} \left( \frac{\text{sin}(\alpha)}{n}\right)\right)}\\
    s &= \SI{0.0109(00003)}{\meter}
\end{align*}
Auf dem ersten Blick wirken die Beiden Methode in ihrem Ergbenis sehr ähnlich. Die Ähnlichkeit Resultiert aus 
aus der Abhängigkeit vom Winkel $\beta$. Das soll heißen das in \ref{brechung} bereits der Brechungsindex $n(\alpha, \beta)$
berechnet wurde und aus diesem schließlich wieder $\beta$ selbst. 

\subsection{Prisma}
Für die Ablenkung $\delta$ am Prisma werden nach der Formel \eqref{...} vier verschiedene Winkel benötigt. 
Da $\alpha_1$ und $\alpha_2$  aus den Messreihen entnommen werden, können daraus dann die Winkel $\beta_1$ und $\beta_2$ aus dem Brechungsgesetz und den Winkelbeziehungen hergeleitet werden.
\begin{align*}
    \text{sin}\alpha &= n \cdot \text{sin}\beta \\
    \beta_1 + \beta_2 &= \gamma
\end{align*}
Wird also der gemittelte Winkel $\bar{\alpha_1}$ und der in \ref{brechung} gefundene Brechungsindex $n$ dort eingesetzt, findet
sich der theoretische Wert für den Winkel $\beta_1$. Dieser ist selbst fehlerbehaftet und lautet für den grünen Laser
\begin{equation}
   \beta_{1\text{grün}} = \SI{0.3501(00043)}{\radian}.
\end{equation}
Mit den Angaben für den Winkel $\gamma = \SI{60}{\degree}$ \cite{skript} und der Winkelbeziehung
lässt sich nun $\beta_{2\text{grün}}$ berechnen
\begin{equation}
    \beta_{2\text{grün}} = \SI{0.6971(00043)}{\radian}.
\end{equation}
Die finale Berechnung von $\delta$ verlangt zuletzt die vier Winkel in die Formel einzusetzen was eine fehlerbehaftete 
Auslenkung von 
\begin{equation}
    \delta = \SI{0.1745(00055)}{\radian}
\end{equation}
gibt. Analog liefern die gleichen Rechnungen mit Messwerten des roten Lasers folgene totale Ablenkung beim Prisma.

    