\section{Auswertung}

\subsection{Reflexionsgesetz}
Die gemessen Reflexionswinkel $\alpha_2$ lassen sich der Tabelle \ref{....} entnehmen. Sie bestätigen ausnahmslos
das Reflexionsgesetz, also %oder ref einfach lol
\begin{equation*}
    \alpha_1=\alpha_2.
\end{equation*}


\subsection{Brechungsgesetze}
\label{brechung}
Die Winkel der zweiten Messreihe wurden in der Tabelle \ref{....} gesammelt. Diesen liegt jeweils ein Messfehler von
$\Delta = \SI{0.5}{\degree}$ bei welcher durch das ungenaue Ablesen kommt. Aus der Formel \eqref{....} ergibt sich dann 
der Brechungsindex für die Platte. Es bietet sich also an die gemessen Winkel für jeden der sieben Messungen in die Formel einzusetzten 
und anschließend die gefundenen Werte zu mitteln. Der Mittelwert des Brechungsindex ergibt sich zu
\begin{equation}
    \label{eqn:n}
    \bar{n} = \SI{1.451(0020)}{}.
\end{equation}
Der Fehler resultiert nur aus der Ableseungenauigkeit. Der statistische Fehler des Mittelwerts ist deutlich geringer mit
\begin{equation*}
    \Delta n = \SI{0.009}{}
\end{equation*}
und somit breits im obrigen Fehler enthalten. Mit der Gleichung \eqref{...} lässt sich zusammen mit der Näherung $n_1 \approx 1$
und $v_1 = c$ \cite{...} die Lichtgeschwindigkeit bestimmen. Hieraus folgt
\begin{align*}
    v &= \frac{c}{n} \\
    v &= \SI{206.60(00284)e6}{\meter\per\second}
\end{align*}
Der Fehler entsteht durch eine Gaußsche Fehlerfortpflanzung die wie folgt lautet
\begin{equation}
    \increment v = \frac{c \increment n}{n^2}
\end{equation}


\subsection{Planparallele Platten}
Mit fünf Messwerten aus Tabelle \ref{..} lässt sich nun der Strahlversatz errechnen. Hierzu wird die Gleichung \eqref{...}
benutzt und die Winkel mit entsprechendem Fehler eingesetzt. Die Dicke der Platte ist dem Aufbau zu entnehmen \ref{idkwhatthisissss}.
\begin{equation*}
    s = \SI{0.0109(00003)}{\meter}
\end{equation*}
Die Rechnung beinhaltet wieder fehlerbehaftete Paramter was analog zur Aufgabe 2 \ref{brechung} auf eine Gaußsche Fehlerfortpflanzung 
fürhrt. Die Abweichung lässt sich also wie folgt errechnen....
\\

Eine alternative Berechnung des Strahlversatz kann durch den theoretischen Wert des Brechungswinkel $\beta$ geschehen. 
Dieser wird nach umformen der Formel \eqref{...} gefunden, wobei es noch gilt den in Aufgabe 2 \ref{brechung} errechneten Brechungsindex
einzusetzen. Der Winkel $\beta$ entspricht also
\begin{equation*}
    \beta  = \text{arcsin} \left( \frac{\text{sin}(\alpha)}{n}\right).
\end{equation*}
Die Formel zur alternativen Berechnung des Strahlversatz lautet also mit eingesetztem $\beta$ und Brechungsindex $n$
\begin{align*}
    s &= d  \cdot \frac{\text{sin} \left (  \alpha- \text{arcsin} \left( \frac{\text{sin}(\alpha)}{n}\right) \right)} {\text{cos}\left(\text{arcsin} \left( \frac{\text{sin}(\alpha)}{n}\right)\right)}\\
    s &= \SI{0.0109(00003)}{\meter}
\end{align*}
Auf dem ersten Blick wirken die Beiden Methode in ihrem Ergbenis sehr ähnlich. Die Ähnlichkeit Resultiert aus 
aus der Abhängigkeit vom Winkel $\beta$. Das soll heißen das in \ref{brechung} bereits der Brechungsindex $n(\alpha, \beta)$
berechnet wurde und aus diesem schließlich wieder $\beta$ selbst. 

\subsection{Prisma}
Für die Ablenkung $\delta$ am Prisma werden nach der Formel \eqref{...} vier verschiedene Winkel benötigt. 
Da $\alpha_1$ und $\alpha_2$  aus den Messreihen entnommen werden, können daraus dann die Winkel $\beta_1$ und $\beta_2$ aus dem Brechungsgesetz und den Winkelbeziehungen hergeleitet werden.
\begin{align*}
    \text{sin}\alpha &= n \cdot \text{sin}\beta \\
    \beta_1 + \beta_2 &= \gamma
\end{align*}
Aus dem Einfallswinkel $\alpha_1$ lässt sich direkt über das Snelliussche Brechungsgesetz mit dem zuvor bestimmten Brechungsindex $n$ aus Gleichung \eqref{eqn:n}, $\beta_1$ bestimmen.
Der oben genannte Zusammenhang zu $\gamma$ gibt dann sofort den Winkel $\beta_2$. Diese Winkel sind in der Tabelle \ref{tab:prismagreenber} dargestellt.

\begin{table}
    \centering
    \caption{Berechnete Winkel $\beta_1$ und $\beta_2$ aus den gemessenen Einfalls- und Ausfallswinkeln am Prisma für den grünen Laser mit Wellenlänge $\protect \lambda = \SI{532}{\nano\meter}$.}
    \label{tab:prismagreenber}
    \begin{tabular}{c c}
        \toprule
        $\beta_1$ [$\si{\degree}$] & $\beta_2$ [$\si{\degree}$] \\
        \midrule
        $\SI{20.16(43)}{}$ & $\SI{39.84(43)}{}$\\
        $\SI{23.28(46)}{}$ & $\SI{36.71(46)}{}$\\  
        $\SI{26.29(49)}{}$ & $\SI{33.71(49)}{}$\\
        $\SI{29.16(52)}{}$ & $\SI{30.84(52)}{}$\\
        $\SI{31.86(55)}{}$ & $\SI{28.14(55)}{}$\\
        \bottomrule    
    \end{tabular}
\end{table}

Zusammen mit den Messdaten aus Tabelle \ref{tab:prismagreen} lässt sich nun mit Gleichung \ref{eqn:deltaw} $\delta$ bestimmen und mitteln. Es ergibt sich
\begin{equation}
    \overline{\delta_{\text{grün}}} = \SI{41.20(32)}{\degree}.
\end{equation}
Ganz analog kann das für die Messwerte des roten Lasers mit Wellenlänge $\lambda = \SI{635}{\nano\meter}$ ausgerechnet werden.
Die berechneten Winkel $\beta_1$ und $\beta_2$ sind in der Tabelle \label{tab:prismaredber} angegeben.

\begin{table}
    \centering
    \caption{Berechnete Winkel $\beta_1$ und $\beta_2$ aus den gemessenen Einfalls- und Ausfallswinkeln am Prisma für den roten Laser mit Wellenlänge $\protect \lambda = \SI{635}{\nano\meter}$.}
    \label{tab:prismaredber}
    \begin{tabular}{c c}
        \toprule
        $\beta_1$ [$\si{\degree}$] & $\beta_2$ [$\si{\degree}$] \\
        \midrule
        $\SI{20.16(43)}{}$ & $\SI{39.84(43)}{}$\\
        $\SI{23.28(46)}{}$ & $\SI{36.71(46)}{}$\\  
        $\SI{26.29(49)}{}$ & $\SI{33.71(49)}{}$\\
        $\SI{29.16(52)}{}$ & $\SI{30.84(52)}{}$\\
        $\SI{31.86(55)}{}$ & $\SI{28.14(55)}{}$\\
        \bottomrule    
    \end{tabular}
\end{table}
Es ergibt sich ein gemittelter Wert für $\delta$ von
\begin{equation}
\overline{\delta_{\text{rot}}} = \SI{40.10(32)}{\degree}.
\end{equation}

\subsection{Beugung am Gitter}
Für diesen Versuchsteil wurden wieder der grüne und der rote Laser verwendet und das Beugungsverhalten an drei unterschiedlichen Gittern untersucht. Die Gitter besitzen
eine bestimmte Anzahl an Linien pro Millimeter. Die Gitterkonstante $d$, also der Abstand zwischen zwei nebeneinanderliegenden Linien lässt sich durch den Kehrwert bestimmen. 
Daraus folgen die Gitterkonstanten der verwendeten Gitter zu
\begin{align*}
\SI{600}{{\text{Linien}}\per\milli\meter} \quad &\to \quad d_{600} = {600^{-1}}\si{\milli\meter\per{\text{Linie}}} \\
\SI{300}{{\text{Linien}}\per\milli\meter} \quad &\to \quad d_{300} = {300^{-1}}\si{\milli\meter\per{\text{Linie}}} \\
\SI{100}{{\text{Linien}}\per\milli\meter} \quad &\to \quad d_{100} = {100^{-1}}\si{\milli\meter\per{\text{Linie}}} \\
\end{align*}
Aus den Messwerten der verschiedenen Gitter für den Beugungswinkel $\alpha$ lassen sich mit Gleichung \ref{eqn:whatev} die Wellenlänge der verwendeten Laser bestimmen.
Dabei wird zunächst zwischen den einzelnen Gittern unterschieden und am Ende über alle Wellenlängen $\lambda$ des jeweiligen Lasers gemittelt.
In den folgenden Tabellen \ref{tab:wave600} bis \ref{tab:wave100} sind die einzelnen Berechnungen für $\lambda$ aus den Messwerten aufgezeigt.

\begin{table}
    \centering
    \caption{Berechnete Wellenlängen bei einem Beugungsverhalten an einem Gitter mit Gitterkonstante $\protect d = {600^{-1}}\si{\milli\meter\per{\text{Linie}}}$.}
    \label{tab:wave600}
    \begin{tabular}{c c}
        \toprule
        $\lambda_{\text{grün}}$ [$\si{\nano\meter}$] & $\lambda_{\text{rot}}$ [$\si{\nano\meter}$] \\
        \midrule
        $\SI{542.61(1375)}{}$ & $\SI{637.81(1344)}{}$ \\
        $\SI{556.34(6855)}{}$ & $\SI{664.58(6700)}{}$  \\
        \bottomrule    
    \end{tabular}
\end{table}

\begin{table}
    \centering
    \caption{Berechnete Wellenlängen bei einem Beugungsverhalten an einem Gitter mit Gitterkonstante $\protect d = {300^{-1}}\si{\milli\meter\per{\text{Linie}}}$.}
    \label{tab:wave300}
    \begin{tabular}{c c}
        \toprule
        $\lambda_{\text{grün}}$ [$\si{\nano\meter}$] & $\lambda_{\text{rot}}$ [$\si{\nano\meter}$] \\
        \midrule
        $\SI{535.81(2872)}{}$ & $\SI{636.03(2855)}{}$ \\
        $\SI{542.98(1435)}{}$ & $\SI{636.03(1428)}{}$ \\
        $\SI{545.38(957)}{}$ & $\SI{636.03(952)}{}$  \\
        \bottomrule    
    \end{tabular}
\end{table}
\begin{table}
    \centering
    \caption{Berechnete Wellenlängen bei einem Beugungsverhalten an einem Gitter mit Gitterkonstante $\protect d = {100^{-1}}\si{\milli\meter\per{\text{Linie}}}$.}
    \label{tab:wave100}
    \begin{tabular}{c c}
        \toprule
        $\lambda_{\text{grün}}$ [$\si{\nano\meter}$] & $\lambda_{\text{rot}}$ [$\si{\nano\meter}$] \\
        \midrule
        ~ & ~ \\
        \bottomrule    
    \end{tabular}
\end{table}
Dabei entsteht der Fehler 
\begin{equation}
\increment \lambda = d \cdot \frac{\text{cos}(\alpha)}{k} \cdot \increment \alpha
\end{equation}
durch eine Fehlerfortpflanzung des Winkels $\alpha$ auf die Wellenlänge.
Die Mittelung über alle Wellenlängen des gleichen Lasers liefern
\begin{align*}
\overline{\lambda_{\text{grün}}} &= \SI{}{\nano\meter}\\
\overline{\lambda_{\text{rot}}} &= \SI{}{\nano\meter}\\
\end{align*}
Dabei ist der Fehler durch die Ableseungenauigkeiten der Winkel wieder deutlich größerer Ordnung als der des Standardfehlers des Mittelwerts. Eine Fehlerfortpflanzung auf eine Mittelwertberechnung
sieht in diesem Fall folgendermaßen aus
\begin{equation}
\increment \lambda = \frac{1}{N} \sqrt{\sum_{i=1}^{N} (\increment \lambda_{i})^2}.
\end{equation}
Dabei ist $N$ die Anzahl aller berechneten Wellenlängen eines Lasers.

%
%Wird also der gemittelte Winkel $\bar{\alpha_1}$ und der in \ref{brechung} gefundene Brechungsindex $n$ dort eingesetzt, findet
%sich der theoretische Wert für den Winkel $\beta_1$. Dieser ist selbst fehlerbehaftet und lautet für den grünen Laser
%\begin{equation}
%   \beta_{1\text{grün}} = \SI{0.3501(00043)}{\radian}.
%\end{equation}
%Mit den Angaben für den Winkel $\gamma = \SI{60}{\degree}$ \cite{skript} und der Winkelbeziehung
%lässt sich nun $\beta_{2\text{grün}}$ berechnen
%\begin{equation}
%    \beta_{2\text{grün}} = \SI{0.6971(00043)}{\radian}.
%\end{equation}
%Die finale Berechnung von $\delta$ verlangt zuletzt die vier Winkel in die Formel einzusetzen was eine fehlerbehaftete 
%Auslenkung von 
%\begin{equation}
%    \delta = \SI{0.1745(00055)}{\radian}
%\end{equation}
%gibt. Analog liefern die gleichen Rechnungen mit Messwerten des roten Lasers folgene totale Ablenkung beim Prisma.

    