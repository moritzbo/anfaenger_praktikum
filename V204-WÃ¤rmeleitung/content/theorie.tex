\section{Theoretische Grundlagen}
Befinden sich zwei Systeme in einem thermischen Gleichgewicht, also die Temperatur der beiden ist gleich, 
wird dieser Zustand auch nach beliebiger Zeit, mit gleichem Ergbenis, erneut messbar sein. Sollte ein System jedoch eine andere Temperatur besitzen, beobachtet 
man einen Energiefluss in Richtung des Systems mit niedriger Temperatur. Dieser Prozess findet in Form von Wärmeleitung, durch frei bewegliche Elektronen und Phononen statt.
\\
\newline
Für eben diesen Prozess muss sowohl die Länge $L$ des Werkstoffs, als auch der Querschnitt $A$ bekannt sein. Die Länge trennt in diesem Fall die zwei Systeme,
hier die verschiedenen Orte der Messung. Desweitern wird die Dichte $\rho$ und spezifische Wärme $c$ des enstprechenden Werkstoffs benötigt.
Die aus der Wärmeleitung resultiernende, fließende Wärmeenergie $dQ$ lässt sich also wie folgt für ein Zeitintervall $dt$ bestimmen.
\begin{equation}
    \label{eqn:itsHeadacheTimeYeeeay}
    dQ = -\kappa A \frac{\partial T}{\partial x} dt 
\end{equation}
Hierbei ist $\kappa$ ein Wert für die sogenannte Wärmeleitfähigkeit des enstprechenden Metalls. Aus der Natur der Thermodynamik 
folgt, dass nur eine natürlicher Fluss von Wärme zum kühleren System möglich ist, nicht andersherum. Ist der Gradient $\frac{\partial T}{\partial x}$ also postiv, 
wird durch das Minuszeichen die Energie eben von wärmeren System abgegeben, nicht aufgenommen.
Analog gilt für die Wärmestromdichte.
\begin{equation}
    \label{eqn:stromdichte}
    j_w = -\kappa \frac{\partial T}{\partial x}
\end{equation}
Mit $ \sigma_T = \frac{\kappa}{\rho c}$ als Temperturleitfähigkeit findet man unter Verwendung der Kontinuitätsgleichung die eindimensionale Wärmeleitungsgleichung.
\begin{equation}
    \label{eqn:1dim}
    \frac{\partial T}{\partial t} = \frac{\kappa}{\rho c} \frac{\partial^2 T}{\partial x^2}
\end{equation}
Die linke Seite der Gleichung beschreibt die Änderung der Temperatur zu jedem Zeitpunkt $t$ während der rechte Teil 
auf die momentane Änderung am Ort $x$ reduziert ist. Eine Lösung findet man mithilfe von Anfangsbedingungen und den Messdaten 
der Verschiedenen Systeme.
\\
\newline
Als Alternative Möglichkeit bietet sich eine Periodische Erwärmung mit anschließender Kühlung von einem der Systeme an.
Die Zeit für eine ganze Periode soll im folgendem T genannt werden.
Dadurch erfolgt eine wellenartige Fortpflanzung der Temperatur im Metall mit der Darstellung.
\begin{equation}
    \label{eqn:nqe}
    \text{T(x,t)}=T_{max}e^{- \sqrt{\frac{\omega \rho c}{2 \kappa}}}cos \left (\omega t - \sqrt{\frac{\omega \rho c}{2 \kappa}} \right)
\end{equation}
Die durch einen Kosinus modulierte Welle bewegt sich also mit einer Phasengeschwindigkeit von.
\begin{equation}
    \label{eqn:v}
    v_{phase}= \frac{\omega}{k} = \frac{w}{\sqrt{\frac{\omega \rho c}{2 \kappa}}} = \sqrt{\frac{2\kappa \omega}{\rho c}}
\end{equation}
Aus den Zusammenhängen des Amplitudenverhältnis $A_{nah}$ mit $A_{fern}$ lässt sich die Wärmeleitfähigkeit nach Umformen
und enstprechendem Einsetzten darstellen als.
\begin{equation}
    \label{eqn:kappa}
    \kappa = \frac{\rho c \left ( \Delta x \right )^2}{2 \Delta t \cdot ln \left (\frac{A_{nah}}{A_{fern}}\right)}
\end{equation}