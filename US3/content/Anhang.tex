\section{Anhang}

\begin{table}
    \centering
    \caption{Literaturwerte der Eigenschaften von Flüssigkeit und Prisma. \cite{skript}}
    \label{tab:lit}
    \begin{tabular}{c c c}
        \toprule
        ~  & ~ & ~\\
        Dopplerphantomflüssigkeit: & $\rho = \SI{1.15}{\gram\per\centi\meter\tothe{3}}$ & Dichte \\
        ~  &  $c_{L} = \SI{1800}{\meter\per\second}$  & Schallgeschwindigkeit \\
        ~ & $\eta = \SI{12}{\milli\pascal\second}$ & Viskosität \\
        ~ & ~ & ~ \\
        \midrule
        ~ & ~ & ~ \\
        Dopplerprisma: & $c_{P} = \SI{2700}{\meter\per\second}$  & Schallgeschwindigkeit \\
        ~ & $l = \SI{30.7}{\milli\meter}$ & Länge der Vorlaufstrecke
    \end{tabular}
\end{table}

\begin{table}
    \centering
    \caption{Frequenzverschiebungen zu dem Rohrdurchmesser $\protect d = \SI{7}{\milli\meter}$.}
    \label{tab:1}
    \begin{tabular}{c c || c c || c c}
        \hline
        \multicolumn{2}{c}{$\theta = \SI{45}{\degree}$} & \multicolumn{2}{c}{$\theta = \SI{15}{\degree}$} & \multicolumn{2}{c}{$\theta = \SI{30}{\degree}$}\\
        \midrule
        $v$ [$\si{{\text{r}}\per\minute}$]  & $\increment f$ [$\si{\hertz}$] & $v$ [$\si{{\text{r}}\per\minute}$]  & $\increment f$ [$\si{\hertz}$]&  $v$ [$\si{{\text{r}}\per\minute}$]  & $\increment f$ [$\si{\hertz}$]\\
        \midrule
        6000    &   549     & 6000  & 244  & 6000 & 366  \\ 
        6500    &   635     & 6500  & 293  & 6500 & 415  \\ 
        7000    &   842     & 7000  & 348  & 7000 & 500  \\ 
        7500    &   1001    & 7500  & 366  & 7500 & 623  \\ 
        8000    &   1367    & 8000  & 439  & 8000 & 708  \\  
        \bottomrule
    \end{tabular}
\end{table}

\begin{table}
    \centering
    \caption{Frequenzverschiebungen zu dem Rohrdurchmesser $\protect d = \SI{10}{\milli\meter}$.}
    \label{tab:2}
    \begin{tabular}{c c || c c || c c}
        \hline
        \multicolumn{2}{c}{$\theta = \SI{45}{\degree}$} & \multicolumn{2}{c}{$\theta = \SI{15}{\degree}$} & \multicolumn{2}{c}{$\theta = \SI{30}{\degree}$}\\
        \midrule
        $v$ [$\si{{\text{r}}\per\minute}$]  & $\increment f$ [$\si{\hertz}$] & $v$ [$\si{{\text{r}}\per\minute}$]  & $\increment f$ [$\si{\hertz}$]&  $v$ [$\si{{\text{r}}\per\minute}$]  & $\increment f$ [$\si{\hertz}$]\\
        \midrule
        6000    &   378    & 6000  & 434  & 6000 & 195  \\ 
        6500    &   464    & 6500  & 159  & 6500 & 244  \\ 
        7000    &   525    & 7000  & 171  & 7000 & 317  \\ 
        7500    &   910    & 7500  & 195  & 7500 & 342  \\ 
        8000    &   708    & 8000  & 220  & 8000 & 378  \\  
        \bottomrule
    \end{tabular}
\end{table}

\begin{table}
    \centering
    \caption{Frequenzverschiebungen zu dem Rohrdurchmesser $\protect d = \SI{16}{\milli\meter}$.}
    \label{tab:3}
    \begin{tabular}{c c || c c || c c}
        \toprule
        \multicolumn{2}{c}{$\theta = \SI{45}{\degree}$} & \multicolumn{2}{c}{$\theta = \SI{15}{\degree}$} & \multicolumn{2}{c}{$\theta = \SI{30}{\degree}$}\\
        \midrule
        $v$ [$\si{{\text{r}}\per\minute}$]  & $\increment f$ [$\si{\hertz}$] & $v$ [$\si{{\text{r}}\per\minute}$]  & $\increment f$ [$\si{\hertz}$]&  $v$ [$\si{{\text{r}}\per\minute}$]  & $\increment f$ [$\si{\hertz}$]\\
        \midrule
        6000    &   159    & 6000  & 61   & 6000 &  98  \\ 
        6500    &   201    & 6500  & 85   & 6500 & 110  \\ 
        7000    &   220    & 7000  & 98   & 7000 & 134  \\ 
        7500    &   250    & 7500  & 110  & 7500 & 159  \\ 
        8000    &   305    & 8000  & 110  & 8000 & 195  \\  
        \bottomrule
    \end{tabular}
\end{table}

