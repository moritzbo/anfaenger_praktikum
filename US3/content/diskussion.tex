\section{Diskussion}
Zunächst lässt sich anmerken, dass der Versuchsaufbau, insbesondere die Ultraschallsonde sehr empfindlich ist. Dies macht das Ablesen in \enquote{FlowView} sehr fehleranfällig. Desweiteren war es 
oft schwer, vor allem bei den dickeren Rohrdurchmessern Frequenzen festzustellen. Die Ausrichtung der Sonde hat schon bei kleinen Änderung eine große Auswirkung. Außerdem war bei niedriger Pumpleistung
keine bis niedrige Frequenzverschiebungen messbar. Die relativ geringen Geschwindigkeiten bei hohen Umdrehungen der Zentrifuge lassen darauf schließen, dass die Pumpleistung nicht mehr der vom Gerät angezeigten Leistung entspricht.
\\
\newline
Die berechneten Geschwindigkeiten zeigen diese Fehleranfälligkeit, denn zu erwarten wären sehr ähnliche Werte pro Durchmesser und Pumpleistung. Erkennbar ist aber auch, dass die Mittelwerte dieser Ergebnisse 
einen sinnvollen Verlauf zeigen. Da der Fluss in dem Rohr bei einer laminaren Strömung zeitlich konstant bleibt wird die Durchflussgeschwindigkeit bei größeren Durchmessern geringer und dies ist auch an den Werten erkennbar.
\\
\newline
Die Abbildungen \ref{fig:plot1} bis \ref{fig:plot3} zeigen den erwarteten linearen Verlauf zwischen Frequenzverschiebung und Geschwindigkeiten. Deutlich erkennbar ist bis auf einige Ausreißer, dass die Geschwindigkeit mit dem Rohrdurchmesser
zunimmt.



