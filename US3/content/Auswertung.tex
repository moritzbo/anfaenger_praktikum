\section{Auswertung}

Da im Folgenden für alle Rechnungen und Diagramme die Dopplerwinkel gebraucht werden, können zunächst die drei verwendeten Prismenwinkel $\theta = \SI{15}{\degree}$, $\SI{30}{\degree}$ und $\SI{45}{\degree} $ mit Gleichung
\eqref{eqn:alphafinal} umgerechnet werden. Dazu werden die Literaturwerte \cite{skript} aus der Tabelle \ref{tab:lit} entnommen und eingesetzt. Es ergeben sich die Werte
\begin{align*}
    \alpha_{15} &= \SI{80.06}{\degree} \\
    \alpha_{30} &= \SI{70.53}{\degree} \\
    \alpha_{45} &= \SI{54.74}{\degree}.
  \end{align*}

\subsection{Bestimmung der Strömungsgeschwindigkeit}
Die Strömungsgeschwindigkeiten sind je nach Rohrdurchmesser anders, somit lassen sich für alle einzelnen Rohre die Geschwindigkeiten in Abhängigkeit von den drei Dopplerwinkeln bestimmen.
Dazu wird Gleichung \eqref{eqn:deltaf} nach $v$ umgeformt. Es ergibt sich
\begin{equation}
v = \frac{c \increment f}{2 f_{0} \text{cos}(\alpha)}.
\end{equation}
Mit den gemessenen Frequenzverschiebungen $\increment f$, der Grundfrequenz $f_{0} = \SI{2}{\mega\hertz}$ und den unterschiedlichen Winkeln lässt sich nun jede Geschwindigkeit ausrechnen. 
In den folgenden Tabellen \ref{tab:4} bis \ref{tab:6} sind die errechneten Strömungsgeschwindigkeit $v_{ber}$ mit ihren Abhängigkeiten für alle Rohrdurchmesser angegeben.

\begin{table}
    \centering
    \caption{Frequenzverschiebungen und Strömungsgeschwindigkeiten in Abhängigkeit vom Prismenwinkel und der geschalteten Strömungsgeschwindigkeit bei Rohrdurchmesser $\protect d = \SI{7}{\milli\meter}$.}
    \label{tab:4}
    \begin{tabular}{c c c c}
        \toprule
        $v_{\text{gesch}}$ [$\si{{\text{r}}\per\minute}$]  & $\alpha$ [$\si{\degree}$]  & $\increment f$ [$\si{\hertz}$]  &$v_{ber}$ [$\si{\meter\per\second}$] \\
        \midrule
        6000    &   80.06    & 244   & 0.079 \\ 
        6000    &   70.53    & 366   & 0.134 \\ 
        6000    &   54.74    & 549   & 0.259 \\ 
        \midrule
        6500    &   80.06   & 293    & 0.090 \\ 
        6500    &   70.53   & 415    & 0.152 \\ 
        6500    &   54.74   & 635    & 0.300 \\ 
        \midrule
        7000    &   80.06   & 348    & 0.112 \\ 
        7000    &   70.53   & 500    & 0.183 \\ 
        7000    &   54.74   & 842    & 0.397 \\ 
        \midrule
        7500    &   80.06   & 366    & 0.118 \\ 
        7500    &   70.53   & 623    & 0.223 \\ 
        7500    &   54.74   & 1001   & 0.472 \\ 
        \midrule
        8000    &   80.06   & 439    & 0.141 \\ 
        8000    &   70.53   & 708    & 0.259 \\ 
        8000    &   54.74   & 1367   & 0.644 \\ 
        \bottomrule
    \end{tabular}
\end{table}
        
\begin{table}
    \centering
    \caption{Frequenzverschiebungen und Strömungsgeschwindigkeiten in Abhängigkeit vom Prismenwinkel und der geschalteten Strömungsgeschwindigkeit bei Rohrdurchmesser $\protect d = \SI{10}{\milli\meter}$.}
    \label{tab:5}
    \begin{tabular}{c c c c}
        \toprule
        $v_{\text{gesch}}$ [$\si{{\text{r}}\per\minute}$]  & $\alpha$ [$\si{\degree}$]  & $\increment f$ [$\si{\hertz}$] &  $v_{ber}$ [$\si{\meter\per\second}$]\\
        \midrule
        6000    &   80.06    & 434   & 0.140 \\ 
        6000    &   70.53    & 195   & 0.071 \\ 
        6000    &   54.74    & 378   & 0.178 \\ 
        \midrule
        6500    &   80.06   & 159    & 0.051 \\ 
        6500    &   70.53   & 244    & 0.089 \\ 
        6500    &   54.74   & 464    & 0.219 \\ 
        \midrule
        7000    &   80.06   & 171    & 0.055 \\ 
        7000    &   70.53   & 317    & 0.116 \\ 
        7000    &   54.74   & 525    & 0.247 \\ 
        \midrule
        7500    &   80.06   & 195    & 0.063 \\ 
        7500    &   70.53   & 342    & 0.125 \\ 
        7500    &   54.74   & 910    & 0.429 \\ 
        \midrule
        8000    &   80.06   & 220    & 0.070 \\ 
        8000    &   70.53   & 378    & 0.138 \\ 
        8000    &   54.74   & 708    & 0.333 \\ 
        \bottomrule
    \end{tabular}
\end{table}

\begin{table}
    \centering
    \caption{Frequenzverschiebungen und Strömungsgeschwindigkeiten in Abhängigkeit vom Prismenwinkel und der geschalteten Strömungsgeschwindigkeit bei Rohrdurchmesser $\protect d = \SI{16}{\milli\meter}$.}
    \label{tab:6}
    \begin{tabular}{c c c c}
        \toprule
        $v_{\text{gesch}}$ [$\si{{\text{r}}\per\minute}$]  & $\alpha$ [$\si{\degree}$]  & $\increment f$ [$\si{\hertz}$] & $v_{ber}$ [$\si{\meter\per\second}$]\\
        \midrule
        6000    &   80.06    & 61   & 0.020 \\ 
        6000    &   70.53    & 98   & 0.036 \\ 
        6000    &   54.74    & 159  & 0.075 \\ 
        \midrule
        6500    &   80.06   & 85    & 0.027 \\ 
        6500    &   70.53   & 110   & 0.040 \\ 
        6500    &   54.74   & 201   & 0.095 \\ 
        \midrule
        7000    &   80.06   & 98    & 0.032 \\ 
        7000    &   70.53   & 134   & 0.049 \\ 
        7000    &   54.74   & 220   & 0.104 \\ 
        \midrule
        7500    &   80.06   & 110   & 0.035 \\ 
        7500    &   70.53   & 159   & 0.058 \\ 
        7500    &   54.74   & 250   & 0.118 \\ 
        \midrule
        8000    &   80.06   & 110   & 0.035 \\ 
        8000    &   70.53   & 195   & 0.071 \\ 
        8000    &   54.74   & 305   & 0.144 \\ 
        \bottomrule
    \end{tabular}
\end{table}   
                            
Aufgrund der drei Winkel folgen insgesamt drei Geschwindigkeitswerte bei gleichbleibender Zentrifugendrehzahl. Über die Werte lassen sich nun Mittelwerte und ihre Fehler bestimmen.
Der Mittelwert lässt sich durch folgenden Zusammenhang berechnen
\begin{equation}
    \label{eqn:mittel}
\bar{v} = \frac{1}{N} \sum_{i=1}^{N} v_{i}
\end{equation}.
Dabei ist $N$ die Anzahl der Messwerte und $v_{i}$ sind die einzelnen berechneten Geschwindigkeiten.
Der statistische Fehler auf den Mittelwert wird über die Standardabweichung $\sigma$ bestimmt, dafür gilt
\begin{equation}
    \label{eqn:sem}
\increment \bar{v} = \frac{\sigma}{\sqrt{N}} = \sqrt{\frac{1}{N(N-1)} \sum_{i=1}^{N} (v_{i} - \bar{v})^{2}}.
\end{equation}                                       
In Tabelle \ref{tab:7} sind die gemittelten Geschwindigkeiten pro Zentrifugendrehzahl und Rohrdurchmesser aufgetragen.

\begin{table}
    \centering
    \caption{Mittelwerte der errechneten Geschwindigkeiten.}
    \label{tab:7}
    \begin{tabular}{c || c | c | c}
        \multicolumn{1}{c}{~} & \multicolumn{1}{c}{$d = \SI{7}{\milli\meter}$} & \multicolumn{1}{c}{$d = \SI{10}{\milli\meter}$} & \multicolumn{1}{c}{$d = \SI{16}{\milli\meter}$}\\
        \toprule
        $v_{\text{gesch}}$ [$\si{{\text{r}}\per\minute}$]  & $\overline{v_{\text{ber}}}$ [$\si{\meter\per\second}$] &  $\overline{v_{\text{ber}}}$ [$\si{\meter\per\second}$] & $\overline{v_{\text{ber}}}$ [$\si{\meter\per\second}$] \\
        \midrule
        6000    &  $\SI{0.157(53)}{}$    & $\SI{0.130(31)}{}$   & $\SI{0.043(16)}{}$   \\ 
        6500    &  $\SI{0.182(61)}{}$    & $\SI{0.120(50)}{}$   & $\SI{0.054(21)}{}$   \\ 
        7000    &  $\SI{0.230(86)}{}$    & $\SI{0.140(57)}{}$   & $\SI{0.061(22)}{}$   \\ 
        7500    &  $\SI{0.272(104)}{}$   & $\SI{0.205(113)}{}$  & $\SI{0.070(25)}{}$  \\ 
        8000    &  $\SI{0.348(152)}{}$   & $\SI{0.181(79)}{}$   & $\SI{0.083(32)}{}$   \\  
        \bottomrule
    \end{tabular}
\end{table}

Es lassen sich ebenfalls für feste Prismenwinkel $\alpha$ die Abhängigkeiten der Geschwindigkeit $v$ von dem Verhältnis $\increment f$/$\text{cos}(\alpha)$ graphisch darstellen. 
In den Diagrammen \ref{fig:plot1} bis \ref{fig:plot3} ist dies gezeigt.

\begin{figure}
    \centering
    \includegraphics[width=\textwidth]{build/plot1.pdf}
    \caption{Geschwindigkeiten der Strömung in Abhängigkeit der Frequenzverschiebung bei Winkel $\protect \alpha_{15}$.} 
    \label{fig:plot1}
\end{figure}
\begin{figure}
    \centering
    \includegraphics[width=\textwidth]{build/plot2.pdf}
    \caption{Geschwindigkeiten der Strömung in Abhängigkeit der Frequenzverschiebung bei Winkel $\protect \alpha_{30}$.} 
    \label{fig:plot2}
\end{figure}
\begin{figure}
    \centering
    \includegraphics[width=\textwidth]{build/plot3.pdf}
    \caption{Geschwindigkeiten der Strömung in Abhängigkeit der Frequenzverschiebung bei Winkel $\protect \alpha_{45}$.} 
    \label{fig:plot3}
\end{figure}

\subsection{Bestimmung des Strömungsprofils}
In diesem Teil des Versuches wurde das Rohr mit $d = \SI{10}{\milli\meter}$ genauer untersucht. Hierbei wurde für den festen Prismenwinkel $\theta = \SI{15}{\degree}$ die Messtiefe variiert und wieder die 
Frequenzverschiebung $\increment f$, sowie die Streuintensität gemessen. Aus dem Zusammenhang \eqref{eqn:deltaf} lässt sich wieder die Geschwindigkeit bestimmen. Das Strömungsprofil lässt sich am besten in einem Diagramm darstellen,
dabei wird jeweils entweder die Geschwindigkeit oder die Streuintensität auf der y-Achse in Abhängigkeit von der Messtiefe aufgetragen. In Diagramm \ref{fig:plot4} ist die Geschwindigkeit und in Diagramm \ref{fig:plot5} ist die Streuintensität dargestellt.
\begin{figure}
    \centering
    \includegraphics[width=\textwidth]{build/plot5.pdf}
    \caption{Geschwindigkeiten des Strömungsprofils.} 
    \label{fig:plot4}
\end{figure}
\begin{figure}
    \centering
    \includegraphics[width=\textwidth]{build/plot4.pdf}
    \caption{Streuintensitäten des Strömungsprofils.} 
    \label{fig:plot5}
\end{figure}
