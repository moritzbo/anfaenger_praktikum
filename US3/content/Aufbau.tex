\section{Aufbau}

Für den Versuch wird ein verbundenes System an Röhren mit unterschiedlichen Durchmessern verwendet. In diesen Strömungsröhren wird eine Flüssigkeit, bestehend aus Wasser, Glycerin und Glaskugeln, hindurchgepumpt. Durch die Verwendung von Ultraschall
können nun die Eigenschaften dieser Strömung ermittelt werden.
Die Flüssigkeit wird auch Dopplerphantomflüssigkeit genannt und besitzt möglichst genau angepasste akkustische Eigenschaften an den verwendeten Ultraschall. Außerdem ist die Viskosität bei dieser Zusammensetzung relativ klein und somit
stellt sich eine laminare Strömung ein. 
Angetrieben wird dieses System durch eine Zentrifugalpumpe mit verschiedenen Geschwindigkeitseinstellungen, im Folgenden sind die Geschwindigkeiten in [$\si{{\text{rounds}}\per\minute}$] angebenen.
Für die Messung der Frequenzverschiebung wird ein Ultraschall-Doppler-Generator verwendet welcher eine Ultraschallwelle von $\SI{2}{\mega\hertz}$ sendet und wieder empfängt. 
Mit dem Programm \enquote{FlowView} lassen sich die Messdaten nun darstellen und notieren.
Ein weiterer Bestandteil des Aufbaus ist ein aus Acryl gefertigtes Dopplerprisma womit sich drei Winkel von $\theta = \SI{15}{\degree}$, $\SI{30}{\degree}$ und $\SI{45}{\degree} $ einstellen lassen. 

\section{Versuchsdurchführung}

Für den ersten Teil des Versuchs werden drei Rohrabschnitte auf ihre Geschwindigkeiten untersucht, dazu wird die Ultraschall-Frequenzverschiebungen gemessen. Je Rohrabschnitt mit unterschiedlichem Durchmesser werden die Prismenwinkel
$\theta$ variiert und für jeweils fünf verschiedene Strömungsgeschwindigkeiten die Messwerte aufgezeichnet. Gewählt wurde hier ein Intervall zwischen $6000$ und $\SI{8000}{{\text{r}}\per\minute}$, da vorher kein sichtbarer Ausschlag erkennbar war.
 Die Dopplerprismen werden auf die einzelnen Abschnitte platziert und der Kontakt zum Rohr, sowie die Kontaktflächen zur Ultraschallsonde
werden mit einem Ultraschallgel versehen. Dies sorgt für einen möglichst reflektionsfreihen Übergang zwischen den Grenzflächen.
Die untersuchten Rohrabschnitte haben Innendurchmesser von $d = \SI{7}{\milli\meter}$, $\SI{10}{\milli\meter}$ und $\SI{16}{\milli\meter}$. Das \textbf{SAMPLE VOLUME} am Ultraschallgenerator wird für die Geschwindigkeitsmessung
auf \enquote{LARGE} gestellt.
\\
\newline
Bei dem zweiten Versuchsteil wird das gesamte Strömungsprofil des Rohres mit Innendurchmesser $\SI{10}{\milli\meter}$ und Prismenwinkel $\theta = \SI{15}{\degree}$ für eine Strömungsgeschwindigkeit von $\SI{70}{}$ und $\SI{45}{\percent}$ Pumpleistung untersucht. Wie zuvor wird hier ein Ultraschallgel aufgetragen. 
Notiert werden die Frequenzverschiebung $\increment f$ und Streuintensitäten $I$ in Abhängigkeit zur Messtiefe der Ultraschallsonde. Dazu lässt sich am Ultraschallgenerator eine Messtiefe in $\si{\micro\second}$ regulieren. 
Interessant für diese Messung ist nur der Bereich innerhalb des Rohres und somit wird durch die unterschiedliche Schallgeschwindigkeit in Acryl und Dopplerphantomflüssigkeit eine Start und Endmesstiefe bestimmt. Im allgemeinen
Fall liegt diese zwischen $\SI{13}{}$-$\SI{19.5}{\micro\second}$ Messtiefe, wobei die Messwerte teilweise vorher bereits nicht mehr messbar und somit ausgelassen wurden. Die Messabstände betragen $\SI{0.75}{\milli\meter}$ dies entspricht
in der Flüssigkeit $\SI{0.5}{\micro\second}$. Im Gegensatz zum vorherigen Versuchsabschnitt wird das \textbf{SAMPLE VOLUME} hier auf \enquote{SMALL} gestellt.