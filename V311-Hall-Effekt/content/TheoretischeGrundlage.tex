\section{Theoretische Grundlage}

Bei der Messung des Hall-Effekts wird eine sehr wichtige Eigenschaft der Atomstrukur von Metallen ausgenutzt.
Die Elektronen auf den äußeren Schalen der Metallatome können sich oft abspalten und sind somit in der Lage
einen Strom zu leiten. Diese Elektronen werden auch Leitungselektronen genannt und sie können von den äußeren Schalen
eines Atoms zur nächsten äußeren Schale eines anderen Atoms wandern. Die Elektronen auf der äußeren Schale die sich noch nicht gelöst haben werden Valenzelektronen
genannt. Legt man also ein Elektrisches Feld an ein solches Metall so kann ein Strom von einem Ende zum nächsten fließen.
Die Struktur von Metallen ist oft kristallin, das bedeutet sie besitzen eine räumlich periodische Strukur. In dieser Strukur liegen die einzelnen Atome sehr dicht aneinander 
was dazu führt, dass die Valenzelektronen ein gemeinsames System bilden. Dieses System lässt sich qualitativ gut durch das Pauli-Prinzip beschreiben welches besagt, dass
jedes Elektron einen anderen Quantenzustand besitzen muss, folglich muss jedes Elektron eine zwar kleine aber andere Energie besitzen. Mit der Ausnahme von Elektronen
mit entgegengesetzten Spin wo dies doch der Fall sein kann, da sie trotz gleicher Energie in einem anderen Quantenzustand sind. Die diskreten Energieniveaus der Schalen
werden durch Überlagerung von vielen dieser Niveaus zu kontinuirlichen Energiebändern. Zwischen dieses Energiebändern kann eine Lücke entstehen, die man 
als verbotene Zone bezeichnet, das bedeutet zwischen zwei Energiebändern existieren Energiebereiche die kein Elektron innerhalb des Körpers annehmen kann.
Wenn man nun ein anzelnes Atom mit seinen Energieniveaus betrachtet stellt man fest, dass eine solche verbotene Zone zwischen den Energieniveaus vorliegen kann, das bedeutet
das die Elektronen in einem niedrigeren Niveau keine Energie aufnehmen können wenn ein äußeres E-Feld angelegt wird. Sie kommen also für die Leitung nicht in Frage. 
Dies gilt grundsätzlich für alle vollständig besetzten Bänder eines Atoms, weshalb nur die Elektronen auf den nicht vollständig
besetzten äußeren Bahnen für die elektrische Leitung in Frage kommen. Dieses wird als Leitfähigkeitsband bezeichnet.
Analog darf dann für die nicht leitenden Metalle, also Isolatoren, die äußere Bahn nicht besetzt sein damit auch kein solches Leitfähigkeitsband entsteht.
Im folgenden kann man sich das Verhalten der mikroskopischen Größen eines Metallleiters anschauen.
Beispielsweise würde bei einer ideal kristallinen Gitterstruktur des Metalls keine Wechselwirkung zwischen Leitungselektronen und Atomkernen, sowie unter den Elektronen
selbst stattfinden. Eine solche ideale Struktur liegt allerdings bei den meisten Metallen nicht vor, also rechnet man mit einer endlichen Leitfähigkeit.



