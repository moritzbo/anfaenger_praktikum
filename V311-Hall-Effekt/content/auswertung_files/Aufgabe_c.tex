\subsection{Berechnung der mikroskopischen Parameter von Kupfer}
\subsubsection{Lädungsträgerdichte}
Für die Berechnung der Lädungsträgerdichte $n$ wird die Beziehung \eqref{eqn:hallspannungfuern} verwendet. Hierbei sind die 
bekannten Größen die Elementarladung $e$, die gemessene Dicke $d$ der Kupferplatte sowie die Magnetische Flussdichte $B$ wenn $I_{b}$ konstant gehalten wird.
Das bedeutet wir verwenden die entstehende Hall-Spannung bei verändertem Strom $I_{q}$.
Außerdem betrachten die gemessene Flussdichte der Remanenzkurve bei positiver Polung mit dem Fehler des Teslameters.
\begin{table}
  \centering
  \caption{Verwendete Größen zur Bestimmung von $n$}
  \label{tab:nbestimmung}
  \begin{tabular}{c c c}
    Dicke {$d \: [\si{\micro\meter}]$} & Elementarladung $e \, [\si{\coulomb}]$ & Magnetische Flussdichte $B \, [\si{\milli\tesla}]$ \\
    \midrule
    105.2   & 1.602 $\cdot 10^{-19}$ & 1230.1 ± 1 \\
    \bottomrule
  \end{tabular}
\end{table}
Wir beschränken uns bei der Auswertung auf die Messung bei zunehmendem Strom ${I_{q}}$, da der Verlauf auch bei Umpolung nahezu linear verläuft.
Es lässt sich nun durch die gegebenen Werte eine Ausgleichsgerade modellieren, wobei der entstehende Vorfaktor von $n$ abhängt.
\begin{equation}
U_{H} = A \cdot I_{q}
\end{equation}
Anhand eines Polyfits lässt sich dieser bestimmen zu
\begin{equation}
\label{eqn:a-a-wert}
A = \SI{2.29(7)e-6}{\ohm}
\end{equation}
Hierbei ist nur die Steigung der Ausgleichsgeraden interessant, da sie die Information über $n$ enthält.
Nun lässt sich mit Gleichung \eqref{eqn:hallspannungfuern} und \eqref{eqn:a-a-wert} folgendes berechnen.
\begin{equation}
%hier habe ich mir ein minus gemopppst
A = \frac{1}{n e} \cdot \frac{B}{d} \quad \to \quad n = \frac{B}{Aed}
\end{equation}
\begin{equation}
n = \SI{3.19(10)e31}{}
\end{equation}