\subsection{Berechnung der mikroskopischen Parameter von Kupfer}
\subsubsection{Lädungsträgerdichte}
Für die Berechnung der Lädungsträgerdichte $n$ wird die Beziehung \eqref{eqn:hallspannungfuern} verwendet. Hierbei sind die 
bekannten Größen die Elementarladung $e$, die gemessene Dicke $d$ der Kupferplatte sowie die Magnetische Flussdichte $B$ wenn $I_{b}$ konstant gehalten wird.
Das bedeutet wir verwenden die entstehende Hall-Spannung bei veränderten Strom $I_{q}$.
\begin{table}
  \centering
  \caption{Verwendete Größen zur Bestimmung von $n$}
  \label{tab:nbestimmung}
  \begin{tabular}{c c c}
    Dicke {$d \: [\si{\micro\meter}]$} & Elementarladung $e \, [\si{\coulomb}]$ & Magnetische Flussdichte $B \, [\si{\milli\tesla}]$ \\
    \midrule
    105.2   & 1.602 $\cdot 10^{-19}$ & 4 ± 0.1 \\
    \bottomrule
  \end{tabular}
\end{table}
