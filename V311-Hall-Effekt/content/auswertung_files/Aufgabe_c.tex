\subsection{Berechnung der mikroskopischen Parameter von Kupfer}
\subsubsection{Lädungsträgerdichte}
Für die Berechnung der Lädungsträgerdichte $n$ wird die Beziehung \eqref{eqn:hallspannungfuern} verwendet. Hierbei sind die 
bekannten Größen die Elementarladung $e$, die gemessene Dicke $d$ der Kupferplatte sowie die Magnetische Flussdichte $B$ wenn $I_{b}$ konstant gehalten wird.
Das bedeutet wir verwenden die entstehende Hall-Spannung bei verändertem Strom $I_{q}$.
Außerdem betrachten die gemessene Flussdichte der Remanenzkurve bei positiver Polung mit dem Fehler des Teslameters.
\begin{table}
  \centering
  \caption{Verwendete Größen zur Bestimmung von $n$}
  \label{tab:nbestimmung}
  \begin{tabular}{c c c}
    Dicke {$d \: [\si{\micro\meter}]$} & Elementarladung $e \, [\si{\coulomb}]$ & Magnetische Flussdichte $B \, [\si{\milli\tesla}]$ \\
    \midrule
    105.2   & 1.602 $\cdot 10^{-19}$ & 1230.1 ± 1 \\
    \bottomrule
  \end{tabular}
\end{table}
Wir beschränken uns bei der Auswertung auf die Messung bei zunehmendem Strom ${I_{q}}$, da der Verlauf auch bei Umpolung nahezu linear verläuft.
Es lässt sich nun durch die gegebenen Werte eine Ausgleichsgerade modellieren, wobei der entstehende Vorfaktor von $n$ abhängt.
\begin{equation}
U_{H} = A \cdot I_{q}
\end{equation}
Anhand eines Polyfits lässt sich dieser bestimmen zu
\begin{equation}
\label{eqn:a-a-wert}
A = \SI{2.29(7)e-6}{\ohm}
\end{equation}
Hierbei ist nur die Steigung der Ausgleichsgeraden interessant, da sie die Information über $n$ enthält.
Nun lässt sich mit Gleichung \eqref{eqn:hallspannungfuern} und \eqref{eqn:a-a-wert} folgendes berechnen.
\begin{equation}
%hier habe ich mir ein minus gemopppst
A = \frac{1}{n e} \cdot \frac{B}{d} \quad \to \quad n = \frac{B}{Aed}
\end{equation}
\begin{equation}
\label{eqn:nwert}
n = \SI{3.19(10)e28}{\per\meter\cubed}
\end{equation}
Den Fehler erhält man gemäß der Gaußschen Fehlerfortpflanzung mit den fehlerbehafteten Werten $A$ und $B$ zu.
\begin{equation}
\increment n = \sqrt{\left( \frac{1}{A e d}\right)^{2} (\increment B)^{2} + \left( \frac{B}{A^{2} e d}\right)^{2} (\increment A)^{2}}
\end{equation}
\subsubsection{Ladungsträger pro Atom}
Um die Ladungsträger pro Kupferatom zu bestimmen, also die Anzahl der Leitungselektronen, benötigt man zwei Konstanten.
Zum einen die Masse von Kupfer und die Dichte von Kupfer. Der Zusammenhang lautet wie folgt

\begin{equation}
\nonumber
\rho = N \frac{m_{\text{Cu,atom}}}{V_{\text{Cu,atom}}}
\end{equation}
Dabei ist $\rho$ die Dichte des gesamten Kupferstücks, $N$ die Teilchenanzahl, $m_{\text{Cu,atom}}$ die Masse des Kupferatoms und $V_{\text{Cu,atom}}$ das Volumen des Kupferatoms.
Zusammen mit der Definiton von der Ladungsträgerdichte wird man das Volumen allerdings los.
\begin{equation}
\nonumber
n = \frac{N}{V_{\text{Cu,atom}}}
\end{equation}
Hier ist $n$ jetzt die Ladungsträgerdichte pro Atom und durch einsetzen erhält man
\begin{equation}
\nonumber
n_{\text{Cu,atom}} = \frac{\rho}{m_{\text{Cu,atom}}}
\end{equation}
Die Zahl der Leitungselektronen pro Atom, also der Ladungsträger pro Atom $z$ erhält man jetzt bei Betrachtung aller errechneter Ladungsträger im Kupfer \eqref{eqn:nwert} pro 
Ladungsträgerdichte der einzelnen Kupferatome. Man erhält also
\begin{equation}
\nonumber
z = \frac{n}{n_{\text{Cu,atom}}} = \frac{n \cdot m_{\text{Cu,atom}}}{\rho}
\end{equation}
Die Literaturwerte der Masse und Dichte von Kupfer sind in \ref{tab:kupferlit} noch einmal notiert.
\begin{table}
  \centering
  \caption{Literaturwerte Kupfer}
  \label{tab:kupferlit}
  \begin{tabular}{c c }
    Dichte {$\rho \: [\si{\kilo\gram\per\meter\cubed}]$} & Masse $m_{\text{Cu,atom}} \, [\si{\kilo\gram}]$\\
    \midrule
    8920   & 63.4 $\cdot \, 10^{-27}$ \\
    \bottomrule
  \end{tabular}
\end{table}
Wenn man diese Werte nun einsetzt erhält man für die Anzahl der Leitungselektronen pro Atom im Kupfer folgendes.
\begin{equation}
z = \SI{0.226(7)}{}
\end{equation}
\newpage
Da hier nur die Größe $n$ fehlerbehaftet ist, kann man den Fehler auf $z$ leicht ermitteln durch
\begin{equation}
\increment z = \biggl| \frac{1}{n_{\text{Cu,atom}}}\biggr| \cdot (\increment n)
\end{equation}

\subsubsection{Mittlere Flugzeit}
Durch die Bestimmung von $n$ kann man nun zusammen mit der Messung des Widerstands $R$ und der Beziehung \eqref{eqn:widerstand} die Mittlere Flugzeit $\bar{\tau}$ bestimmen. 
Zunächst stellt man die Gleichung nach $\bar{\tau}$ um.
\begin{equation}
\bar{\tau} = \frac{2m_{0}}{e^{2}n R} \cdot \frac{\symup{L}}{\symup{Q}}
\end{equation}
Der Querschnitt $Q$ setzt sich aus der Breite $b$ und Dicke $d$ zusammen aus \ref{tab:kupfergeo}.
Die zusätzlichen Konstanten die benötigt werden sind in der folgenden Tabelle aufgelistet.
\begin{table}
  \centering
  \caption{Konstanten und Messung}
  \label{tab:konstantenrechnung}
  \begin{tabular}{c c c}
    Ruhemasse des Elektrons {$m_{0} \: [\si{\kilo\gram}]$} & Elementarladung {$e \, [\si{\coulomb}]$} & Gemessener Widerstands {$R \, [\si{\ohm}]$}\\
    \midrule
    9.109 $\cdot 10^{-31}$   & 1.602 $\cdot 10^{-19}$ & 2.60 $\pm$ 0.13\\
    \bottomrule
  \end{tabular}
\end{table}
Wenn man anschließend alle Werte einsetzt erhält man das Ergebnis.
\begin{equation}
\bar{\tau} = \SI{1.30(10)e-17}{\second}
\end{equation}
Den Fehler erhält man durch eine Gaußsche Fehlerfortpflanzung der Größen $n$, $R$, $L$ und $b$.
\begin{equation}
\increment \bar{\tau} = \sqrt{\sum_{\zeta}\left( \left( \frac{2m_{0}L}{e^{2}Rnbd} \cdot \frac{1}{\zeta}\right)^{2} (\increment \zeta)^2 \right) + \left( \frac{2m_{0}}{e^{2}Rnbd} \right)^{2} (\increment L)^{2}} \quad \text{mit } \zeta =[R, n, b]
\end{equation}

\subsubsection{Mittlere Driftgeschwindigkeit}
Nun kann man die mittlere Driftgeschwindigkeit bei gegebener Stromdichte $j$ leicht berechnen. Die benötigte Beziehung wurde bereits in \ref{sec:theo} erläutet.
Hier wird die Gleichung \eqref{eqn:stromdichteunddrift} verwendet. Bei einer Stromdichte $j = \SI{1}{\ampere\per\milli\meter\squared}$ muss man nur noch die zuvor berechneten Werte einsetzen und
die Einheiten in die SI-Einheiten umrechnen. Hier gilt
\begin{equation}
j = \SI{1}{\ampere\per\milli\meter\squared} = \SI{1e6}{\ampere\per\meter\squared}
\end{equation}
\begin{equation}
\bar{v_{d}} = - \frac{j}{n e} = \SI{0.196(6)e-3}{\meter\per\second}
\end{equation}
Also bewegen sich die Elektronen bei der gegebenen Stromdichte im Millimeter pro Sekunde Bereich. In der Diskussion werden wir auf den 
hier berechneten Wert noch genauer eingehen.
