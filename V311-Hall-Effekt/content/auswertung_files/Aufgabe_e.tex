\subsection{Hysteresekurve}
\label{sec:Hysteresekurve}

 
 Das durch einen stetigen Strom hervorgerufene Magnetfeld lässt sich als Abhängigkeit vom eben genannten Strom $\symup{I}_b$ durch eine sogenannte \textbf{Hysteresekurve} modellieren.
 Als Fehler für das wird eine Abweichung von $\pm 1 \symup{m\si{\tesla}}$ für die Messung des Magnetfeld gewertet. Also kaum wahrnehmbar bei hier sehr hohen Werten. Ebenso lässt sich der Fehler für die x-Achse, den Teil des Stroms 
 $\symup{I_b}$ durch die genaue Messvorrichtung auf der Darstellung \ref{fig:Bfeld} kaum bemerken.
 Der Ansteigende und entsprechend Abfallende Teil der Kurve liegen überraschend nahe beieinander, was natürlich für Annehmlichkeiten bezogen auf Auswertung und Berechnungen sorgt, da von man einen einzigen Wert annehmen kann.
 Die Kurve selbst wird geformt durch einen $polyfit$ aus der $numpy$ Bibliothek von Python. So entsteht ein Polynom dritten Grades. (Funktion und Parameter sind im Anhang gegeben)

\begin{figure}
    \centering
    \includegraphics[width=\textwidth]{"build/hysterese.pdf"}
    \caption{Hysteresekurve}
    \label{fig:Bfeld}
 \end{figure}
