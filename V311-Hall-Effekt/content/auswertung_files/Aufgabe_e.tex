\subsection{Aufgabe e)}
\label{sec:auswertung_hyst}

 
 Das durch einen stetigen Strom hervorgerufene Magnetfeld lässt sich als abhängigkeit vom eben genannten Strom $\symup{I}_b$ durch eine sogennante \textbf{Hysteresekurve} modellieren.
 Als Fehler für das wird eine Abweichung von $\pm 1 \symup{m\si{\tesla}}$ für die Messung des Magnetfeld gewertet. Also kaum wahrnehmbar bei hier sehr hohen Werten. Ebeneso lässt sich der Fehler für die x-Achse, den teil des Stroms 
 $\symup{I_b}$ durch die genaue Messvorrichtung auf der Darstellung \ref{fig:Bfeld} kaum bemerken.
 Der Ansteigende und enstrechend Abfallende Teil der Kurve liegen überraschend nahe beiandender, was natürlich für Annehmlichkeiten bezogen auf Auswertung und Berechnungen sorgt, da von man einen einzigen Wert annehmen kann.
 Die Kurve selbst wird geformt durch einen "polyfit" aus der "numpy" Bibliothek von Python. So entsteht ein Polynom dritten Grades mit den Parametern 

\begin{align*}
a &= -9.148899 &&(\pm 3.8297) \\
b &= 59.538658 &&(\pm 27.2499) \\
c &= 186.559181 &&(\pm 52.6997) \\
d &= 24.964638 &&(\pm 26.6440 )
\end{align*}
 und der Funnktion

\begin{figure}
    \centering
    \includegraphics[width=\textwidth]{"build/hysterese.pdf"}
    \caption{Annahme: Fehler }
    \label{fig:Bfeld}
 \end{figure}
