\subsection{Diskussion zur vorliegenden Leitung}


%
%Verweise zur Theorie
%

Wie in \ref{sec:anomhall} beschrieben kann es bei Messung der Hall-Spannung auch zu einem anomalen Hall-Effekt kommen.
Dieser würde einen entgegengesetzten Ladungsfluss verursachen und somit die gemessene Spannung reduzieren.
Bei Kupfer überwiegt allerdings die Elektronenleitung \cite[8]{V311.pdf}, sodass anzunehmen ist, dass kaum bis keine Löcher entstehen.
Dies vereinfacht die Betrachtung des gesamten Hall-Effekts bei Kupfer natürlich deutlich.
