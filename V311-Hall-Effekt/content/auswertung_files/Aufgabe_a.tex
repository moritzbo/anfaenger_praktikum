\subsection{Messung des Widerstands und Geometrie}
\label{sec:aufgabe_a}
%eventuell als equation
Die Messung des elektrischen Wiederstandes $\si{\ohm}$ und die geometrische Abmessung der metallischen Probe, in unserem Fall Kupfer sind im Folgenden angegeben.
Zur Verfügung gestellt wurde uns ein Draht der Länge $137\si{\cm}$, Dicke von $\SI{0.101(1)}{\milli\meter}$ mit einem Reinheitsgrad von $99.99$. 
Das Messmittel der Wahl ist ein Multimeter welches eine Spannung anlegt und den dadurch entstehenden Stromfluss misst.
Durch den Zusammenhang über das Ohm'sche Gesetz \eqref{eqn:ohmgesetz} rechnet der Multimeter die Werte in den Widerstand $\si{\ohm}$ um.

\begin{equation}
   \symup{U} = \symup{R} \cdot \symup{I}
\end{equation}

Der abgelesene Wert besitzt einen Fehler von $\SI{5}{\percent}$ und wir erhalten.

\begin{equation}
\label{eqn:ohmgesetz}
R = \SI{2.60(13)}{\ohm}
\end{equation}

Die Maße des Bleches sind uns gegeben, was jedoch nur die Dicke einschließt. % eventuell mit neuen Daten speisen
Der Rest bleibt uns aufgrund der Tatsache, dass der Werkstoff in eine Arte Platine eingeschlossen ist, verborgen.
Angegebene Maße für die Dicke der Platte sind exakte $0,1052\si{\mm}$.
Die gemessenen geometrischen Maße sind in der folgenden Tabelle aufgelistet, dabei wurde die Länge und die Breite
nicht gemessen und daher abgeschätzt mit einer großen Messunsicherheit. 
Diese Ungenauigkeit bezieht sich im folgenden vor allem auf die mittlere Flugdauer $\bar{\tau}$.

\begin{table}
  \centering
  \caption{Geometrische Maße des Kupfers}
  \label{tab:kupfergeo}
  \begin{tabular}{c c c}
    Dicke {$d \: [\si{\micro\meter}]$} & Breite $b \, [\si{\centi\meter}]$ & Länge $L \, [\si{\centi\meter}]$ \\
    \midrule
    105.2   & 2.5 ± 0.5 & 4 ± 0.5 \\
    \bottomrule
  \end{tabular}
\end{table}