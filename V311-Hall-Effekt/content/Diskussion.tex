\section{Diskussion}

Zunächst kann die aufgezeichneten Messungen diskutiert werden, diese lagen alle in einem realitätsnahem Bereich ohne große Ausreißer.
Auch bei der Bestimmung des Magnetischen Flusses wurde keine große Remanenz festgestellt, was vor allem an dem verwendeten Elektromagneten lag.
Dadurch wurden die Rechnungen deutlich geringer verfälscht und auch die Störspannung war kaum ausschlaggebend.
Das Spannungs-Strom-Verhältnis war wie erwartet nahezu linear und dies wurde durch mehrfaches messen auch mit Umpolung bestätigt.
\\
Trotz geringen Spannungswerten kann davon ausgegangen werden, dass kaum bis keine Löcher im Kupfer zur Leitung beigetragen haben. 
Die errechnete Teilchenträgerdichte $n$ ergibt größentechnisch Sinn, allerdings fällt die Ladungsträgeranzahl der einzelnen Atome $z$
relativ gering aus. Das ist aber keinesfalls ein unrealistisches Ergebnis, denn es scheint durchaus plausibel, dass nicht alle Atome ein Elektron zur Leitung
beitragen. Ebenfalls kann daran die Größenordnung von $n$ als glaubhaft angesehen werden, da durch nur kleine Änderungen in $n$ die Ladungsträgeranzahl eines Atoms entweder utopisch hoch wird, 
also mehr Teilchen als vorhande Teilchen im Atom vorliegen, oder so gering wird, dass kaum Elektronen geleitet werden.
\\
Aufgrund der dichten Struktur von Kupfer wird vermutet, dass eine sehr geringe mittlere Flugdauer der Fall ist, da die Elektronen sehr schnell wider auf eine Fehlstellung des
Metalls treffen und abgebremst werden. Das errechnete Ergebnis bestätigt diese Vermutung und gilt Ebenfalls für die sehr klein ausfallende freie Weglänge.
\\
Bei der Driftgeschwindigkeit handelt es sich um die Geschwindigkeit der Leitungselektronen bei denen , anders als bei völlig freien Elektronen in einem E-Feld, ein relatives geringes
Ergebnis erwartet wird. Das liegt wie zuvor bereits genannt vor allem an der Struktur des Kupfers. 
Unser Ergebnis liegt im Millimeter pro Sekunde Bereich und diese Größenordnung findet sich auch oft in der Literatur für die Driftgeschwindigkeit in verschiedenen Metallen.
