\section{Zielsetzung}
\label{sec:zielsetzung}
Die Idee des Versuches ist es, mit einem elektrischen Widerstand und der sogenannten Hallspannung, mehrere mikroskopische Parameter zu bestimmen. 
Eben benannte Messgrößen eignen sich besonders gut und werden im Verlauf des Versuches durch Messungen genauer bestimmt. Wichtige mikroskopische Parameter sind unter anderem:
\begin{align*}
\label{al:Sammlung}
&\text{Ladungsträger pro Volumen } &&- n \\
&\text{Zahl der Ladungsträger pro Atom} &&- z \\
&\text{mittlere Flugzeit} &&- \bar{\tau} \\
&\text{mittlere Driftgeschwindigkeit} &&- \bar{v}_d \hspace{0.5cm}\text{für} \hspace{0.5cm} j = 1 \frac{\si{\ampere}}{{\si{\mm}}^2} \hspace{2cm}\\
&\text{Beweglichkeit} &&-\mu \\
&\text{Totalgeschwindigkeit} &&- v \\
&\text{mittlere freie Weglänge} &&- \bar{l}
\end{align*} 

\flushleft
Diese werden in den kommenden Abschnitten berechnet und ausführlich diskutiert.
