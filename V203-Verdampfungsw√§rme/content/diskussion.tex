\section{Diskussion}

Die ermittelte Verdampfungswärme $L$ im ersten Teil des Versuchs, also bei $p \leq \SI{1}{\bar}$, ist sehr nah an dem Literaturwert \cite{Verdampfungswärme} für Wasser der Temperatur $\SI{100}{\celsius}$.
Das erzeugte Vakuum durch die Wasserstrahlpumpe war allerdings nicht perfekt und könnte somit noch verbessert werden. Dennoch lässt sich sagen, dass die Verdampfungswärme hier ziemlich genau bestimmt werden
konnte. Für die Berechnung der Verdampfungswärme wurde angenommen, dass sie konstant ist, dies stellt
sich in der Realität nur als eine grobe Annahme heraus, daher kommt es zu Ungenauigkeiten.
\newline
\\
Für den zweiten Versuch hingegen ist die Verdampfungswärme um einiges über dem vorherigen Wert, dort kam es also zu größeren Abweichungen. Das kann unteranderem an 
der ungenaueren Temperaturmessung liegen, da das Thermometer hier nicht direkt im Wasser platziert ist. Zu erwarten wäre ebenfalls ein Abfall der Verdampfungswärme bei höher werdender
Temperatur, da die Verdampfungswärme ab dem kritischen Punkt nicht mehr definiert ist. Dies lässt sich ebenfalls an den Literaturwerten für die Verdampfungswärme erkennen \cite{Verdampfungswärme}. Die Betrachtung der temperaturabhängigen Verdampfungswärme mit dem Additionsterm \ref{fig:plot6} stellt sich also als passende Beschreibung heraus.
\newline
\\
Weitere Fehlerquellen können durch die Erwärmung des Druckmessgeräts und zu schnelles Erhitzen des Wassers auftreten. Die Messungenauigkeiten der Geräte scheinen hierbei nur eine kleine Auswirkung auf die Ergebnisse zu haben. 
