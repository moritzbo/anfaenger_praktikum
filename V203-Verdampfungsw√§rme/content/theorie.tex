\section{Theoretische Grundlagen}
Stoffe haben die Möglichkeit unter bestimmten Bedingungen drei verschiedene Arten an Aggregatzuständen, also Phasen, anzunehmen.
Zu diesen zählen: fest, flüssig und gasförmig.

Maßgeblich für diese Bedingungen ist der Druck $p$ und die Temeratur $T$.  Zusammamen bilden diese Variblen
ein pT-Diagramm. Dieses Zeigt unter welchem Druck, mit welcher Temperatur,  welche Phase des Stoffes vorliegt.
Es exestieren also in jeder Phase genau zwei Freiheitsgrade ohne das sich der Zustand des Stoffes verändert, eben bis der Bereich eines anderen Bezirks erreicht ist.


% screen einfügen

Die verschieden Aggregatzuständen werden durch Kurven getrennt, im Falle vom Wechsel: gasförmig $\longleftrightarrow$ flüssig,
nennt man diese Begrenzung Dampfdurckkurve. (In der Darstellung % citen
die Kurve zwischen TP und KP)

Sollten Druck und Temnperatur einen Zustand auf eben einer solchen Kurve erzeugen existiert der Stoff in beiden Phasen.
Um so zu bleiben reduzieren sich die Freiheitsgrade um einen Grad. Dieser Parameter wird Verdampfungwärme gennant und ist von der Temperatur
abhängig. 

\subsection{Mikroskopische Betrachtung bei Verdampfung und Kondensation}
Wenn im evakuiertem Raum die Flüssigkeit zum Verdampfen grbacht wird, lässt sich ein Druckanstieg feststellen.
Dieser Resultiert aus der jetzt gasförmigen Flüssigkeit. Auf molekularer Ebene verlasen eben die Teilchen die Flüssigkeit, die genug kintische Energie 
inne haben um die Oberfläche zu verlassen. Dafür müssen sie mehr Kraft aufbringen können als die Molekularkraft des Stoffes zum Binden benötigt.
Diese Zusätliche Arbeit wird durch den Wärmevorrat des Stoffes unterstützt oder von außen hinzugefügt.
Als molare Verdampfungswärme $L$ versteht sich also das Maß an Energie, pro Mol der Flüssigkleit, um den Stoff bei gleicher Temperatur gasförmig zu machen.
Die Abgabe der aufgenommen Energie passiert durch Kondensation. Sie ist quasi das Gegenstück zur Verdampfung.
Kondensation führt der Flüssigkeit wieder Molekühle zurück, ebenso werden von der Oberfläche der Flüssigkeit wieder Teile des Gases aufgefangen.
Folglich, bei konstanten Bedingungen, verliert die Flüssigkeit eine Menge durch Phasenwechsel zu Gas, gewinnt aber auch eben gleichviel 
zurück durch Kondensation und Wiederaufnahme. Es hat sich ein Gleichgewicht mit konstantem Druck gebildet welcher Sättigungsdruck genannt wird.
Der Druck würde also bei zunehmender Temperatur, aufgrund der erhöhten kinetischen Energie und den folgloch stärkeren Stößen, ansteigen.
Trotz Änderung des Volumens bleibt der Sättigungsdruck immer, nach entsprechender Anpassung durch Verdampfung oder Kondensation, gleich.

\subsection{Darstellung durch eine passende Differentialgleichung}
Die Natur des Systems lässt auf eine Lösung der Allgemeinen Gasgleichung vermuten,
$p \cdot V = R \cdot T$
welcher aber den Druck antiproportional zum Volumen wachsen lässt. Der Zusammenhang ist hier nicht zutreffend.

% abb 2 einfügen
Um also eine passende Darstellung für den Druck zu finden lässt sich zunächst einen anderen Thermodynamischen Zusammenhang nutzen. 
Cite !!! Die Darstellung zeigt die Änderung von Druck und Volumen einer Flüssigkeit mit Wärmekapazität $C_F$ in einem reversiblen Kreisprozess.
Beginnend bei Punkt A mit einem Druck $p$, einer Temperatur $T$ und einem Volumen $V_F$ ist das System ruhig und konstant. Die Temperartur wird so
gewählt, dass die einzige Phase in der sich Die Flüssigkeit befinden kann flüssig ist.
Nach Zugabe der Energie $\Delta Q_{AB}$ in Form von Wärme steigt der Druck und mit ihr die Temperatur um jeweils $\Delta p$ bzw $\Delta T$.
Die Zugefügte Energie muss nach den Hauptsätzen der Thermodynamik im Stoff enthalten sein und es lässt sich sagen:
\begin{equation}
    \Delta Q_{AB} = C_F \cdot \Delta T
\end{equation}
Nach wiederholter Zugabe von Energie, hier als Verdampfungswärme um die Phase zu wechseln bei gleichem Druck, befindet sich das System im dritten Zustand
C. Hier ist die Flüssigkeit restlos verdampft und das Volum hat sich vergrößert. Bei der Vergrößerung wurde eine Arbeit $A_{BC}$ verrichtet. 
Da bei einem gewissen Druck $p +\Delta p$ sich das Volumen $V$  um $\Delta V$ erweitert hat, lässt sich die Arbeit also beschreiben als
\begin{equation}
\label{eqn:BC}
    -A_{BC} = (p+ \Delta p) \cdot (V_D-V_F)
\end{equation}
Hierbei wurde verwendet, dass $\Delta V$ eben der Abstand zwischen $V_F \text{und} V_D$ ist. Die Arbiet wird vom Gas verrichet um in eben diese Phase
zu kommen, folglich ist das Vorzeichen negativ. \\
Lässt man von hier das Gas auf die ursprüngliche Temperatur $T$ abkühlen, wird wiederum Energie vom System abgegeben.
Die Tatsache, dass die Molwärme eines Stoffes multipliziert mit der Temperatur die Wärmeenergie angiebt, führt anlog zu vorher auf den Zusammenhang
\begin{equation}
    -\Delta Q_{CD} = C_D \cdot \Delta T
\end{equation}
$C_D$ beschreibt die Wärmekapazität des Gases und da die Wärme hier verschwindet ist auch das Vorzeichen Negativ.
Die vierte Strecke findet bei konstanter Temperatur und konstantem Druck statt. Wie bei \eqref{eqn:BC} wird hier die Phase gewechselt, es wird also Wärmeenergie 
abgegeben.
\begin{equation}
    -A_{DA} = p \cdot (V_D - V_F)
\end{equation}
Summiert man die hinzu - und abgegebene Wärme muss diese nach dem ersten Hauptsatz der Thermodynamik gleich der summierten, verrichteten Arbeit sein.
Also gilt:
\begin{align*}
     \Delta Q_{ges} &= C_F \cdot \Delta T - C_D \cdot \Delta T + L(T) + \Delta L - L(T) \\
     \intertext{Das $-L(T)$ ensteht durch die wieder abgezogene Verdampfungswärme bei der Kondensation}
    \Delta A_{ges} &= -((p+ \Delta p) \cdot (V_D-V_F)) \cdot -(p \cdot (V_D - V_F)) \\
    \hookrightarrow \Delta A_{ges} &= \Delta p (V_D - V_F) 
\end{align*}
Gleichsetzen liefert 
\begin{equation}
    (C_F-C_D)\Delta T + \Delta L = (V_D - V_F) \Delta p
\end{equation}
Ebenso wird verlangt, dass nicht mehr Energie abgegeben wird als aufgenommen. Es lässt sich also nach dem zweiten Hauptsatz der Thermodynamik
sagen, dass die Summe aller reduzierten Wärmemengen null sein muss.
\begin{align*}
    \sum_i \frac{Q_i}{T_i} &= 0 \\
    \hookrightarrow \frac{C_F \cdot\Delta T}{T} + \frac{L + \Delta L}{T+ \Delta T}-\frac{C_D \cdot \Delta T}{T}-\frac{L}{T} &= 0
\end{align*}