\section{Theoretische Grundlagen}
Stoffe haben die Möglichkeit unter bestimmten Bedingungen drei verschiedene Arten an Aggregatzuständen, also Phasen, anzunehmen.
Zu diesen zählen: fest, flüssig und gasförmig.

Maßgeblich für diese Bedingungen ist der Druck $p$ und die Temeratur $T$.  Zusammamen bilden diese Variblen
ein pT-Diagramm. Dieses Zeigt unter welchem Druck, mit welcher Temperatur,  welche Phase des Stoffes vorliegt.
Es exestieren also in jeder Phase genau zwei Freiheitsgrade ohne das sich der Zustand des Stoffes verändert, eben bis der Bereich eines anderen Bezirks erreicht ist.


% screen einfügen

Die verschieden Aggregatzuständen werden durch Kurven getrennt, im Falle vom Wechsel: gasförmig $\longleftrightarrow$ flüssig \\
nennt man diese Kurve