\section{Diskussion}

Der Versuchsaufbau ist grundlegend für einfallendes Licht abseits des Lasers anfällig, somit muss der Dunkelstrom herausgerechnet werden. Dieser kann allerdings zwischen jeden einzelnen
Messungen variieren und sorgt somit für einen Fehler. Außerdem ist auch das Pikoamperemeter sehr empfindlich und es zeigten sich große Abweichungen zwischen dem Umskalieren von Mikro- zu Nanoampere. Es ist daher sinnvoll bei der Messung möglichst
auf einer Skalierung zu bleiben. Zudem empfiehlt es sich die Spaltausrichtung in der Halterung und zum Detektor möglichst genau zu justieren.
\\
Die Ergebnisse für den Einzelspalt stellen sich als genau heraus, die Abweichung des Spaltabstands $b$ zu seinem Literaturwert in Tabelle \ref{tab:lit} beträgt lediglich $\SI{1.3(13)}{\percent}$. Auch die Fitfunktion passt sehr gut zu den
gemessenen Werten.
\\
\newline
Bei dem Doppelspalt hingegen sind die Abweichungen auf Grund der Umskalierung von Mikro- zu Nanoampere deutlich größer. Die Abbildung zeigt das zu erwartende große Hauptmaximum, jedoch werden die Nebenmaxima nur sehr schwach dargestellt, da die 
Werte im Vergleich sehr viel kleiner sind. Bei der Theoriekurve zeigt sich eine größere Abweichung zu den Messwerten, aber ein grober Verlauf ist dennoch erkennbar. Auch bei dem Übereinanderlegen von Beugungsmuster am Einfach- und Doppelspalt kann festgestellt werden, dass
die Einzelspalte das Intensitätsmuster des Doppelspalts modulieren.
%hier evtl noch LITERATURVERGLEICH