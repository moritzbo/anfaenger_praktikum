\section{Diskussion}

Der Versuchsaufbau ist grundlegend für einfallendes Licht abseits des Lasers anfällig, somit muss diese Dunkelintensität herausgerechnet werden. Diese kann allerdings zwischen jeden einzelnen
Messungen variieren und sorgt somit für einen Fehler. Außerdem ist auch das Pikoamperemeter sehr empfindlich und es zeigten sich große Abweichungen zwischen dem Umskalieren von Mikro- zu Nanoampere. Es ist daher sinnvoll bei der Messung möglichst
auf einer Skalierung zu bleiben. Zudem empfiehlt es sich die Spaltausrichtung in der Halterung und zum Detektor möglichst genau zu justieren.
\\
\newline
Die Ergebnisse für den Einzelspalt stellen sich als genau heraus, die Abweichung des Spaltabstands $b$ zu seinem Literaturwert in Tabelle \ref{tab:lit} beträgt lediglich $\SI{1.3(13)}{\percent}$. Auch die Fitfunktion passt sehr gut zu den
gemessenen Werten.
\\
\newline
Bei dem Doppelspalt hingegen sind die Abweichungen auf Grund der Umskalierung von Mikro- zu Nanoampere deutlich größer. Die Abbildung zeigt das zu erwartene große Hauptmaximum, jedoch werden die Nebenmaxima nur sehr schwach dargestellt, da die 
Werte sehr viel kleiner sind im Vergleich. Bei der Theoriekurve zeit sich die erwartete Kurve, aber ein grober Verlauf ist dennoch erkennbar. Auch bei dem Übereinanderlegen von Beugungsmuster am Einfach- und Doppelspalt kann festgestellt werden, dass
die Einzelspalte das Intensitätsmuster des Doppelspalts moduliert.
%hier evtl noch LITERATURVERGLEICH