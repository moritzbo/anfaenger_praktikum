\subsection{Diskussion}

Bei der Untersuchung der Bragg-Bedingung wurde eine Winkelabweichung von $\increment \theta = 0.1\textdegree$ festgestellt, dieser Wert
wurde für die anschließenden Berechnungen mit einbezogen und fällt relativ klein aus. Eine kritische Winkelabweichung wäre über $1\textdegree$ und somit liegt
dieser hier gut im Rahmen.
\\
\newline
Die Aufzeichnung der Messwerte des Emissionsspektrums zeigen klar zu erkennende charakteristische Linien in Diagramm \ref{fig:plot2}. Die daraus bestimmten Halbwertsbreiten
passen ebenfalls gut in das Diagramm und besitzen einen Fehler basierend auf der Winkelabweichung von vorher. Dieser fällt bei der Halbwertsbreite allerdings klein aus.
\\
\newline
Die Energiewerte aus Tabelle \ref{tab:lol} haben einen kleinen Fehler welcher ebenfalls alleine aus der Winkelungenauigkeit entspringt. Das Auflösungsvermögen
hingegen bezieht sich jetzt auf zwei fehlerbehaftete Energien (siehe Gleichung \ref{eqn:idk}) und wird desahlb verhältnismäßig größer. Diese Abweichung ist allerdings
noch unbedenklich. Die Abschirmkonstanten für Kupfer haben ebenfalls keine große Abweichung.
\\
\newline
Bei der Betrachtung der Absoprtionsspektren wurden die Zählraten der einzelnen Stoffe in Diagramme gezeichnet. Die Bestimmung der Intensitätsmittelpunkte
ist allerdings nicht sehr genau und ist somit sehr fehleranfällig. Der Winkel an diesen Stellen konnte nur grob an den Werten abgeschätzt werden und das führt zu unterschiedlich
großen Abweichungen zu den Literaturwerte. Dabei ist beispielweise die Energieabweichung der Absorptionsenergie von Zink $E_{k,\text{abs}}^{\text{zink}}$ zum Literaturwert nur $\SI{1.7(5)}{\percent}$. Von
Brom hingegen schon $\SI{5.6(7)}{\percent}$. Die Schwierigkeit beim Bestimmen der Intensitätsmittelpunkte von Brom lässt sich leicht am Diagramm \ref{fig:plotbrom} erkennen. Hier ist die Kante über einen größeren Winkel verteilt
und der Winkel des Mittelpunkts kann nicht genau angegeben werden.
\\
\newline
Da der Fehler auf den Energien einen direkten Zusammenhang mit denen der Abschirmkonstanten $\sigma$ hat (siehe Gleichung \ref{eqn:sigmadings}) folgen dementsprechend hohe Abweichungen zu den Literaturwerten.
Diese reichen von $\SI{6.6(19)}{\percent}$ bei Zink bis zu $\SI{23.6(28)}{\percent}$ bei Brom.
\\
\newline
Über das Moseley´sche Gesetz konnte die Rydbergenergie $R_{\infty}$ und die Rydbergfrequenz $R$ bestimmt werden. Dabei hat die Bestimmung über eine
lineare Ausgleichsgerade den Literaturwert etwas übertroffen. Die Abweichung der Konstanten zum Literaturwert beträgt allerdings nur $\SI{5.80(29)}{\percent}$ und ist somit relativ genau.


