\section{Auswertung}

Die gesammelten Werte aus den Messreihen liefern Aufschluss auf die Kreisfrequenzen. Im Folgenden sind für die einzelnen Schwingungstypen und Pendellängen die gesuchten Parameter berechnet.

\subsection{Gleichsinnige Schwingung}
Zunächst werden für beide Messreihen mit den unterschiedlichen Pendellängen von $l = \SI{71}{\centi\meter}$ und $l = \SI{32}{\centi\meter}$ Mittelwerte der Periodendauern $T$ bestimmt.
Diese berechnen sich hier durch 
\begin{equation}
    \label{eqn:mean}
\bar{T} = \frac{1}{N} \sum_{i=1}^{N} T_{i},
\end{equation}
wobei $N$ die Anzahl der Messungen angibt. Der statistische Fehler auf den Mittelwert wird über die Standardabweichung $\sigma$ bestimmt, dafür gilt
\begin{equation}
    \label{eqn:sem}
\increment \bar{T} = \frac{\sigma}{\sqrt{N}} = \sqrt{\frac{1}{N(N-1)} \sum_{i=1}^{N} (T_{i} - \bar{T})^{2}}.
\end{equation}
In der Versuchsdurchführung wurden die Pendel einzeln gemessen und somit der ideale Fall ohne Feder betrachtet, dass heißt bei identischen Winkelauslenkungen.
Für die weitere Rechnung wird die Frequenz des Gesamtsystems $\omega_{+}$ benötigt, dazu kann man über alle Periodendauern $T$ einer Pendellänge mitteln.
Die errechneten Mittelwerte ergeben sich somit zu
\begin{align}
    \overline{T_{+\text{,}71}} &= \SI{1.714(5)}{\second} \\
    \overline{T_{+\text{,}32}} &= \SI{1.296(6)}{\second} 
\end{align}
Über die Gleichung \eqref{eqn:omega} können die Periodendauern in die gemessenen Kreisfrequenzen $\omega_{+}$ 
\begin{align}
    \omega_{+\text{,}71} &= \SI{3.666(11)}{\per\second}, \\
    \omega_{+\text{,}32} &= \SI{4.849(22)}{\per\second}, 
\end{align}
umgerechnet werden. Der übertragene Fehler wird durch die folgende Gaußsche Fehlerfortpflanzung bestimmt.
\begin{equation}
    \label{eqn:gauss}
\increment \omega = \frac{2 \pi}{T^2} (\increment T)
\end{equation}

Nun lassen sich des Weiteren noch die Theoriewerte, also die Eigenfrequenzen $\widetilde{\omega_{+}}$ dieses Systems mit Gleichung \eqref{eqn:omegaeasy} ermitteln. Dazu wird lediglich die jeweilige Pendellänge und die Gravitationsbeschleunigung $g$ \cite{naturkonstanten} benötigt.
\begin{align}
    \widetilde{\omega_{+\text{,}71}} &= \SI{3.716}{\per\second} \\
    \widetilde{\omega_{+\text{,}32}} &=  \SI{5.536}{\per\second}
\end{align}

\subsection{Gegenseitige Schwingung}
Mit den Messreihen \ref{tab:t-_71} und \ref{tab:t-_32} kann nun ganz analog ein Mittelwert für die Periodendauern $T_{-}$ bestimmt werden. Somit folgen wieder aus den Gleichungen \eqref{eqn:mean} und \eqref{eqn:sem} die Werte
\begin{align*}
    \overline{T_{-\text{,}71}} &= \SI{1.682(13)}{\second}, \\
    \overline{T_{-\text{,}32}} &= \SI{1.241(10)}{\second}.
\end{align*}
Wieder wird über die Gleichung \eqref{eqn:omega} die Kreisfrequenz $\omega_{-}$ ermittelt.
\begin{align}
    \omega_{-\text{,}71} &= \SI{3.735(28)}{\per\second}, \\
    \omega_{-\text{,}32} &= \SI{5.060(40)}{\per\second}, 
\end{align}
Auch hier folgt der Fehler wieder durch die gleiche Fehlerfortpflanzung \eqref{eqn:gauss}.
\\
Für die Eigenfrequenz bei der gegensinnigen Schwingung wird die Kopplungskonstante $K$ benötigt. Dafür werden die zuvor bestimmten Kreisfrequenzen $\omega$ verwendet.
Diese werden in die Gleichung \eqref{eqn:K} eingesetzt und liefern
\begin{align*}
K_{71} &= \SI{0.018(8)}{}, \\
K_{32} &= \SI{0.043(9)}{}.
\end{align*}
Die Fehler werden durch folgende Fehlerfortpflanzung bestimmt.
\begin{equation}
\increment K = \sqrt{\left(\frac{4{\omega_{+}} {\omega_{-}}^2}{({\omega_{+}}^2 + {\omega_{-}}^2)^2}\right)^2 \cdot (\increment {\omega_{+}})^2 + \left(\frac{4{\omega_{-}} {\omega_{+}}^2}{({\omega_{+}}^2 + {\omega_{-}}^2)^2}\right)^2 \cdot (\increment {\omega_{-}})^2}
\end{equation}
Dadurch ergibt sich für die Eigenfrequenzen nach Gleichung \eqref{eqn:yeah}
\begin{align*}
    \widetilde{\omega_{-\text{,}71}} &= \SI{3.7235(30)}{\per\second} \\
    \widetilde{\omega_{-\text{,}32}} &=  \SI{5.5600(50)}{\per\second}
\end{align*}

\subsection{Gekoppelte Schwingung}
Zunächst lassen sich die Mittelwerte der gemessenen Schwebungsdauern $T_{S}$ aus den Tabellen \ref{tab:tundts-_71} und \ref{tab:tundts-_32} bestimmen.
Dafür ergeben sich mit der analogen Berechnung die Werte
\begin{alignat*}{2}
    \overline{T_{S\text{,}71}} &= \SI{67.30(26)}{\second} \quad \to \quad \omega_{S\text{,}71} &&= \SI{0.0934(4)}{\per\second}, \\
    \overline{T_{S\text{,}32}} &= \SI{26.13(31)}{\second}\quad \to \quad \omega_{S\text{,}32} &&= \SI{0.2404(28)}{\per\second}. 
\end{alignat*}
Nun kann dieses Ergebnis mit einer theoretischen Überlegung verglichen werden. Aus den beiden Schwingungstypen der gegnsinnigen und gleichsinnigen Schwingung lässt sich nun eine
Schwebungsfrequenz und Periodendauer ermittelt. Dabei gilt die Gleichung \eqref{eqn:tundomega}, wobei einzeln die Werte für $l = \SI{32}{\centi\meter}$ und $l = \SI{71}{\centi\meter}$ eingesetzt werden. Das Ergebnis für die mit
der Theorie bestimmten Schwebungsfrequenzen $\tilde{\omega_{S}}$ und Schwebungsdauern $\tilde{T_{S}}$ lauten
\begin{alignat*}{2}
    \widetilde{T_{S\text{,}71}} &= \SI{91.825(40209)}{\second} \quad \to \quad \widetilde{\omega_{S\text{,}71}} &&= \SI{0.068(30)}{\per\second}, \\
    \widetilde{T_{S\text{,}32}} &= \SI{29.496(6348)}{\second}\quad \to \quad \widetilde{\omega_{S\text{,}32}} &&= \SI{0.210(50)}{\per\second}. 
\end{alignat*}

\subsection{Vergleich der Messungen und Eigenfrequenzen}
In den vorherigen Abschnitten wurden jeweils die gemessenen Frequenzen, sowie die Eigenfrequenzen bestimmt. Diese lassen sich nun auf ihre prozentuale Abweichung überprüfen. 
Ganz Allgemein gilt
\begin{equation}
        \label{eqn:lol}
    \omega_{\si{\percent}} = 100 \cdot \left( \frac{|\omega_{\text{eigen}} - \omega_{\text{mess}}|}{\omega_{\text{eigen}}} \right).
\end{equation}
Für die einzelnen Schwingungstypen sind die prozentualen Abweichungen von ihren Eigenfrequenz $\tilde{\omega}$ in den Tabellen ... und ..., abhängig von der verwendeten Pendellänge angegeben.


