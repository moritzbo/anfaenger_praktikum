\section{Zielsetzung}

Ziel ist es die Schwingungs- und Schwebungsdauer von gleichsinniger, gegensinniger und gekoppelter Schwingung zweier Pendel unterschiedlicher Länge zu bestimmen. 

\section{Theoretische Grundlagen}
Um das Verhalten von Pendelsystemen zu verstehen ist es zunächst vonnöten die einzelnen Pendel beschreiben zu können. In der folgenden mathematischen Beschreibung wird die Reibung vernachlässigt, dadurch muss 
bei dem Pendel eine möglichst reibungsfreihe Aufhängung garantiert sein. Wird ein Pendel mit Fadenlänge $l$ und Masse $m$ ausgelenkt, wirkt die Gewichtskraft dieser entgegen. Anstatt hier die vektoriellen Kräfte zu betrachten bieten 
sich hier besonders eine Betrachtung der Drehmomente also der winkelabhängigkeit an. Das durch die Gravition induzierte Drehmoment $M$ kann als 
\begin{equation}
M = D_{p} \phi
\end{equation}
beschrieben werden, wobei $D_{p}$ die Winkelrichtgröße angibt.
Außerdem gilt für das Drehmoment eines beliebigen Körpers mit einem Trägheitsmoment $J$ ein Zusammenhang mit der Winkelbeschleunigung $\ddot{\phi}$. Dieser lautet wie folgt
\begin{equation}
M = J \ddot{\phi}.
\end{equation}
Aus dem zweiten Newtonsches Gesetz folgt eine homogene lineare Differentialgleichung zweiter Ordnung
\begin{equation}
    \label{eqn:dgl}
J \ddot{\phi} + D_{p} \phi = 0.
\end{equation}
Dabei wird zusätzlich eine Kleinwinkelnäherung verwendet, wobei $\text{sin}(x) \approx x$ angenommen wird. Folglich muss in der Durchführung darauf geachtet werden, dass die Pendel nicht zu 
weit ausgelenkt werden.
Die Gleichung \eqref{eqn:dgl} hat die typische Form einer Differentialgleichung des harmonischen Oszillators und sie lässt sich durch eine harmonische Schwingung mit der Schwingungsfrequenz
\begin{equation}
\omega = \sqrt{\frac{D_{p}}{J}} = \sqrt{\frac{g}{l}}
\end{equation}
lösen. Eine harmonische Schwingung lässt sich im Allgemeinen durch eine Superposition von Sinus- und Cosinuslösungen beschreiben.
Wichtig ist hier die Unabhängigkeit der Masse von der Winkelfrequent $\omega$ und somit auch von der Periodendauer $T$ für die gilt
\begin{equation}
    \label{eqn:omega}
T = \frac{2\pi}{\omega}.
\end{equation}
\\
\newline
Nun lassen sich zwei Pendel welche mit einer Feder gekoppelt sind betrachten. Für zwei identische Pendel wirken jetzt weitere Drehmomente welche abhängig von beiden Winkeln $\phi_{1}$ und $\phi_{2}$ sind. 
Es ergeben sich die zwei gekoppelten Differentialgleichung
\begin{align}
    \label{eqn:y1}
    J \ddot{{\phi_{1}}} + D_{p} \phi_{1} &=  D_{F} (\phi_{2}-\phi_{1}), \\
    \label{eqn:y2}
    J \ddot{{\phi_{2}}} + D_{p} \phi_{2} &= D_{F} (\phi_{1} - \phi_{2}),
\end{align}
wobei $D_{F}$ eine Federkonstante ist. Die Form auf der linken Seite ist identisch zu der Gleichung \eqref{eqn:dgl}, sie beschreibt also die Schwingung der einzelnen Pendel und die rechte Seite beschreibt die Kopplung
durch die Feder. Grundlegend für die Lösung einer solchen Differentialgleichung sind passende Randbedingungen für die Winkel $\phi$. Einige Randbedingungen können die Differentialgleichung entkoppeln und somit die Lösung
deutlich vereinfachen. Im Folgenden sind einige verschiedene Schwingungstypen je nach Randbedingungen erläutert.

\subsection{Gleichsinnige Schwingung}
Bei der gleichsinnigen Schwingung werden die beiden Pendel um den gleichen Winkel in die gleiche Richtung ausgelenkt, somit gilt $\phi_{1} = \phi_{2}$. Dadurch werden die Differentialgleichung entkoppelt.
Das bedeutet auch sofort, dass bei einer perfekten Auslenkung die Feder keine Auswirkung auf die Schwingung der beiden Pendel hat. Diese Vereinfachung wird ebenfalls in der Durchführung verwendet.
Die Lösung dieser Gleichung liefert eine Schwingungsfrequenz $\omega_{+}$
\begin{equation}
    \label{eqn:omegaeasy}
    \omega_{+} = \sqrt{\frac{g}{l}},
\end{equation}
wobei das $+$ im Index die gleichsinnige Schwingung andeuten soll. Dies ist wie bereits herausgestellt die Schwingungsfrequenz welche bei einem einzelnen Pendel erwartet wird.
Die Schwingungsdauer folgt über die Gleichung \eqref{eqn:omega}, dabei ergibt sich
\begin{equation}
T_{+} = 2\pi \cdot \sqrt{\frac{l}{g}}.
\end{equation}
Eine schematische Darstellung der gleichsinnigen Schwingung ist in Abbildung \ref{fig:4} gezeigt.

\subsection{Gegensinnige Schwingung}
Die Gegensinnige Schwingung wird durch die Randbedingungen $\phi_{1} = -\phi_{2}$ charakterisiert. Eingesetzt in die beiden Differentialgleichung \eqref{eqn:y1} und \eqref{eqn:y2} folgt zwar kein entkoppeltes
System, allerdings ist die durch die Feder ausgeübete Kraft bei beiden Pendeln betragsmäßig gleich. Die Schwingung der Pendel verhalten sich also gleich mit jeweils umgekehrter Schwingrichtung und dadurch ist ebenfalls
die Frequenz von beiden bei perfekter Auslenkung identisch.
Diese Schwingungsfrequenz $\omega_{-}$ hängt nun zusätzlich von einer Kopplungskonstante $K$ ab und sieht folgendermaßen aus
\begin{equation}
    \label{eqn:yeah}
\omega_{-} = \sqrt{\frac{g+2K}{l}}.
\end{equation}
Analog folgt wieder die Schwingungsdauer $T_{-}$
\begin{equation}
    T_{-} = 2\pi \cdot \sqrt{\frac{l}{g+2K}}.
\end{equation}
Die Kopplungskonstante $K$ ist für diese Gruppe an Differentialgleichung folgendermaßen definiert
\begin{equation}
    \label{eqn:K}
    K = \frac{{\omega_{-}}^2 - {\omega_{+}}^2}{{\omega_{-}}^2 + {\omega_{+}}^2} = \frac{{T_{-}}^2 - {T_{+}}^2}{{T_{-}}^2 + {T_{+}}^2}.
\end{equation}
In Abbildung \ref{fig:5} ist die gegenläufige Schwingung dargestellt. 

\subsection{Gekoppelte Schwingung}
Der Fall einer gekoppelten Schwinung entsteht, wenn zu Beginn ein Pendel um einen Winkel ausgelenkt wird während der andere auf Null gehalten wird. Mit jeder Schwingung des zu Beginn ausgelenkten Pendels
wird die Energie über die Feder auf das andere Pendel übertragen. Ab einem gewissen Zeitpunkt hat dann das zweite Pendel die gesamte Energie erhalten und das erste Pendel hat einen Winkel von $\phi = 0$. Der Zeitraum
zwischen zwei Stillständen eines Pendels wird als Schwebungsdauer $T_{S}$ bezeichnet. Die Schwebungsfrequenz $\omega_{S}$ lässt sich als die Differenz der einzelnen Schwingungsfrequenzen angeben. Es gilt 
\begin{equation}
\omega_{S} = \omega_{+} - \omega_{-} \quad \to \quad T_{S} = \frac{T_{+} \cdot T_{-}}{T_{+} - T_{-}},
\end{equation}
wobei die Folgerung auf $T$ wieder durch Einsetzen von $\omega_{S}$ in Gleichung \ref{eqn:omega} entsteht.
Die Randbedingungen der gekoppelten Schwingung sind in Abbildung \ref{fig:6} dargestellt.