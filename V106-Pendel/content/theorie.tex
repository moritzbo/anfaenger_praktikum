\section{Zielsetzung}

Ziel ist es die Schwingungs- und Schwebungsdauer von gleichsinniger, gegensinniger und gekoppelter Schwingung zweier Pendel unterschiedlicher Länge zu bestimmen. 

\section{Theoretische Grundlagen}
Um das Verhalten von Pendelsystemen zu verstehen ist es zunächst vonnöten die einzelnen Pendel beschreiben zu können. In der folgenden mathematischen Beschreibung wird die Reibung vernachlässigt, dadurch muss 
bei dem Pendel eine möglichst reibungsfreihe Aufhängung garantiert sein. Wird ein Pendel mit Fadenlänge $l$ und Masse $m$ ausgelenkt, wirkt die Gewichtskraft dieser entgegen. Anstatt hier die vektoriellen Kräfte zu betrachten bieten 
sich hier besonders eine Betrachtung der Drehmomente also der winkelabhängigkeit an. Das durch die Gravition induzierte Drehmoment $M$ kann als 
\begin{equation}
M = D_{p} \phi
\end{equation}
beschrieben werden, wobei $D_{p}$ die Winkelrichtgröße angibt.
Außerdem gilt für das Drehmoment eines beliebigen Körpers mit einem Trägheitsmoment $J$ ein Zusammenhang mit der Winkelbeschleunigung $\ddot{\phi}$. Dieser lautet wie folgt
\begin{equation}
M = J \ddot{\phi}
\end{equation}
Aus dem zweiten Newtonsches Gesetz folgt eine homogene lineare Differentialgleichung zweiter Ordnung
\begin{equation}
    \label{eqn:dgl}
J \ddot{\phi} + D_{p} \phi = 0.
\end{equation}
Dabei wird zusätzlich eine Kleinwinkelnäherung verwendet, wobei $\text{sin}(x) \approx x$ angenommen wird. Folglich muss in der Durchführung darauf geachtet werden, dass die Pendel nicht zu 
weit ausgelenkt werden.
Die Gleichung \eqref{eqn:dgl} hat die typische Form einer Differentialgleichung des harmonischen Oszillators und sie lässt sich durch eine harmonische Schwingung mit der Schwingungsfrequenz
\begin{equation}
\omega = \sqrt{\frac{D_{p}}{J}} = \sqrt{\frac{g}{l}}
\end{equation}
lösen. Eine harmonische Schwingung lässt sich im Allgemeinen durch eine Superposition von Sinus- und Cosinuslösungen beschreiben.
Wichtig ist hier die Unabhängigkeit der Masse von der Winkelfrequent $\omega$ und somit auch von der Periodendauer $T$ für die gilt
\begin{equation}
T = \frac{2\pi}{\omega}.
\end{equation}

Nun lassen sich zwei Pendel welche mit einer Feder gekoppelt sind betrachten. Für zwei identische Pendel wirken jetzt weitere Drehmomente welche abhängig von beiden Winkeln $\phi_{1}$ und $\phi_{2}$ sind. 
Es ergeben sich die zwei gekoppelten Differentialgleichung
\begin{align*}
    J \ddot{{\phi_{1}}} + D_{p} \phi_{1} &=  D_{F} (\phi_{2}-\phi_{1}),
    J \ddot{{\phi_{2}}} + D_{p} \phi_{2} &= D_{F} (\phi_{1} - \phi_{2}),
\end{align*}
wobei $D_{F}$ eine Federkonstante ist. Die Form auf der linken Seite ist identisch zu der Gleichung \eqref{eqn:dgl}, sie beschreibt also die Schwingung der einzelnen Pendel und die rechte Seite beschreibt die Kopplung
durch die Feder. Grundlegend für die Lösung einer solchen Differentialgleichung sind passende Randbedingungen für die Winkel $\phi$. Einige Randbedingungen können die Differentialgleichung entkoppeln und somit die Lösung
deutlich vereinfachen. Im Folgenden sind einige verschiedene Schwingungstypen je nach Randbedingungen erläutert.

\subsection{Gleichsinnige Schwingung}
Bei der gleichsinnigen Schwingung werden die beiden Pendel um den gleichen Winkel in die gleiche Richtung ausgelenkt, somit gilt $\phi_{1} = \phi{2}$. Dadurch werden die 