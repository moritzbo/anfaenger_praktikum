\section{Diskussion}

Grundsätzlich entstehen bei der Versuchdürchführung einige Ablesefehler. Zum Einen spielt die Länge des Pendels eine große Rolle für die Schwingdauer und somit auch für die Frequenz und die daraus resultierenden Rechnungen. 
Bei der gegensinnigen Schwingung sollten die Pendel um genau den gleichen Winkel ausgelenkt werden, was sich in der Durchführung als schwierig herausstellt, somit entsteht mit langer Schwingzeit ein gekoppeltes Schwingsystem.
Der größte Messfehler entsteht durch die Ablesegenauigkeit der Schwingdauern $T$, da allerdings eine Mittelung über alle Messwerte durchgeführt wird können die Ergebnisse als brauchbar angesehen werden.
Erkennbar ist auch ein Unterschied zwischen den Pendellängen, auf Grund der kürzeren Schwingdauer bei $l = \SI{32}{\centi\meter}$ wird das Messen per Hand ungenauer.
\\
\newline
Die Abweichungen der gleichsinnigen- und gegensinnigen Schwingungen \ref{tab:proz71} bei Pendellänge $l = \SI{71}{\centi\meter}$ sind sehr gering und liegen unter $\SI{2}{\percent}$. Bei der Schwebung hingegen stellt sich ein deutlich größerer
Fehler heraus, dies könnte an der kleineren Messreihe von gegensinnigen Schwingdauern liegen. Außerdem wird hier der Fehler von $\omega_{+}$, $\omega_{-}$ und der Kopplungskonstante $K$ fortgepflanzt und dadurch erhöht.
\\
\newline
Bei der Pendellänge $l = \SI{32}{\centi\meter}$ bestätigt sich die Erwartung einer höheren Ablesegenauigkeit. Dies zeigt sich an den Werten für $\omega_{+{,}\si{\percent}}$ und $\omega_{-{,}\si{\percent}}$ in Tabelle \ref{tab:proz32}. Hier liegen
die Abweichungen bei knapp $\SI{10}{\percent}$. Die gemessene Schwebung liegt in diesem Fall näher an dem theoretischen Wert, hat aber weiterhin eine erkennbare Abweichung von $\SI{12.87(2433)}{\percent}$.