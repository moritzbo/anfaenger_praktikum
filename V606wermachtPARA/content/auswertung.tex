\section{Auswertung}

\subsection{Filterkurve des Selektivverstärkers}

Zur Untersuchung der Filterkurve des Selektivverstärkers werden zunächst die gemessenen Ausgangsspannungen $U_{\text{A}}$ abhängig von der Spannungsfrequenz $f$ in einem Diagramm dargestellt.  Die Originalen Messdaten sind in ihrer noch verstärkten Form in der Tabelle \ref{tab:filterkurvewerte} notiert.
Anschließend werden die Spannungen aufgrund der Verstärkung des Selektivverstärkers noch durch zehn geteilt werden.
Das Diagramm dieser Werte ist in Abbildung \ref{fig:kurve1} gezeigt.

\begin{figure}
    \centering
    \includegraphics[width=\textwidth]{build/kurve1.pdf}
    \caption{Nichtnormierte Darstellung der frequenzabhängigen Ausgangsspannung der Filterkurve.} 
    \label{fig:kurve1}
\end{figure}

Erkennbar ist ein Spannungsmaximum von $U = \SI{1}{\volt}$ bei $f = \SI{21.6}{\hertz}$. Das Maximum der gemessenen Ausgangsspannung entspricht der verwendeten Eingangsspannung. Da diese vorher nicht bekannt war kann diese hier als 
\begin{equation*}
U_{\text{E}} = \SI{1}{\volt}
\end{equation*}
identifiziert werden. In der Regeln wird die Filterkurve noch normiert angegeben, dazu wird auf der Ordinate $U_{\text{A}}$/$U_{\text{E}}$ aufgetragen. Da die vorliegende Eingangsspannung allerdings $U_{\text{E}} = \SI{1}{\volt}$ beträgt 
unterscheiden sich diese Graphen bis auf die verschwindende Einheit auf der Ordinate nicht. 
Der Vollständigkeit halber ist diese Darstellung ebenfalls noch einmal in Diagramm \ref{fig:kurve2} angegeben.
Desweiteren wird nun für die Bestimmung der realen Güte des Selektivverstärkers eine Halbwertsbreite ausgerechnet. Hierbei gilt die Gleichung .......
und in der Abbildung ist bereits die Gerade $U = (1$/$\sqrt{2}) U_{\text{A}}$ eingetragen. Die Form der Filterkurve ist eine Gaußsche Glockenkurve und daher wird hier für eine
genauere Bestimmung der Halbwertszeiten eine Fitfunktion in der Form \cite{gauss}
\begin{equation}
G_{\sigma, \beta} =  \frac{\beta}{\sqrt{\pi \sigma^2}} \cdot \text{exp}\left( -\frac{(x- \alpha)^2}{\sigma^2}\right)
\end{equation}
angesetzt.
Ein \enquote{Curvefit} der Bibliothek \enquote{Scipy} \cite{scipy} in Python liefert die folgenden Parameter
\begin{align*}
\alpha &= \SI{21.787(28)}{\hertz},\\ 
\sigma &= \SI{0.435(33)}{\hertz},\\
\beta  &= \SI{1.089(64)}{\hertz}.
\end{align*}
Das zusätzliche $\beta$ beschreibt dabei einen Korrekturterm und die Funktion mit diesen Parametern ist ebenfalls in Abbildung \ref{fig:kurve2} eingetragen. Die Schnittpunkte dieser Glockenkurve und der Geraden $U = (1$/$\sqrt{2}) U_{\text{A}}$ haben die folgenden Frequenzwerte
\begin{align*}
f_{-} &= \SI{21.42(5)}{\hertz},\\
f_{+} &= \SI{22.15(5)}{\hertz}.\\
\end{align*}
Wobei die Fehler $\increment f = \SI{0.05}{\hertz}$ Abschätzungen der Genauigkeit der Fitfunktion darstellen.
Daraus lässt sich nun zusammen mit dem Frequenzmaximum der Messwerte aus Tabelle \ref{tab:filterkurvewerte}, $f_{0} = \SI{21.6}{\hertz}$ die Güte $Q$ nach Gleichung ... berechnen.
\begin{equation}
Q = \SI{29.80(291)}{}
\end{equation}

\begin{figure}
    \centering
    \includegraphics[width=\textwidth]{build/kurve2.pdf}
    \caption{Normierte Darstellung der frequenzabhängigen Ausgangsspannung der Filterkurve.} 
    \label{fig:kurve2}
\end{figure}

\newpage
\subsection{Berechnung der Suszeptibilitäten}
 
Zunächst lassen sich für die Bestimmung der Suszeptibilitäten einige Vorbereitungen treffen. Im Folgenden werden drei unterschiedliche Methoden der Bestimmung verwendet welche andere Parameter benötigen.
Eine wichtige Größe wird der Querschnitt der untersuchten Proben darstellen, dazu wird ein Querschnitt $Q_{\text{real}}$ berechnet, den die Probe haben müsste, wenn sie aus einem Einkristall bestünde. Dabei ergibt
sich \cite{skript}
\begin{equation}
    \label{eqn:qreal}
Q_{\text{real}} = \frac{m_{\text{probe}}}{L \rho_{\text{w}}},
\end{equation}
wobei $m_{\text{probe}}$ die Masse, $L$ die Länge und $\rho_{\text{w}}$ die Dichte des Materials beschreibt.
\\
In der theoretischen Berechnung der Suszeptibilität nach Gleichung .. wird eine Beschreibung der Zahl der Momente $N$ benötigt.
Diese lässt sich durch 
\begin{equation*}
N = 2\frac{N_{A} \rho_{p}}{M_{\text{mol}}}
\end{equation*}
bestimmen, wobei $N_{A}$ die Avogadrokonstante \cite{lit} und $M_{\text{mol}}$ die molare Masse der Probe angibt. Die molare Masse bestimmt sich dabei aus \cite{molar}
\begin{equation*}
    M_{\text{mol}} = m_{\text{M}} \cdot N_{A}
\end{equation*}
mit der Molekülmasse $m_{\text{M}}$. Diese lassen sich aus dem Periodensystem für die jeweiligen Moleküle zusammenrechnen. 
In der folgenden Tabelle sind die ermittelten molaren Massen angegeben, wobei die Literaturwerte aus \cite{lit2} genommen wurden. Das $u$ steht dabei für die atomare Masseneinheit.
\begin{table}
    \caption{Molare Massen der untersuchten Materialen.}
    \centering
    \label{tab:molmass}
    \begin{tabular}{c | c c}
        \toprule
        Molekül & Molare Masse $M_{\text{mol}}$ / $\si{\g\per\mol}$ & Molekülmasse $m_{\text{M}}$ / $u$ \\
        \midrule
        $\ce{Nd2O3}$ & $\SI{336.48}{}$      & $\SI{336.48}{}$\\
        $\ce{Dy2O3}$ & $\SI{373.00}{}$    & $\SI{373.00}{}$\\
        $\ce{Gd2O3}$ & $\SI{362.50}{}$    & $\SI{362.50}{}$\\
        \bottomrule    
    \end{tabular}
\end{table}
Die Werte unterscheiden sich dabei grob nur in der Größenordnung da $u \cdot N_{A} \approx 10$.

\subsubsection{Theoretische Bestimmung der Suszeptibilität}

Die magnetische Suszeptibilität lässt sich in der Theorie mit Gleichung \eqref{eqn:CHIIII} bestimmen. Die Konstanten $\mu_0$, $\mu_B$ und $k$ werden dazu der Literatur \cite{lit2} entnommen.
Bei der Temperatur wird eine Raumtemperatur $T = \SI{293.15}{\kelvin}$ angenommen.
Anschließend werden noch die jeweiligen Landé-Faktoren benötigt, dazu gilt es die verschiedenen Drehimpulse der seltenen Erden zu bestimmen. Alle untersuchten Stoffe haben die Eigenschaft
den Drehimpuls alleine aus der 4f Schale zu erhalten. Dies folgt ebenfalls aus den Hundschen Regeln. Das \enquote{f} in der Notation legt dabei die Drehimpulsquantenzahl $l = 3$ fest und es ergeben sich
durch die magnetischen Quantenzahlen $m_l$, $m_s$ dann $2(2l+1) = 14$ Besetzungsmöglichkeiten. 
\\
Für die Berechnung der Drehimpulsbeiträge muss nur die Anzahl der Elektronen auf dieser Schale bekannt sein. Das $\ce{Nd3+}$ besitzt drei, $\ce{Gd3+}$ und $\ce{Dy3+}$ neun 4f Elektronen.
\\
Als Beispiel für eine Berechnung ist hier $\ce{Nd3+}$ erläutert.
Aus der ersten Hundschen Regel folgt, dass immer die größten Spinquantenzahlen bei der Befüllung bervorzugt werden, da die 4f Schale $(2l+1) = 6$ Plätze für positive Spinquantenzahlen der Elektronen
von $m_s = 1$/$2$ frei haben folgt daraus
\begin{equation*}
S = \frac{1}{2} + \frac{1}{2} + \frac{1}{2} =  \frac{3}{2}.
\end{equation*}
Die zweite Hundsche Regel besagt, dass auch die Drehimpulsquantenzahl maximiert wird, solange sie mit dem maximalen Spin und Pauli-Prinzip verträglich bleibt. 
\\
Da allerdings $l > \text{\enquote{Anzahl der Elektronen}}$
werden hier die drei größten Drehimpulsquantenzahlen aufgefüllt
\begin{equation*}
    L = 3 + 2 + 1 =  6.
\end{equation*}
Zuletzt gilt wegen der dritten Hundschen Regel bei weniger als halbgefüllten Schalen
\begin{equation*}
    J = |L - S| = 4.5.
\end{equation*}
Aus der Berechnungsformel \eqref{eqn:landeeq} für den Landé-Faktor ergibt sich somit
\begin{equation*}
g_{j,\text{Nd}} = \frac{8}{11}.
\end{equation*}
Die zu untersuchenden Stoffe bestehen ebenfalls aus $\ce{O2-}$ dieser trägt allerdings nicht zum Gesamtdrehimpuls bei und hat somit Landé-Faktor Null.
Ganz analog wird dies für die anderen beiden seltenen Erden durchgeführt es ergeben sich die in der Tabelle \ref{tab:lande} angegebenen Faktoren.
\begin{table}
    \caption{Landé-Faktoren der zu untersuchenden Stoffe.}
    \centering
    \label{tab:lande}
    \begin{tabular}{c | c c}
        \toprule
        Molekül & Landé-Faktor $g_J$ & $J(J+1)$ \\
        \midrule
        $\ce{Nd2O3}$ & $\SI{0.727}{}$     & $\SI{24.75}{}$\\
        $\ce{Dy2O3}$ & $\SI{1.333}{}$    & $\SI{63.75}{}$\\
        $\ce{Gd2O3}$ & $\SI{2.000}{}$    & $\SI{15.75}{}$\\
        \bottomrule    
    \end{tabular}
\end{table}
Alle zuvor gesammelten Werte lassen sich nun in die Formel \eqref{eqn:CHIIII} einsetzen und es ergeben sich die folgenden magnetischen Suszeptibilitäten
\begin{align}
\chi_{\ce{Nd2O3}} &= \SI{0.0030}{},\\
\chi_{\ce{Dy2O3}} &= \SI{0.0254}{},\\
\chi_{\ce{Gd2O3}} &= \SI{0.0138}{}.
\end{align}

\subsubsection{Berechnung aus der Brückenspannung}

Zunächst soll hier die experimentelle Berechnung durch die Brückenspannung $U_{\text{Br}}$ durchgeführt werden. Dazu wird die Näherung
$\omega^2 L^2 >> R^2$ verwendet, da eine Frequenz $f$ von einigen $\si{\kilo\hertz}$ verwendet wird und somit deutlich größer ist, als die geringfügigen Innenwiderstände der Spule.
Verwendet wird hier die Gleichung \eqref{eqn:XD} dabei wird zunächst der reale Querschnitt benötigt. Dieser ergibt sich aus der Vorbereitung in Gleichung \eqref{eqn:qreal}, dabei werden die passenden Literaturwerte aus der Tabelle
\ref{tab:proben} eingesetzt und es ergibt sich
\begin{align}
    \label{eqn:11}
    Q_{\text{real,}\ce{Nd2O3}} &= \SI{0.11}{\centi\meter\squared},\\
    \label{eqn:12}
    Q_{\text{real,}\ce{Dy2O3}} &= \SI{0.11}{\centi\meter\squared},\\
    \label{eqn:13}
    Q_{\text{real,}\ce{Gd2O3}} &= \SI{0.06}{\centi\meter\squared}.
\end{align}
Des Weiteren wird der konstante Spulenquerschnitt $F$ aus der Tabelle \ref{tab:messspule} entnommen und die Brückenspannung sind die $\Delta U$-Werte aus den Tabellen \ref{tab:1} bis \ref{tab:3}. Nun können die Eingangsspannung $U_E = \SI{0.72}{\volt}$ und die vorherigen Werte in die Gleichung \eqref{eqn:XD} eingesetzt werden.
Da pro Material drei Messungen vorgenommen wurden lassen sich die einzelnen $\chi$-Werte mitteln und es ergibt sich
\begin{align}
    \overline{\chi_{\ce{Nd2O3}}} &= \SI{0.0188(39)}{},\\
    \overline{\chi_{\ce{Dy2O3}}} &= \SI{0.6103(119)}{},\\
    \overline{\chi_{\ce{Gd2O3}}} &= \SI{0.5093(62)}{}.
\end{align}
Der Fehler auf diese Werte ergibt sich aus dem Standardfehler des Mitterlwerts gemäß
\begin{equation}
    \increment \bar{\chi} = \frac{1}{\sqrt{6}} \sqrt{\sum_{i=1}^{3} (\chi_{i} -\bar{\chi})^2 }.
\end{equation}

\subsubsection{Berechnung aus der Abgleichbedingung}

Als zweite Methode der Bestimmung der magnetischen Suszeptibilität wird die Abgleichbedingung verwendet. Dazu sind die Widerstandsunterschiede des Potentiometers von Interesse.
Es gelten wie zuvor die gleichen Werte für $Q_{\text{real}}$ aus den Gleichungen \eqref{eqn:11} bis \eqref{eqn:13}, sowie die gleiche Spulenquerschnittsfläche $F$. Der Innenwiderstand $R_{3}$ beträgt $\SI{1000}{\ohm}$.
Zuletzt werden die Widerstandsunterschiede aus den Tabellen \ref{tab:1} bis \ref{tab:3} der jeweiligen Stoffe abgelesen und in Gleichung \eqref{eqn:werdasliestistOMEGAundGIGAdummXD} eingesetzt.
Analog werden hier auch direkt die Mitterlwerte der Suszeptibilitäten angegeben.
\begin{align}
    \overline{\chi_{\ce{Nd2O3}}} &= \SI{0.0013(1)}{},\\
    \overline{\chi_{\ce{Dy2O3}}} &= \SI{0.0237(1)}{},\\
    \overline{\chi_{\ce{Gd2O3}}} &= \SI{0.0200(2)}{}.
\end{align}