\section{Auswertung}

\subsection{Filterkurve des Selektivverstärkers}

Zur Untersuchung der Filterkurve des Selektivverstärkers werden zunächst die gemessenen Ausgangsspannungen $U_{\text{A}}$ abhängig von der Spannungsfrequenz $f$ in einem Diagramm dargestellt.  Die Originalen Messdaten sind in ihrer noch verstärkten Form in der Tabelle \ref{tab:filterkurvewerte} notiert.
Anschließend werden die Spannungen aufgrund der Verstärkung des Selektivverstärkers noch durch zehn geteilt werden.
Das Diagramm dieser Werte ist in Abbildung \ref{fig:kurve1} gezeigt.

\begin{figure}
    \centering
    \includegraphics[width=\textwidth]{build/kurve1.pdf}
    \caption{Nichtnormierte Darstellung der frequenzabhängigen Ausgangsspannung der Filterkurve.} 
    \label{fig:kurve1}
\end{figure}

Erkennbar ist ein Spannungsmaximum von $U = \SI{1}{\volt}$ bei $f = \SI{21.6}{\hertz}$. Das Maximum der gemessenen Ausgangsspannung entspricht der verwendeten Eingangsspannung. Da diese vorher nicht bekannt war kann diese hier als 
\begin{equation*}
U_{\text{E}} = \SI{1}{\volt}
\end{equation*}
identifiziert werden. In der Regeln wird die Filterkurve noch normiert angegeben, dazu wird auf der Ordinate $U_{\text{A}}$/$U_{\text{E}}$ aufgetragen. Da die vorliegende Eingangsspannung allerdings $U_{\text{E}} = \SI{1}{\volt}$ beträgt 
unterscheiden sich diese Graphen bis auf die verschwindende Einheit auf der Ordinate nicht. 
Der Vollständigkeit halber ist diese Darstellung ebenfalls noch einmal in Diagramm \ref{fig:kurve2} angegeben.
Desweiteren wird nun für die Bestimmung der realen Güte des Selektivverstärkers eine Halbwertsbreite ausgerechnet. Hierbei gilt die Gleichung .......
und in der Abbildung ist bereits die Gerade $U = (1$/$\sqrt{2}) U_{\text{A}}$ eingetragen. Die Form der Filterkurve ist eine Gaußsche Glockenkurve und daher wird hier für eine
genauere Bestimmung der Halbwertszeiten eine Fitfunktion in der Form \cite{gauss}
\begin{equation}
G_{\sigma, \beta} =  \frac{\beta}{\sqrt{\pi \sigma^2}} \cdot \text{exp}\left( -\frac{(x- \alpha)^2}{\sigma^2}\right)
\end{equation}
angesetzt.
Ein \enquote{Curvefit} der Bibliothek \enquote{Scipy} \cite{scipy} in Python liefert die folgenden Parameter
\begin{align*}
\alpha &= \SI{21.787(28)}{\hertz},\\ 
\sigma &= \SI{0.435(33)}{\hertz},\\
\beta  &= \SI{1.089(64)}{\hertz}.\\
\end{align*}
Das zusätzliche $\beta$ beschreibt dabei einen Korrekturterm und die Funktion mit diesen Parametern ist ebenfalls in Abbildung \ref{fig:kurve2} eingetragen. Die Schnittpunkte dieser Glockenkurve und der Geraden $U = (1$/$\sqrt{2}) U_{\text{A}}$ haben die folgenden Frequenzwerte
\begin{align*}
f_{-} &= \SI{21.42(5)}{\hertz},\\
f_{+} &= \SI{22.15(5)}{\hertz}.\\
\end{align*}
Wobei die Fehler $\increment f = \SI{0.05}{\hertz}$ Abschätzungen der Genauigkeit der Fitfunktion darstellen.
Daraus lässt sich nun zusammen mit dem Frequenzmaximum der Messwerte aus Tabelle \ref{tab:filterkurvewerte}, $f_{0} = \SI{21.6}{\hertz}$ die Güte $Q$ nach Gleichung ... berechnen.
\begin{equation}
Q = \SI{29.80(291)}{}
\end{equation}

\begin{figure}
    \centering
    \includegraphics[width=\textwidth]{build/kurve2.pdf}
    \caption{Normierte Darstellung der frequenzabhängigen Ausgangsspannung der Filterkurve.} 
    \label{fig:kurve2}
\end{figure}

\subsection{Berechnung der Suszeptibilitäten}
 
Zunächst lassen sich für die Bestimmung der Suszeptibilitäten einige Vorbereitungen treffen. Im Folgenden werden drei unterschiedliche Methoden der Bestimmung verwendet welche andere Parameter benötigen.
Eine wichtige Größe wird der Querschnitt der untersuchten Proben darstellen, dazu müssen folgende Annahmen getroffen werden.


