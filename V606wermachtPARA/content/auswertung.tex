\section{Auswertung}

\subsection{Filterkurve des Selektivverstärkers}

Zur Untersuchung der Filterkurve des Selektivverstärkers werden zunächst die gemessenen Ausgangsspannungen $U_{\text{A}}$ abhängig von der Spannungsfrequenz $f$ in einem Diagramm dargestellt.  Die Originalen Messdaten sind in ihrer noch verstärkten Form in der Tabelle \ref{tab:filterkurvewerte} notiert.
Anschließend werden die Spannungen aufgrund der Verstärkung des Selektivverstärkers noch durch zehn geteilt werden.
Das Diagramm dieser Werte ist in Abbildung \ref{fig:kurve1} gezeigt.

\begin{figure}
    \centering
    \includegraphics[width=\textwidth]{build/kurve1.pdf}
    \caption{Nichtnormierte Darstellung der frequenzabhängigen Ausgangsspannung der Filterkurve.} 
    \label{fig:kurve1}
\end{figure}

Erkennbar ist ein Spannungsmaximum von $U = \SI{1}{\volt}$ bei $f = \SI{21.6}{\hertz}$. Das Maximum der gemessenen Ausgangsspannung entspricht der verwendeten Eingangsspannung. Da diese vorher nicht bekannt war kann diese hier als 
\begin{equation*}
U_{\text{E}} = \SI{1}{\volt}
\end{equation*}
identifiziert werden. In der Regeln wird die Filterkurve noch normiert angegeben, dazu wird auf der Ordinate $U_{\text{A}}$/$U_{\text{E}}$ aufgetragen. Da die vorliegende Eingangsspannung allerdings $U_{\text{E}} = \SI{1}{\volt}$ beträgt 
unterscheiden sich diese Graphen bis auf die verschwindende Einheit auf der Ordinate nicht. 
Der Vollständigkeit halber ist diese Darstellung ebenfalls noch einmal in Diagramm \ref{fig:kurve2} angegeben.
Desweiteren wird nun für die Bestimmung der realen Güte des Selektivverstärkers eine Halbwertsbreite ausgerechnet. Hierbei gilt die Gleichung .......
und in der Abbildung ist bereits die Linie $U = 1$/$\sqrt{2} U_{\text{A}}$ eingetragen. Die Form der Filterkurve ist eine Gaußsche Glockenkurve und daher wird hier für eine
genauere Bestimmung der Halbwertszeiten eine Fitfunktion in der Form \cite{gauss}
\begin{equation}
G_{\sigma, \beta} =  \frac{\beta}{\sqrt{\pi \sigma^2}} \cdot \text{exp}\left( -\frac{(x- \alpha)^2}{\sigma^2}\right).
\end{equation}

\begin{figure}
    \centering
    \includegraphics[width=\textwidth]{build/kurve2.pdf}
    \caption{Normierte Darstellung der frequenzabhängigen Ausgangsspannung der Filterkurve.} 
    \label{fig:kurve2}
\end{figure}


