\section{Theorie}
Die Grundlagen des Paramagnetismus, sowie die Art der Messung eben solcher Subsatzenen und auftretenden Problemen 
lassen sich in vier Themen unterteilen. Von besonderem Interesse ist das eigentliche Verständnis der Suszeptibilität 
auf welches im folgenden eingegangen wird.

\subsection{Berechnung der Suszeptibilität paramagnetischer Subsatzenen}
Ein magnetisches Feld im Vakuum kann beschrieben werden durch eine Induktionskonstante $\mu_0$ und der eigentlichen magnetischen Feldstärke
$\vec{H}$. Das Feld ist invariant, ändert sich aber wenn Matiere in das Feld reicht um den Summanden $\vec{M}$. 
\begin{equation}
    \vec{B}=\mu_0 \vec{H} \hspace{1cm} \xrightarrow[]{\text{mit Material}} \hspace{1cm} \vec{B}=\mu_0 \vec{H} + \vec{M}
\end{equation}
Dieser zusätzliche Term beschreibt die gemittelten magnetischen Momente der eingeführten Materie und ist wiederum propotional zum externen Feld $\vec{H}$. 
In ihm ist zudem die Suszeptibilität $\chi$ enthalten.
\begin{equation}
    \vec{M} = \mu_0 \chi \vec{H}
\end{equation}
Die durch $\vec{M}$ beschriebene Magnetisierung lässt sich unterscheiden zwischen Dia - und Paramagnetismus. 
Diamagnetismus liegt allen Atomen und deren finalen Foramtionen bei, wobei er dem induktionerregenden, magnetischem Feld $\vec{H}$
entegenwirkt. Folglich muss $\chi$ negativ sein.
Einige seltenen Atome besitzten zudem die Eigenschaften des Paramagnetismus, der durch einen nicht verschwindenen Drehimpuls entsteht und in einer 
Magnetisierung relativ zum externen Feld endet.
Dieses Phönomen ist auf Grund der thermischen Störung der magnetischen Momente im Gegensatz zum Diamagnetismus stark Temperatur abhängig.
Der für diese Art der Magnetisierung zentrale Drehimpuls ist dreiteilig und besteht aus dem
\begin{description}
    \item \textbf{Bahndrehimpuls} der Elektronenhülle,\\
    \item \textbf{Eigendrehimpuls} der Elektronen(auch Spin genant), \\
    \item \textbf{Kerndrehimpuls} (vernachlässigbar).
\end{description}
Es folgt nach dem Superpositionsprinzip ein Gesamtdrehimpuls $\vec{J}$ von 
\begin{equation*}
    \vec{J} = \vec{L} +\vec{S}. \hspace{2cm}\text{mit:}
    \begin{cases}
        \vec{L}\text{= Gesamtbahndrehimpuls}\\
        \vec{S}\text{= Gesamtspin}\\
    \end{cases}
\end{equation*}
Die magnetischen Momente $\vec{\mu_L}$ von Gesamtbahndrehimpuls und $\vec{\mu_S}$ vom Gesamtspin sind nun beide vom Bohrschen Magneton $\mu_B$ abhängig.
Zusätzlich wird $\vec{\mu_S}$ noch durch eine weiteren Faktor $g_s$, dem gyromagnetischem Verhältnis beschrieben.
\begin{align}
    \vec{\mu}_L &= - \frac{\vec{\mu}_B}{\hbar}\vec{L} \\
    \vec{\mu}_S &= -g_S \frac{\vec{\mu}_B}{\hbar}\vec{S} 
\end{align}
Mit Hilfe der Beträge dieser Zusammenhänge lässt sich nun der gesamte magnetische Moment in Verbindung mit trigonometrischen Funktionen angeben.
Diese folgen 
