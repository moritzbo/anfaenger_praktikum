\section{Theorie}
Die Grundlagen des Paramagnetismus, sowie die Art der Messung eben solcher Subsatzenen und auftretenden Problemen 
lassen sich in vier Themen unterteilen. Von besonderem Interesse ist das eigentliche Verständnis der Suszeptibilität 
auf welches im folgenden eingegangen wird.

\subsection{Berechnung der Suszeptibilität paramagnetischer Subsatzenen}
Ein magnetisches Feld im Vakuum kann beschrieben werden durch eine Induktionskonstante $\mu_0$ und der eigentlichen magnetischen Feldstärke
$\vec{H}$. Das Feld ist invariant, ändert sich aber wenn Matiere in das Feld reicht um den Summanden $\vec{M}$. 
\begin{equation}
    \vec{B}=\mu_0 \vec{H} \hspace{1cm} \xrightarrow[]{\text{mit Material}} \hspace{1cm} \mu_0 \vec{H} + \vec{M}
\end{equation}
Dieser beschreibt die gemittelten magnetischen Momente der eingeführten Materie und ist wiederrum propotional zum externen Feld $\vec{H}$. 


