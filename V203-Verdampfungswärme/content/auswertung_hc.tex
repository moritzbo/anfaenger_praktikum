Um festzustellen wie viel Arbeit aufgebracht werden muss, um die Anziehungskärfte der Restlichen Mülekühle zu überwinden,
nährt man mit der allgemeinem Gasgleichung an Verdampfungswärme $L$ an.
Dazu wird, der Dimension wegen, die Gleichung \eqref{eqn:gas} auf  $[\SI{Joule}/\SI{mol}]$ angepasst und entsprechend gekürzt.
\begin{equation}
    \frac{}\SI{bar}\SI{V}}{\SI{mol}}=\SI{R}\SI{Kelvin}
\end{equation}
Mit der gegebenen Temperatur erschließt sich ein Wert von $3101\SI{Joule/mol}$ für $L_a$. Die Differenz folgt.
\begin{align*}
    L_i&=L-L_a \hspace{1cm}  | L = \SI{39.168(0363)}{\kilo\joule\per\mol\per\kelvin}
    L_i= \SI{36.067(0363)}{\kilo\joule\per\mol\per\kelvin}
\end{align*}
Um ein Ergbenis pro Molekül zu bekommen gilt es nun $L_i$ durch den für ein $\SI{mol}$ definierten Wert zu teilen.
In Elektronenvolt umgeformt beträgt die Verdampfungswärme.
\begin{equation}    
    L_i= \SI{0.406(0003)}{eV}