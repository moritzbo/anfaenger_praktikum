\section{Anhang}

\begin{table}
    \centering
    \caption{Messreihe zur Wheatstonesches Brücke mit Speisefrequenz $f = \SI{80}{\hertz}$ für die Bestimmung von $R_{10}$.}
    \label{tab:weed}
    \begin{tabular}{c c c}
        \toprule
       ~ & Messung 1 & Messung 2 \\
        \midrule
      $R_{2}$[$\si{\ohm}$] & 332 &    664  \\
      $R_{3}$[$\si{\ohm}$] & 416 &    260\\
      $R_{4}$[$\si{\ohm}$] & 584 &    740\\
        \bottomrule
    \end{tabular}
    \end{table}

\begin{table}
    \centering
    \caption{Messreihe zur Kapazitätsmessbrücke mit Speisefrequenz $f = \SI{80}{\hertz}$ für die Bestimmung von $C_{9}$ und $R_{9}$.}
    \label{tab:nocap}
    \begin{tabular}{c c c c}
        \toprule
        Kapazität $C_{2}$[$\si{\nano\farad}$] & Widerstand $R_{2}$[$\si{\ohm}$] & Widerstand $R_{3}$[$\si{\ohm}$] & Widerstand $R_{4}$[$\si{\ohm}$]\\
        \midrule
        399 & 493 & 477 & 523  \\
        \bottomrule
    \end{tabular}
    \end{table}

\begin{table}
    \centering
    \caption{Messreihe zur Induktivitätsmessbrücke mit Speisefrequenz $f = \SI{1000}{\hertz}$ für die Bestimmung von $L_{17}$ und $R_{17}$.}
    \label{tab:spuleyo}
    \begin{tabular}{c c c c}
        \toprule
        Induktivität $L_{2}$[$\si{\milli\henry}$] & Widerstand $R_{2}$[$\si{\ohm}$] & Widerstand $R_{3}$[$\si{\ohm}$] & Widerstand $R_{4}$[$\si{\ohm}$]\\
        \midrule
        27.5 & 49.0 & 581.0 & 419.0  \\
        \bottomrule
    \end{tabular}
    \end{table}

\begin{table}[h!]
    \centering
    \caption{Induktivitätsmessung mittels Maxwell-Brücke mit Speisefrequenz $f = \SI{1000}{\hertz}$ für die Bestimmung von $L_{17}$ und $R_{17}$.}
    \label{tab:spuleyodiezweite}
    \begin{tabular}{c c c c}
        \toprule
        Kapazität $C_{4}$[$\si{\nano\farad}$] & Widerstand $R_{2}$[$\si{\ohm}$] & Widerstand $R_{3}$[$\si{\ohm}$] & Widerstand $R_{4}$[$\si{\ohm}$]\\
        \midrule
        399 & 1000 & 89 & 811  \\
        \bottomrule
    \end{tabular}
    \end{table}
    
        \begin{table}
            \centering
            \caption{Frequenzabhängige Brückenspannung in der Wien-Robinson-Brücke bei konstanter Speisespannung $U_{S} = \SI{1}{\volt}$.}
            \label{tab:frequencyitis}
            \begin{tabular}{c c}
                \toprule
                Frequenz $f$ \/[$\si{\hertz}$] & Brückenspannung $U_{B}$[$\si{\volt}$]\\
                \midrule
                20    &   0.24 \\
                100   &   0.22 \\             
                200   &   0.15 \\                 
                300   &   0.09 \\             
                400   &   0.05 \\            
                450   &   0.03 \\            
                460   &   0.026 \\            
                470   &   0.024 \\           
                480   &   0.02 \\          
                490   &   0.018 \\           
                500   &   0.015 \\           
                510   &   0.012 \\          
                520   &   0.008 \\         
                530   &   0.003 \\         
                540   &   0.001 \\          
                550   &   0.003 \\         
                560   &   0.0032 \\          
                570   &   0.008 \\          
                580   &   0.0115 \\           
                590   &   0.0135 \\           
                600   &   0.016 \\           
                610   &   0.02 \\         
                620   &   0.022 \\           
                630   &   0.024 \\         
                700   &   0.04 \\         
                800   &   0.06 \\          
                900   &   0.08 \\           
                1000  &   0.09 \\              
                1200  &   0.12 \\             
                1500  &   0.15 \\             
                2000  &   0.18 \\             
                3000  &   0.2 \\              
                5000  &   0.22 \\             
                10000 &   0.21 \\              
                20000 &   0.18 \\       
                \bottomrule
            \end{tabular}
            \end{table}

            \begin{table}
                \centering
                \caption{Literaturwerte.}
                \label{tab:lit}
                \begin{tabular}{c c c c c}
                    \toprule
                    $R_{10\text{,lit}}$[$\si{\ohm}$] & $R_{9\text{,lit}}$[$\si{\ohm}$] & $C_{9\text{,lit}}$[$\si{\nano\farad}$] & $L_{17\text{,lit}}$[$\si{\milli\henry}$] &$R_{17\text{,lit}}$[$\si{\ohm}$]\\
                    \midrule
                    239.00 & 464.90 & 433.71 & 41.85 & 93.65 \\
                    \bottomrule
                \end{tabular}
                \end{table}

                \begin{table}
                    \centering
                    \caption{Prozentuale Abweichungen der verlustbehafteten Induktivitäten.}
                    \label{tab:hello}
                    \begin{tabular}{c c c }
                        \toprule
                        ~ & $\increment R_{17{\text{,}}\si{\percent}}$ & $\increment L_{17{\text{,}}\si{\percent}}$ \\
                        \midrule
                        Abweichung zwischen Induktivitätsbrücke und Literatur & $\SI{27.4(22)}{}$ & $\SI{8.9(5)}{}$ \\
                        Abweichung zwischen Maxwellbrücke und Literatur & $\SI{15.0(40)}{}$ & $\SI{15.1(26)}{}$ \\
                        Abweichungen untereinander & $\SI{38.1(33)}{}$ & $\SI{6.8(29)}{}$ \\
                        \bottomrule
                    \end{tabular}
                    \end{table}