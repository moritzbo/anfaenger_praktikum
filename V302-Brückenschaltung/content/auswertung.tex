\section{Auswertung}
Im Folgenden werden die Berechnungen der einzelnen Größen durchgeführt, wobei die jeweils bereits genannten Fehler aus der Versuchdurchführung zu einer
Gaußschen Fehlerfortpflanzung führen. 

\subsection{Wheatstonesche Brücke}
In dem Abschnitt \ref{sec:weedyo} sind die relevanten Fehler der gemessenen Größen aus Tabelle \ref{tab:weed} angegeben. Diese fehlerbehafteten Größen lassen sich nun in die Berechnungsgleichung ...
für den ohmschen Widerstand $R$ einsetzen.
Da hier zwei Messungen mit verschiedenen Referenzwiderständen $R_{2}$ durchgeführt wurden, ergeben sich folgende zwei Resultate.
\begin{align}
R_{10\text{,}1} &= \SI{236.5(13)}{\ohm} \\
R_{10\text{,}2} &= \SI{233.3(13)}{\ohm} 
\end{align}
Daraus lässt sich der folgende Mittelwert bestimmen, wobei sich der Fehler mittels Gaußscher Fehlerfortpflanzung ergibt.
\begin{equation}
R_{10} = \SI{234.9(9)}{\ohm}
\end{equation}
Die Fehlerfortpflanzung sieht folgendermaßen aus.
\begin{equation}
\increment R_{10} = \frac{1}{2} \sqrt{(\increment R_{10\text{,}1})^{2} + (\increment R_{10\text{,}2})^{2} }
\end{equation}



\subsection{Kapazitätsmessbrücke}
Aus der Messreihe in Tabelle \ref{tab:nocap} kann zusammen mit den Fehlern wie in Abschnitt \ref{sec:nocapyo} beschrieben eine Kapazität $C_{9}$ berechnet werden. Da diese allerdings dielektrische Verluste 
aufweist muss ebenfalls der Widerstand $R_{9}$ bestimmt werden. Gemäß den Formeln ... und ..., lassen sich diese bestimmen zu.
\begin{align}
C_{9} &= \SI{437.5(24)}{\nano\farad} \\
R_{9} &= \SI{449.6(137)}{\ohm} 
\end{align}

\subsection{Induktivitätsmessbrücke}
Die Messwerte zur Bestimmung der Induktivität $L_{17}$ anhand einer Induktivitätsmessbrücke sind in Tabelle \ref{tab:spuleyo} angegeben. Die bestehenden Fehlerquellen
folgen wieder wie in Abschnitt \ref{sec:spuleman} beschrieben. Diese weist ebenfalls dielektrische Verluste 
auf und es kann ebenfalls der Widerstand $R_{17}$ bestimmt werden. Anschließend können die Messwerte in die Gleichungen .... und ... eingesetzt werden.
\begin{align}
L_{17} &= \SI{38.1(2)}{\milli\henry} \\
R_{17} &= \SI{67.9(21)}{\ohm} 
\end{align}

\subsection{Maxwell-Brücke}
Analog wie bei der Induktivitätsmessbrücke wird hier die Induktivität $L_{17}$ gemessen. Mit den Formeln ... und ...., sowie den Werten aus der Tabelle ergibt sich.
\begin{align}
    L_{17} &= \SI{35.5(11)}{\milli\henry} \\
    R_{17} &= \SI{109.7(47)}{\ohm} 
\end{align}


\subsection{Wien-Robinson-Brücke}

\begin{figure}
    \centering
    \includegraphics[width=\textwidth]{build/plot1.pdf}
    \caption{Messwerte und Theoriekurve der frequenzabhängigen Brückenspannung.} 
    \label{fig:plot1}
\end{figure}