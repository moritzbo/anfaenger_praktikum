gesucht

\begin{equation}
\frac{Q}{N} = Z = ...
\end{equation}
 
mit den zwei gleichungen und anschlißend ineinander eingestzt ...

\begin{equation}
Z =  \frac{\bar{\si{I}}}{e_0N} \hspace{2cm} \bar{\si{I}}= \frac{\Delta Q(U)}{ \Delta t} N
\end{equation}

kürzt sich das N raus. es bleibt stehen 

\begin{equation}
Z= \frac{\Delta Q(U)}{\Delta t \cdot e_0}
\end{equation}

Wobei Q ja 

\begin{equation}
\int {\bar{I(t)}} dt = Q
\end{equation}

ist, also zweifaches integral von I $dt$ ergibt $\Delta Q$. \\
Also...  wir "kennen" $\bar{I}$ nur leider ohne wiederstand egal...\\
also könenn wir diesen Wert zwei mal integrieren. Jetzt haben wir $\Delta Q(U,R)$ in abhängigkeit vom Wiederstand R und klar, von u auch wobei wir den wert der spannung aus den messwerten/tabellen entnehmen können. lassen wir ihn einfach mal als u stehen.\\
Die integral Grenßen geben uns das $\Delta t$ aus (3). 


\begin{equation}
\Delta Q = \int{\int {\int_{0}^{\tau} \frac{U(t)}{R} (dt)^3}}
\end{equation}

also:

\begin{equation}
Z= \frac{\int{\int {\int \frac{U(t)}{R} (dt)^3}}}{\int t \cdot e_0 dt}
\end{equation}