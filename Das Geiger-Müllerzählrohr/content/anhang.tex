\section{Anhang}

\begin{table}
\centering
\caption{Messdaten der Zwei-Quellen-Methode}
\label{tab:ogemessdaten3}
\begin{tabular}{c c c}
    \toprule
    Zählrate $N_{1}$[$\text{Imp}$/$120\si{\second}$] & Zählrate $N_{1+2}$[$\text{Imp}$/$120\si{\second}$] & Zählrate $N_{2}$[$\text{Imp}$/$120\si{\second}$]\\
    \midrule
    96041 & 158479 & 76518 \\
    \bottomrule
\end{tabular}
\end{table}

\begin{table}
\centering
\caption{Gemessener Zählerstrom des Geiger-Müllerzählrohrs.}
\label{tab:ogemessdaten2}
\begin{tabular}{c c c}
    \toprule
    Spannung $U$[$\text{Imp}$/$60\si{\second}$] & Stromstärke $I$[$\si{\micro\ampere}$] & Fehler $\increment I$[$\si{\micro\ampere}$]\\
    \midrule
    350   & 0.3 & 0.05\\
    400	  & 0.4 & 0.05\\
    450	  & 0.7 & 0.05\\
    500	  & 0.8 & 0.05\\
    550	  & 1.0 & 0.05\\
    600	  & 1.3 & 0.05\\
    650	  & 1.4 & 0.05\\
    700	  & 1.8 & 0.05\\
    \bottomrule
\end{tabular}
\end{table}

\begin{table}
\centering
\caption{Messdaten der Zählrohr-Charakteristik.}
\label{tab:ogemessdaten}
\begin{tabular}{c c}
    \toprule
    Spannung $U$[$\si{\volt}$] & Zählrate $N$[$\text{Imp}$/$60\si{\second}$]\\
    \midrule
    320	& 9672 \\
    330	& 9689  \\
    340	& 9580  \\
    350	& 9837  \\
    360	& 9886  \\
    370	& 10041 \\
    380	& 9996\\
    390	& 9943\\
    400	& 9995\\
    410	& 9980\\
    420	& 9986\\
    430	& 9960\\
    440	& 10219\\
    450	& 10264\\
    460	& 10174\\
    470	& 10035\\
    480	& 10350\\
    490	& 10290\\
    500	& 10151\\
    510	& 10110\\
    520	& 10255\\
    530	& 10151\\
    540	& 10351\\
    550	& 10184\\
    560	& 10137\\
    570	& 10186\\
    580	& 10171\\
    590	& 10171\\
    600	& 10253\\
    610	& 10368\\
    620	& 10365\\
    630	& 10224\\
    640	& 10338\\
    650	& 10493\\
    660	& 10467\\
    670	& 10640\\
    680	& 10939\\
    690	& 11159\\
    700	& 11547\\
    \bottomrule
\end{tabular}
%Geiger-Müller Kennlinie in 60s Intervallen
\end{table}