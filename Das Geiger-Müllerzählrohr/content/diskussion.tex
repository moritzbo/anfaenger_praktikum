\section{Diskussion}
\label{sec:Diskussion}

Bei der Auswertung der Charakteristik des Zählrohrs wurde ein Plateu-Bereich von $\SI{370}{\volt}$ bis $\SI{630}{\volt}$
ausgewählt, welches im idealen Fall nur eine sehr kleine bis keine Steigung der Zählraten in dem Spannungsintervall aufweisen sollte. Das Ergebnis einer linearen Regression ergab eine Steigung von 
knapp ${1}\si{\percent\per{100}\volt}$ welche im Rahmen eines realistischen Plateau-Bereichs liegt. Der Fehler auf die Steigung dieser Ausgleichsgeraden ist 
mit $\increment a_{\si{\percent}} = \pm \SI{0{,}218}{\percent\per{100}\volt}$ zwar relativ hoch, dies liegt daran, dass trotz Alkoholdampfzusatz einige Nachentladungen stattfinden können. 
Bei poissonverteilten Zählraten sind die Ungenauigkeiten allerdings wie erwartet relativ hoch, also scheint die Größenordnung des Fehlers plausibel.
\\
\newline
Die Totzeit wurde durch zwei unterschiedlichen Methoden bestimmt, wobei eine große Diskrepanz auffällt. Bei der Bestimmung über das Oszilloskop
wurde die Totzeit graphisch abgelesen, dies führt offensichtlich schon zu einer erhöhten Unsicherheit. Die Zwei-Quellen-Methode ist wie erwartet genauer, obwohl
auch dort nur eine Annäherung getroffen wird. Der abgelesene Wert der Totzeit am Oszilloskop liegt jedoch im Fehlerintervall der bestimmten Totzeit über die Zwei-Quellen-Methode.
\\
\newline
Im letzten Teil der Auswertung wurde die pro Teilchen freigesetzte Ladung bestimmt und in dem Diagramm \ref{fig:plot2} dargestellt. Hier lässt sich eine linearer Zusammenhang feststellen
und die Anzahl $Z$ liegt in der Größenordnung $10^{10}$ mit dementsprechend hohen Unsicherheiten. 
\\
\newline
Bei der gesamten Durchführung wurde darauf geachtet, dass eine Zählrate von $\SI{100}{{\text{Imp}}\per\second}$ nicht großartig überschritten wird damit es nicht zur Dauerentladung kommt.
Die Dauerentladung und hohe Impulse sorgen am Anodendraht für einen hohen Strom was ebenfalls zu einer Schädigung der Apperatur führen kann.
