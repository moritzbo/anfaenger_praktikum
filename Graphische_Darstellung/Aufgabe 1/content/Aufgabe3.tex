\section{Aufgabe 3}
\subsection{Berechnungen zum Absorptionsgesetz}
In dieser Aufgabe wird aus der Anzahl an gezählten Gamma-Quanten $N$ und deren Fehlerbehaftung ein $d$-$N$ Diagramm erstellt.
Dabei wird eine lin-lin und eine lin-log Darstellung verwendet und anschließend der Absorptionskoeffizient $\mu$ bestimmt.
\subsection{Lösung}
Das Absorptionsgesetz lautet

\begin{equation}
N = N_{0} \cdot e^{- \mu d}
\label{eqn:absorptionsges}
\end{equation}
\begin{flushleft}
Hierbei ist $d$ die Dicke der verwendeten Bleiplatte, $\mu$ der Absorptionskoeffizient und $N_{0}$ der Achsenabschnitt, bzw maximale Zählwert.
Der Fehler einer Zählmessung kann mit der Poisson-Statistik beschrieben werden.
\end{flushleft}

\begin{equation}
\increment N = \sqrt{N}
\label{eqn:Poissonvert}
\end{equation}

\begin{flushleft}
Dadurch ergeben sich die folgenden Fehler
\end{flushleft}
\begin{table}
\centering
\caption{Berechnung der Messunsicherheiten der Zählraten.}
\label{tab:messwerte3}
\begin{tabular}{c c c}
    Bleidicke $d [\symup{cm}]$ & Anzahl der Gamma-Quanten $N$ & Fehler $\increment N$\\
    \midrule
    0.1 & 7565 & 86.98\\ 
    0.2 & 6907 & 83.11\\
    0.3 & 6214 & 78.83\\
    0.4 & 5531 & 74.37\\
    0.5 & 4942 & 70.30\\
    1.0 & 2652 & 51.50\\
    1.2 & 2166 & 46.54\\
    1.5 & 1466 & 38.29\\
    2.0 & 970  & 31.14\\
    3.0 & 333  & 18.25\\
    4.0 & 127  & 11.27\\
    5.0 & 48   & 6.93 \\
    \bottomrule
\end{tabular}
\end{table}
\begin{flushleft}
Nun lassen sich die Werte in ein $d$-$N$ Diagramm einzeichnen und ein Fehlerbalken mit den $\increment N$-Werten plotten. 
In Diagramm \ref{fig:plot4} sind die beiden Achsen linear dargestellt, also entsteht der zu erwartende exponentielle Verlauf.
In Diagramm \ref{fig:plot5} ist nun die Ordinate logarithmiert es folgt ein linearer Verlauf.
Um nun den Absorptionskoeffizienten $\mu$ herauszufinden können mit einem curvefit die gegebenen Werte an das Absorptionsgesetz \eqref{eqn:absorptionsges} angepasst werden.
Für den Absorptionskoeffizienten $\mu$ und $N_{0}$ folgt
\end{flushleft}
\begin{align}
\mu &= 1.139 \pm 0.019 \\
N_{0} &= 8625.263 \pm 72.810
\end{align}
\begin{flushleft}
Das Absorptionsgesetz mit eingesetzten Werten kann nun geplotten werden und es entsteht das Diagramm \ref{fig:plot6}.
\end{flushleft}
\newpage
\begin{figure}[t]
  \centering
  \includegraphics[width=\textwidth]{build/plot4.pdf}
  \caption{Zählwerte und Fehlerbalken der Gamma-Quanten bei unterschiedlicher Dicke $d$ in linearer Darstellung.}
  \label{fig:plot4}
\end{figure}
\begin{figure}[t]
  \centering
  \includegraphics[width=\textwidth]{build/plot5.pdf}
  \caption{Zählwerte und Fehlerbalken der Gamma-Quanten bei unterschiedlicher Dicke $d$ in linear-logarithmischer Darstellung.} 
  \label{fig:plot5}
\end{figure}
\clearpage
\begin{figure}[t]
  \centering
  \includegraphics[width=\textwidth]{build/plot6.pdf}
  \caption{Exponentieller Fit an die gegebenen Messwerte zur Bestimmung des Absorptionskoeffizienten.} 
  \label{fig:plot6}
\end{figure}
\begin{flushleft}
~
\end{flushleft}