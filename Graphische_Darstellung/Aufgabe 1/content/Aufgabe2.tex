\section{Aufgabe 2}
\subsection{Linsengleichung}
Die allgemeine Linsengleichung lautet
\begin{equation}
\frac{1}{f} = \frac{1}{g} + \frac{1}{b}
\label{eqn:linsengleichung}
\end{equation}
\subsection{a)}
Die sechs verschiedenen Kombinationen der Brennweite können über die Linsengleichung \eqref{eqn:linsengleichung} berechnet werden.

\begin{table}
\centering
\label{tab:lösunga}
\begin{tabular}{c c c}
    Gegenstandsweite $g [\symup{mm}]$ & Bildweite $b [\symup{mm}]$ & Brennweite $f [\symup{mm}]$\\
    \midrule
    60 &285 &49.57\\
    80 &142 &51.17\\
    100& 117&53.92\\
    110& 85 &47.95\\
    120& 86 &50.10\\
    125& 82 &49.52\\
    \bottomrule
\end{tabular}
\end{table}
\begin{flushleft}
Der Mittelwert lässt sich wie folgt berechnen
\end{flushleft}
\begin{equation}
\bar{f} = \frac{1}{n} \sum_{i=1}^n f_i = \SI{50.2}{\milli\meter}
\label{eqn:mittelwert_f}
\end{equation}
Die Standardabweichung auf zwei Kommastellen gerundet ist
\begin{equation}
\sigma_f = \sqrt{\frac{1}{n-1} \sum_{i=1}^n (f_i - \bar{f})^2} = \SI{2.04}{\milli\meter}
\label{eqn:sigma_f}
\end{equation}
Der Fehler auf dem Mittelwert beträgt
\begin{equation}
\increment \bar{f} = \sqrt{\frac{1}{n(n-1)} \sum_{i=1}^n (f_i - \bar{f})^2} = \SI{0.83}{\milli\meter}
\label{eqn:seoftm_f}
\end{equation}
\subsection{b)}
Definiert man $\symup{G} = \frac{1}{g}$ und $\symup{B} = \frac{1}{B}$, dann kann man ein G-B Diagramm mit den gegebenen
Daten skizzieren. Die Lineare Regression lässt sich wieder durch die Methode der kleinsten Abstandsquadrate ermittlen analog zur Aufgabe 1. Diesmal mache ich dies direkt
über die Bibliothek \enquote{matplotlib} in Python und man erhält.

\begin{align}
a &= -1.010 \pm 0.118 \\
b &= 0.020 \pm 0.001
\end{align}
\newpage
\begin{figure}[t]
  \centering
  \includegraphics[width=\textwidth]{build/plot3.pdf}
  \caption{Messdaten und Lineare Regression zur Bestimmung der Brennweite}
  \label{fig:plot3}
\end{figure}
\begin{flushleft}
Also erhält man eine Ausgleichsgerade \eqref{eqn:ausglg2} die man in den Graphen einzeichnen kann
\end{flushleft}
\begin{equation}
\symup{B} = -1.01 \symup{G} + 0.02
\label{eqn:ausglg2}
\end{equation}
\begin{flushleft}
Die Brennweite $f$ kann man nun durch den Achsenabschnitt errechnen, denn für $G=0$ gilt
\end{flushleft}
\begin{equation}
\frac{1}{f} = \frac{1}{b} = \symup{B} \to f = \frac{1}{\symup{B}} = \SI{50}{\milli\meter}
\label{eqn:f_linregress}
\end{equation}
\subsection{c)}
Die beiden Ergebnisse für die Brennweiten sind also nicht identisch.
Dies liegt aber vor allem daran, dass die Gerade welche durch die lineare Regression erstellt wurde, den berechneten Fehler auf den Achsenabschnitt $b$ nicht mit anzeigt.
Trotzdem liegt der Wert \eqref{eqn:f_linregress} noch im Bereich der Standardabweichung \eqref{eqn:sigma_f} für $f$ welche zuvor bestimmt wurde.
