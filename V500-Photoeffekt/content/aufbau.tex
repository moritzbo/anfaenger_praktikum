\section{Aufbau}
Grundlegende Bauelemente sind zwei Platten, eine Photokathode und eine Auffängerelektrone.
Die im Verlauf des Versuches mit Photonen bestrahlte Photokathode ist zudem mit der anderen Platte über ein Strommessgerät verbunden.
Dadurch lässt sich eine nun entstandene Spannug messung und gibt zudem Aufschluss über die Anzahl der ausgelösten Elektronen 
und die Energie eben dieser Photoelektronen.

\begin{figure}
    \centering
    \includegraphics[width=0.5\textwidth]{bilder/müde.png}
    \caption{Schematischer Aufbau einer Messvorrichtung zur Beobachtung des Photoeffekts. \cite{skript}.} 
    \label{fig:müde}
\end{figure}
\subsection{Die Photozelle}
\begin{figure}
    \begin{minipage}{0.5\textwidth}
Der oben genannte Aufbau wird realisiert in einer Photozelle. In einem Vakuum befindet sich eine Photokathode die durch 
eine Metallschicht den Ansprüchen gerecht gemacht wurde. Um diese Schicht herum verläuft ein dünner Draht der so die Funktion der Anode 
erfüllt. Der Draht und die Photokathode liegen parallel zueinder, das heißt an jeder Stelle teilen sich Kathode und Anode einen gemeinsamen Normalenvektor.
Die Messreihen verlangt, dass eine externes Feld in diesem Raum angelegt wird um den Transport der Photoelektronen und somit 
messbare Ergbenise, zu erleichtern.  Dafür ist der Draht und die Photokathode mit einem einstellbaren Generator verbunden was es also möglich macht, 
das gewünschte Feld im Glaskörper zu erzeugen. Zusätzlich misst ein weiteres Messgerät den fließenden Strom welcher auf Grund 
der Kontaktlosigkeit nur durch eben ausgelösten Elektronen entstehen kann. 
     \end{minipage}
\hfill
    \begin{minipage}{0.5\textwidth}
        \centering
        \includegraphics[width=0.9\textwidth]{bilder/sehrmüde.png}
        \captionsetup{justification=centering}
    \captionof{figure}{Schematischer Aufbau \\ einer Messvorrichtung zur Beobachtung \\ des Photoeffekts. \cite{skript}.} 
    \label{fig:sehrmüde}
    \end{minipage}
\end{figure}

\subsection{Monochromatisches Licht}
Es bietet sich für die Auswertung an, die Photozelle nur mit Licht einer bekannten Wellenlänge zu bestrahlen.
Dazu müssen die einzelnen Spekrallinien, erzeugt von einer Spektrallampe, räumlich von einander getrennt werden. 
Die Umsetzung dieser Trennung erfolgt durch Linsen und einem Spalt, der das Licht auf ein Gradsichtprisma lenken.
Das Prisma bricht anschließend das Licht unterschiedlicher Wellenlängen, gemäß dem Snellius'schen Brechungsgesetz unterschiedlich 
stark wodurch nacher die gewünschte Farbe mit räumlichen Abstand von der Photozelle eingefangen werden kann.

\section{Durchführung}