\section{Diskussion}

Zunächst lässt sich anmerken, dass das Licht mit geringeren Intensitäten, beispielsweise die zweite violette Spektrallinie der Quecksilberlampe mit Wellenlänge $\lambda = \SI{408}{\nano\meter}$, deutlich schwieriger erkennbar und somit nicht 
so gut auf den Eintrittsspalt lenkbar war. Außerdem kam es durch den Versuchaufbau zu einer geringfügigen Kippung der Photozelle bei höheren Wellenlänge aufgrund der Montierung. Dadurch kann es zu geringeren Intensitäten an der Kathode kommen.
\\
\newline
Die größte Fehlerquelle stellt das Pikoamperemeter dar, denn dies ist sehr sensibel und zeigt zum Teil sehr stark schwankende Werte an. Außerdem kann es durch das Umstellen der Messskala teils Abweichungen des Stroms bei gleicher Spannung geben. Es wurde daher
möglichst lange nur eine Skala betrachtet, was wiederum die Ablesegenauigkeit verringert. Zudem sollte das Koaxialkabel möglichst gut geerdet und nicht bewegt werden, da dies zu Abweichungen der Strommessung führt. Die Spannung lies sich über den 
Spannungsgenerator in einigermaßen konstanten Abschnitten gut regeln.
\\
\newline
Aus den aufgestellten Strom-Spannungs-Kurven \ref{fig:gelb} bis \ref{fig:GIGAviolet} liesen sich die Grenzspannung gut durch lineare Ausgleichsgeraden ermitteln dabei liegen die gelbe, grüne und erste violette Spektrallinie in dem zu erwartenden 
Bereich. Bei steigender Frequenz nehmen die Grenzspannungen zu und dies wird durch diese Messwerte bestätigt. Als einzige Ausnahme und deutlichen Ausreißer ist die zweite violette Spektrallinie anzusehen. Die Grenzspannung besitzt hier einen viel zu kleinen
Wert und die Betrachtung dieser Linie muss somit aus den folgenden Betrachtungen entzogen werden. Die Messreihe stellt sich als fehlerhaft heraus, dies könnte unteranderem an dem empfindlichen Amperemeter oder der Fokussierung des Lichtes in die Photozelle liegen.
\\
\newline
Das ermittelte Verhältnisse der Naturkonstanten $h$/$e_{0}$ hat eine prozentuale Abweichung von

\begin{equation}
\increment \left( \frac{h}{e_0} \right) = \SI{2.96(3268)}{\percent} 
\end{equation}

vom Literaturwert \cite{lit}. Dieses Ergebnis ist also bereits durch drei Messreihen ziemlich genau, allerdings mit einem dementsprechend großen Fehlerintervall.
Die Austrittsarbeit der verwendeten Kathode kann aufgrund des unbekannten Materials nicht sehr gut verglichen werden, allerdings liegt es im Elektronenvoltbereich und dadurch bei typischen Bindungsenergien von Elektronen in Atomen.
Das Ergebnis scheint somit auch plausibel zu sein.
\\
\newline
Die zu untersuchenden Phänomene liesen sich gut durch die genauere Betrachtung der gelben Spektrallinie erklären.