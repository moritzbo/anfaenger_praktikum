\subsection{Diskussion}

Bei der Untersuchung der Bragg-Bedingung wurde eine Winkelabweichung von $\increment \theta = 0.1\textdegree$ festgestellt, dieser Wert
wurde für die anschließenden Berechnungen mit einbezogen und fällt relativ klein aus. Eine kritische Winkelabweichung wäre über $1\textdegree$ und somit liegt
dieser hier gut im Rahmen.
\\
Die Aufzeichnung der Messwerte des Emissionsspektrums zeigen klar zu erkennende charakteristische Linien \ref{fig:plot2}. Die daraus bestimmten Halbwertsbreiten
passen ebenfalls gut in das Diagramm und besitzen einen Fehler basierend auf der Winkelabweichung von vorher. Dieser fällt bei der Halbwertsbreite allerdings klein aus.
\\
Die Energiewerte aus Tabelle \ref{tab:lol}


Die Bestimmung des Maximums bei feststehendem Kristall führt zu einer Abweichung
von $\incrementφ=\SI{0.4}{\degree}$.
Dieser Fehler wird bei der Beurteilung
der weiteren Ergebnisse nicht berücksichtigt werden, da er bei jedem Messwert auftritt.