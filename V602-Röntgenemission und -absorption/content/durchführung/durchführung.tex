\section{Versuchsdurchführung}
Zu Beginn gilt es mehrere Fehlerquellen zu überprüfen. \\
Wichtig ist vor allem die Anwesenheit der 1mm Blende und des LiF - Kristalls in der Halterung.
Außerdem muss gewährleistet werden, dass die Schlitzblende senkrecht auf der Drehrichtung steht. Die zwei bilden also einen rechten Winkel.
Somit ist auch die waagerechte Ausrichtung zu dem Geiger-Müller-Zählrohr garantiert. %ref zur theorie warum das soll
\\
\newline
Um mögliche Ungenauigkeiten oder falsche Ergebnisse früh auszuschließen wir am Anfang die Bragg Bedingung %ref 
validiert. Der Kristallwinkel wir dazu auf einen festen Wert $\theta = 14 \textdegree $  gestellt.
Anschließend misst das Geiger-Müller-Zählrohr die Intensität der Strahlung und zwar in einem Winkelbereich von 
$\Delta \alpha_{GM} = 4 \textdegree$ beginnend bei $\alpha_{GM} = 26 \textdegree$. Das Intervall der Messung soll sich 
auf $\Delta \alpha = 0.1 \textdegree$ beschränken, es werden also insgesamt 40 Messungen mit einer Integrationszeit von $\Delta t = 4 \si{s}$ durchgeführt. 
Die Daten werden ausgewertet und als Kurve dargestellt. Das Maximum, mit einem Fehler von $\Delta 1\textdegree$ sollte im Bereich des gegebenen Sollwinkel liegen.
Die Röntgenröhre wurde in allen Messungen bei einer Spannung $U_{B} = \SI{35}{\kilo\volt}$ und einem Emissionsstrom von $I = \SI{1}{\milli\ampere}$ betrieben.
\\
\newline
\subsection{Emessionsspektrum}
Mit den nun ausgeschlossenen Fehlerquellen wird das Emessionsspektrum einer Cu-Röntgenröhre gemessen.
Durch die Programmwahl zum Koppelmodus am Rechner lässt sich das Spektrum bestimmen.
%idk ob das wichtig ist
Die Parameter lauten wie folgt:
\begin{itemize}
    \item{Winkelbereich   $4 \textdegree \leq \theta \leq 26 \textdegree $}
    \item{Winkelzuwachs $\Delta \theta = 0.2 \textdegree$}
    \item{Integrationszeit $\Delta t = 5 \si{s}$}
\end{itemize}
Als nächstes folgt die Bestimmung des Detailsprektrums. Von Belang sind hier die $K_{\alpha} \text{-und} K_{\beta}$ Linien. 
Die Messung findet hier statt mit gegeben Intervallen
\begin{itemize}
    \item{Winkelzuwachs $\Delta \theta = 0.1 \textdegree$}
    \item{Integrationszeit $\Delta t = 5 \si{s}$}
\end{itemize}

\subsection{Absorptionsspektrum}
Beim Absorptionsspektrum gilt es mit einem Zinkabsorber zu arbeiten. Dieser wird vor das Geiger-Müller-Zählrohr gesetzt 
und entsprechend gemessen. Auch hier findet eine Drehung des Stoffes statt, dieses mal mit den Werten
\begin{itemize}
    \item{Winkelzuwachs $\Delta \theta = 0.1 \textdegree$}
    \item{Integrationszeit $\Delta t = 20 \si{s}$}
\end{itemize}
Der Messbereich bleibt dem Durchführer/der Durchführerin selbst überlassen.\\
Mit vier anderen Absorbern wird dieser Schritt ausgeführt. Die Bedingung für die Absorber ist, dass die Kernladungszahl im Bereich 
von $30 \leq \si{Z} \leq 50 $ liegt.

