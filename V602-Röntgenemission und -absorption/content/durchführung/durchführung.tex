\section{Versuchsdurchführung}
Zu beginn gilt es mehrere Fehlerquellen zu überprüfen. \\
Wichtig ist vorallem die Anwesenheit der 1mm Blende und des LiF - Kristall in der Halterung.
Außerdem muss gewährleistet werden, dass die Schlitzblende senkrecht auf der Drehrichtung steht. Die zwei bilden also einen rechten Winkel.
Somit ist auch die waagerechte Ausrichtung zu dem Geiger-Müller-Zählrohr garantiert. %ref zur theorie warum das soll
\\
\newline
Um mögliche Ungenauigkeiten oder falsche Ergbnise früh auszuschließen wir am Anfang die Bragg Bedingung %ref 
validiert. Der Kristallwinkel wir dazu auf einen festen Wert $\theta = 14 \textdegree $  gestellt.
Anschließend misst das Geiger-Müller-Zählrohr die Intensiät der Strahlung und zwar in einem Winkelbereich von 
$\Delta \alpha_{GM} = 4 \textdegree$ beginnend bei $\alpha_{GM} = 26 \textdegree$. Das Intervall der Messung soll sich 
auf $\Delta \alpha = 0.1 \textdegree$ beschränken, es werden also insgesamt 40 Messungen mit einer Integrationszeit von $\Delta t = 4 \si{s} $stattfinden. 
