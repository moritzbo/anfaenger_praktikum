\section{Theorie}
Der Begriff \enquote{Ultraschall} beschreibt den für Menschen nicht wahrnehmbaren Schall in einem Frequenzbereich von $\{\SI{20}{\kilo\hertz},\SI{1}{\giga\hertz}\}$.
\begin{align*}
    \text{Infraschall}  \textrightarrow \text{Hörbarer Bereich}  \textrightarrow \text{Ultraschall}  \textrightarrow \text{Hyperschall} 
\end{align*} 
Schall weist einen longitudinalen Wellencharakter auf, der duch Brechung oder Reflexion nachgewisen werden kann. Entsprechend lässt dieser sich also darstellen als
\begin{equation*}
    p(x,t) =  p_0 + \nu_0 Z \text{cos} (\omega t - kx).
\end{equation*}
Die Ausbreitung der Welle hängt also von der sogenannten akustischen Impedanz ab $Z=c\cdot\rho$ ab, welche selbst ein Produkt der
Schallgeschwindigkeit $c$ und der Dichte $\rho$ des entsprechenden Mediums ist. Die Schallgeschwindkeit wiederum variert bei verschieden Agreagtzuständen des verwendet Mediums
und wird bei Flüssigkeiten mit einer Kompressibilität $\kappa$ zu
\begin{equation*}
    c_{FL}=\sqrt{\frac{1}{\kappa \cdot \rho}}.
\end{equation*}
Fast Analog folgt der Zusammenhang bei Festkörpern der maßgeblich durch die nun zusätzliche Schubspannung dominiert und zudem stark Richungsabhängig ist.
Diese Spannung lässt zusätzlich transversala Wellen zu und anstelle der reziproken Kompressibilität findet sich hier das Elastitsitätsmoduk $E$.
\begin{equation*}
    c_{FE}=\sqrt{\frac{E}{\rho}}
\end{equation*}
Die Bewegung von Schall verlangt ein Medium, wobei die Intensität aber exponentiell abfällt.
Der maßgebende Faktor dafür ist der Absorbationskoeffizient $\alpha$ und die ursprüglich einfallende Schallamplitude $I_0$.
\begin{equation}
    \label{eq:dummsumm}
    I(x)=I_0 \cdot e^{-\alpha x}
\end{equation}
Verschieden Messungen von Absorbationskoeffizient %maybe eine tabllle oder os refefn
zeigen, dass Luft eine äußerst hohe Absorbation aufweist was sich nicht als nürtzlich in manchen Anwendungen erweist.
Des weitern wird einfallender Schall bei Grenzflächen mit einem Reflexionkoeffizienten $R$ reflektiert was anschließende Messungen erschwert.
Da die Intensität zu jedem Zeitpunkt erhalten bleiben muss findet sich der Transmissionskoeffizient der den transmittierten Anteil beschreibt.
Beide Koeffizienten müssen also in Summe $T+R=1$ geben um Verlust auszuschließen.
Die Reflexion wird durch die verschieden akustischen Impedanzen der verschieden Medien gebildet und lautet
\begin{equation*}
    R= \left( \frac{Z_1 - Z_2}{Z_1 + Z_2}\right)^2.
\end{equation*}
Um Schall mit solch hohen Frequenzen zu erzeugen, wird das Resultat des \enquote{piezo-elektrische Effekt} genutzt.
Dieser Effekt beschreibt das Wirken eines Piezokristalls in einem extern angelegtem elektrischen Feld. Die passenden Ausrichtung der polaren Achse des Kristalls zur Richtung des Felds 
lässt beispielsweise Quarz schwingen und somit Ultraschallwellen mit hoher Intensität abstrahlen wenn Resonanz zu beobachten ist.  
Umkehren lässt sich dieser Effekt, indem der Kristall Schallwellen ausgesetzt wird und so ein eigenes, später messbares, elektrisches Feld erzeugt. 
\\
\newline
In der Anwendung bieten sich besonders zwei Verfahren an um beispielsweise Informationen über Werkstoffe oder den Kröper zu bekommen, ohne dabei 
die Medien zu zerstören oder deformieren.
\\
\newline
\begin{figure}
\begin{minipage}{0.5\textwidth}
\begin{description}
    \item [Durschallungs-Verfahren:] Dieser zweitelige Aufbau ist aufgeteilt in eine Sonde und einen Empfänger, welcher unter das Medium gelegt werden muss.
    Somit werden Fehlstellen durch abgeschwächte Intensitäten bemerkbar. \\
    \item [Impuls-Echo-Verfahren:] In diesem Fall fungiert der Sender zeitgleich als Empfänger. Es ist also nicht mehr nötig, unter das Medium reichen zu müssen was die Anwendung 
    erliechtert und handlicher macht. Außerdem gibt die Laufzeit $\Delta t$ des Schall, mit bekannter Schallgeschwindigkeit im zu durchstrahlenden Medium, 
    Aufschluss über die Entfernung der Fehlstelle
    \begin{equation}
        \label{eqn:WegvonSchallDurchEinMediumMitSchallgeschwindkeitCUndLaufzeitDeltaTWoebiDasBestimmtAuchAlsLichgeschwindgeitGesehenwerdenkannwennmankeineahnunghatundnichtdenkontextcheckt.}
        s = \frac{1}{2}c \Delta t
    \end{equation}
\end{description}
\end{minipage}
\hfill
\begin{minipage}{0.4\textwidth}
    \centering
    \includegraphics[width=0.7\textwidth]{Bilder/stuff.png}
    \captionsetup{justification=centering}
    \captionof{figure}{Schematische Darstellung der \\ zwei oben benannten Methoden. \cite{skript}}
    \label{fig:fig:fig:fig}
\end{minipage}
\end{figure}