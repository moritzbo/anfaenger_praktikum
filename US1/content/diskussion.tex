\section{Diskussion}

Zunächst lässt sich anmerken, dass die Ultraschallmessungen über eine Echoskop und einem Ausleseprogramm sehr gut funktionieren, dennoch spielt die Sensibilität 
der Ultraschallsonde, vor allem bei den Spannungen eine große Rolle. Desweiteren ist zu erkennen, dass die Intensität der gemessenen Spannung mit dem Abstand abnimmt. Dieser Effekt ist zwar bemerkbar aber gering und somit lässt 
sich ein kleiner Absorptionskoeffizient in Acryl erwarten.
\\
\newline
Die ermittelte Schallgeschwindigkeit in Acryl aus Gleichung \ref{eqn:1} besitzt eine prozentuale Abweichung von
\begin{equation}
\increment c_{\text{acryl}} = \SI{3.11(199)}{\percent},
\end{equation}
zum Literaturwert $c_{\text{lit}} = \SI{2730}{\meter\per\second}$ \cite{online}. Es lässt sich also sagen, dass über die 22 Messungen der Laufzeit durch
den Acrylquader, ein relativ genaues Ergebnis bestimmt werden konnte. Auf Grund dieser hohen Genauigkeit lassen sich nun auch relativ geringe Abweichungen 
in den Abständen zwischen Oberfläche und Fehlstelle erwarten. Dies ist an den in der Tabelle \ref{tab:wertewerte} angegebenen Werten zu erkennen. Dabei gibt es einige wenige Werte welche eine höhere Abweichung
zeigen, beispielsweise die Werte mit kleineren gemessenen Abständen $s$. Bei diesen hat ein relativer statistischer Fehler prozentual eine größere Auswirkung. 
\\
\newline
Bei der Untersuchung des Augenmodells stellen sich realistische Ergebnisse heraus. Die Größenordnung des Abstands zwischen Sonde, also Hornhaut, und den einzelnen Bestandteilen liegt in einer nach dem Augenmaß realistischen Größenordnung von einigen Zentimetern.
