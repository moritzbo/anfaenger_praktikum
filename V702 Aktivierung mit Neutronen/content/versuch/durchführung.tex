Die erzeugten thermischen Neutronen %ref dazupacken auzs therorie
werden auf den Werkstoff gelenkt. Die anschließende auftrentende Strahlung wir von  dem Geiger-Müller-Zählrohr detektiert und mit den Zählern erfasst.
Durch die zwei Anzeigen der Zähller kann man nun die Ausgewertete Anazhl an Zerfällen ablesen, wärend der Zeitgeber automatisch nach $\Delta t$ auf die Zweite Anzeige wechselt.
Dieser Vorgamg wiederholt sich periodisch. Es lassen sich also ohne Pause genaue Messungen erzielen.

\begin{figure}
  \centering
  \includegraphics[width=0.8\textwidth]{bilder/Screenshot 2021-01-22 103203.png}
  \caption{Aufbau  \cite{hinweis}.}
  \label{fig:aufbau}
\end{figure}

Der Zeitgeber hat eine Genauigkeit von $\si{e-5}$ auf das Gegebene Messintervall $\Delta t$.
Zusätlich wird vor Beginn der Durchführung mit radiaktiven Substanzen der \textbf{Nulleffekt} $N_U$ %\ref zur theorie 
gemessen um ungewollte Strahlung nacher vom Ergbenis abziehen zu können.  Der Messung liegt ein statistischer Fehler bei, weswegen sich möglichst lange Messungen des Nulleffekts empfehlen.


