\section{Diskussion}

Eine Fehlerquelle bei der Auswertung der Halbwertszeiten ist die Untergrundrate. Diese wurde über ein langes Zeitintervall hinweg gemessen um die
Schwankung möglichst gering zu halten. Durch eine bessere Isolation des Geiger-Müllerzählrohrs könnte diese geringer gehalten werden.
Die Ungenauigkeit hierbei spielt aber eine Rolle für die anschließenden Ergebnisse. Des Weiteren entstehen Ungenauigkeit bei
den gemessenen Zählraten aufgrund der Totzeit des Geiger-Müllerzählrohrs. 
\\
\newline
Die erste ermittelte Halbwertszeit für Vanadium über eine Ausgleichsgerade im ganzen gemessenen Zeitintervall lässt sich gut mit den Literaturwerten in Tabelle \ref{tab:params123} vergleichen. Die Abweichung des in diesem Versuch bestimmten Werts 
liegt bei $\SI{2.75}{\percent}$. Dieser konnte also ziemlich genau bestimmt werden, dadurch kann ebenfalls argumentiert werden, dass der Einfluss der Fehlerquellen
durch Geiger-Müllerzählrohr und Untergrundrate eher gering sind. Die zweite Methode der Bestimmung der Halbwertszeit anhand einer Ausgleichsgeraden
bis zur doppelten Halbwertszeit hat eine Abweichung von $\SI{14.49(901)}{\percent}$ zum zuvor ermittelten Wert. Dies ist also ziemlich groß und somit ist sogar
die Abweichung vom Literaturwert um $\SI{16.67(778)}{\percent}$ deutlich erhöht. Die erste Methode ist also durch die Verwendung eines größeren Messintervalls, trotz kleiner
Abweichungen zur Untergrundrate bei höheren Zeiten $t$, genauer.
\\
\newline
Bei der Halbwertszeit für Rhodium stellen die zwei unterschiedlichen Zerfallsvorgänge eine Komplikation dar. Durch die relativ willkürliche Wahl eines Zeitpunkts $t^{*}$ werden
automatisch Fehler erzeugt. Die prozentuale Abweichung vom Literaturwert für den Zerfall des Zustands $\ce{^{104}_{45}Rh}$ beträgt dabei $\SI{7.16}{\percent}$ und für den 
isomeren Zustand $\ce{^{104{\text{i}}}_{45}Rh}$ bereits $\SI{17.05}{\percent}$. 
Die etwas größer werdende Abweichung bei Rhodium war allerdings zu erwarten, durch die verwendete Methode der Auswertung, deshalb sind die Ergebnisse durchaus plausibel.
