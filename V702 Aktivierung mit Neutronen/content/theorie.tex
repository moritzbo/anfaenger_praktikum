\section{Theoretische Grundlagen}

Bei einem stabilen Atomen ist das Verhältnis zwischen der Neutronen und Protonenanzahl im Kern meist in engen Grenzen.
Wenn das Verhältnis der beiden Teilchen diese Grenze allerdings überschreitet wird der vorliegende Atomkern instabil und er zerfällt in 
unterschiedlichen Zerfallsprozessen in ein stabiles Atom. Je nach instabilem Atom und Anzahl der Protonen und Neutronen findet ein anderer
Zerfallsprozess statt.
\\
Eine wichtige zu betrachtende Größe in der Kernphysik ist die Halbwertszeit $T$, sie entspricht dem Zeitraum $\increment t$ in der eine Anzahl von instabilen
Kernen gerade um die Hälfte zerfallen ist. Für die Bestimmung der Halbwertszeit existieren unterschiedliche Methoden, im Folgenden soll 