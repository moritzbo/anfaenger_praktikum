\section{auswertung}

\subsection{Nulleffekt}
\begin{equation*}
N_U= \{ 129, 143, 144, 136, 139, 126, 158 \}
\end{equation*}
Die Fehler der gemessenen Werte des Nulleffekts werden jeweils durch eine Poisson Verteilung herausgefunden.
Da es eine Vielzahl an Messungen gibt, werden die resultierenden Fehler anschließend gemittelt.
Nach der Poisson Verteilung sieht der Fehler auf $N$ aus wie folgt
\begin{align}
\label{eqn:mittel}
\Delta N &= \sqrt{N}
\intertext{Diesen gilt es wiederum über die Anzahl der Messungen zu mitteln}
 \bar{\Delta N} &= \frac{1}{n} \sum_{i=1}^n \Delta N_i% BAR MUSS GRÖ?ER
\intertext{mit ensprechende werten (siehe Anhang) ergibt sich als Mitteltwert der Poission Verteilung und somit als finalen Wert des Nulleffekts} % idk obd as schön ist, bzw ob das geht
\bar{\Delta N} & = \si{139}{\pm4}
\end{align}
\\
\paragraph{mittelung der messwerte von Vanadium} \mbox{}\\
Analog zu \eqref{eqn:mittel} lässt sich durch die Poisson Verteilung der Fehler auf  die Werte bestimmnen. Die genauen Ergbnisse befinden sich im Anhang.


\begin{figure}
  \centering
  \includegraphics[width=0.8\textwidth]{build/plot1.pdf}
  \caption{Aufbau  \cite{hinweis}.}
  \label{fig:aufbau1}
\end{figure}


VANDAIUM WANN INS ZÄHLROHR IN DURCHFÜHRUNG PLUS RODIUM