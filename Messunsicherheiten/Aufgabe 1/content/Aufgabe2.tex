\section{Aufgabe 2}

\subsection{Fragestellung}
Nach mehrmaliger Messung der Schallgeschwindigkeit $u$ zieht ein Student den
Schluß, dass die Standardabweichung $\sigma_u$ der Meßwerte $\sigma_u = 10 \si{\meter\per\second}$ ist. Wir nehmen
an, die Abweichungen seien zufällig. Dann kann der Student jede beliebige
Präzision erreichen, indem er genug Messungen durchführt und deren Mittelwert
berechnet. Wie viele Messungen sind nötig, damit die entgültige Unsicherheit
$\pm 3 \si{\meter\per\second}$ beträgt? Wie viele für eine Unsicherheit von nur $0{,}5 \si{\meter\per\second}$?

\subsection{Antwort}

Gegeben ist also die Standardabweichung $\sigma_u$ und jeweils ein Fehler auf dem Mittelwert $\bar{x}$.
Gesucht wird die Anzahl der Messungen die benötigt werden um den Fehler $\increment\bar{x}$ auf ein bestimmtes Intervall zu reduzieren.

\begin{flushleft}
Dies lässt sich über die Formel zur Standardabweichung des Mittelwerts \eqref{eqn:seoftm} berechnen.

Wenn man $\increment\bar{x} = 3 \si{\meter\per\second}$ setzt erhällt man
\end{flushleft}
\begin{equation}
3\si[per-mode=symbol]{\meter\per\second} = \frac{\SI{10}{\meter\per\second}}{\sqrt{N}}
%Ist es notwendig das als Eq anzugeben???
\end{equation}

\begin{flushleft}
Dies lässt sich nach $\sqrt{N}$ umstellen und quadrieren. Es folgt
\end{flushleft}
\begin{equation}
N = \left( \frac{10}{3} \right)^2 = \frac{100}{9} = 11{,}\bar{1}
\end{equation}

\begin{flushleft}
Da nur ganzzahlige Werte sinnvoll sind braucht man mindestens $N = 12$ Messungen, um 
den Mittelwert $\bar{x}$ der Schallgeschwindigkeit auf $\pm 3 \si{\meter\per\second}$ genau zu bestimmen.

Für $\increment\bar{x} = 0{,}5$ folgt analog
\end{flushleft}
\begin{equation}
N = \left( \frac{10}{0{,}5} \right)^2 = 400
\end{equation}

\begin{flushleft}
Man benötigt also genau $N = 400$ Messungen um den Fehler auf $\pm 0{,}5 \si{\meter\per\second}$ zu reduzieren. 
\end{flushleft}