\section{Aufgabe 1}

\subsection{Mittelwert}
Der Mittelwert beschreibt den Durchschnitt einer Anzahl an Zahlen/Werten, die z.B durch 
eine Messung entnommen wurden. Bei gleichbleibender Verteilung gibt der Mittelwert also den Wert 
mit der größten Häufigkeit an. Dieser Wert ist allerdings lediglich ein Schätzwert und wird umso
genauer, je mehr Messwerte man einbezieht.
%evt nicht gut formuliert

\begin{flushleft}
Der Mittelwert $\bar{x}$ bzw. das arithmetische Mittel wird wie folgt berechnet
\end{flushleft}

\begin{equation}
\bar{x} = \frac{1}{N} \sum_{k=1}^N x_k
\label{eqn:mean}
\end{equation}

\begin{flushleft}
Dabei gibt $N$ die Gesamtanzahl und $x_k$ jeweils die einzelnen Messwerte an.
\end{flushleft}

\subsection{Standardabweichung}
Die Standardabweichung $\sigma$ gibt an wie weit die einzelnen Werte im Durchschnitt
vom Mittelwert $\bar{x}$ entfernt sind. Sie ist über die Varianz definiert welche die mittlere 
quadratische Abweichung vom Mittelwert angibt. 

\begin{flushleft}
Die Varianz lässt sich wie folgt berechnen
\end{flushleft}
\begin{equation}
\symup{Var} = \frac{1}{N-1} \sum_{k=1}^N (x_k - \bar{x})^2
\label{eqn:varianz}
\end{equation}

\begin{flushleft}
Die Dimension der Varianz \eqref{eqn:varianz} stimmt allerdings nicht mit der des Mittelwerts \eqref{eqn:mean} überein,
deshalb wird die Standardabweichung eingeführt. 
%nicht sicher mit den eqrefs ob notwendig
\end{flushleft}

\begin{equation}
\sigma = \sqrt{\symup{Var}} = \sqrt{\frac{1}{N-1} \sum_{k=1}^N (x_k - \bar{x})^2}
\label{eqn:sigma}
\end{equation}
\subsection{Streuung und Fehler des Mittelwerts}
\begin{flushleft}
Bei der Streuung der einzelnen Messwerte handelt es sich um eine Verteilung um den zuvor bestimmten Mittelwert \eqref{eqn:mean}. 
Diese Verteilung wird durch die Standardabweichung \eqref{eqn:sigma} beschrieben und ist nicht gleichbedeutent zu dem Fehler des Mittelwerts.
Bei der Berechnung des Mittelwerts an einer abzählbaren Menge an Werten kommt es zu einer Abweichung vom tatsächlichen Wert.
Um den Bereich in dem sich der Mittelwert befindet möglichst genau zu berechnen, bestimmt man den Fehler des Mittelwerts. Dieser wird
auch Standardabweichung des Mittelwerts $\increment\bar{x}$ genannt und wird wie folgt berechnet
\end{flushleft}

\begin{equation}
\increment\bar{x} = \frac{\sigma}{\sqrt{N}} = \sqrt{\frac{1}{N(N-1)} \sum_{k=1}^N (x_k - \bar{x})^2}
\label{eqn:seoftm} %seoftm = standard error of the mean
\end{equation}